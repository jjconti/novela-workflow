\documentclass[11pt,twoside,openright]{book}

\usepackage[papersize={120mm,200mm}, hmarginratio={3:2},bottom=0.8in,top=0.8in]{geometry}

\usepackage{times}
\usepackage[Lenny]{fncychap}
%\usepackage[Conny]{fncychap}

\usepackage[spanish]{babel}
\usepackage[utf8]{inputenc}
\usepackage[T1]{fontenc}

\newcommand{\personaje}{Nombre del personaje}

\usepackage{mathptmx}
\usepackage{etoolbox}

\usepackage[titles]{tocloft}

\usepackage{pdfpages}

\renewcommand{\cftchapleader}{\cftdotfill{\cftdotsep}}

% change the space before the titles
\makeatletter
\patchcmd{\@makechapterhead}{\vspace*{50\p@}}{\vspace*{0pt}}{}{}
\patchcmd{\@makeschapterhead}{\vspace*{50\p@}}{\vspace*{0pt}}{}{}
\makeatother

% change the space after the titles
\renewcommand{\DOTI}[1]{%
    \raggedright
    \CTV\FmTi{#1}\par\nobreak
    \vskip 10pt}% original: 40pt
\renewcommand{\DOTIS}[1]{%
    \raggedright
    \CTV\FmTi{#1}\par\nobreak
    \vskip 10pt}% original: 40pt


\title{Carne de los dioses}
\author{Juanjo Conti}
\date{}


% Evitar viudas y huérfanas
\widowpenalty=10000
\clubpenalty=10000

\begin{document}

\pagestyle{plain}

%IMPRENTA
\includepdf{empty.pdf}
%IMPRENTA
\includepdf{empty.pdf}

%\maketitle
\includepdf{carne_de_los_dioses__bn.pdf}

%\cleardoublepage

\thispagestyle{empty}
\noindent
Edición automágica, 2016.\\

\vspace{0.5cm}

\noindent
\emph{Carne de los dioses} lleva la licencia
\emph{Creative Commons Attribution - NonCommercial - ShareAlike 4.0 Iternational}.
Esto significa que podés compartir esta obra y crear obras derivadas
mencionando al autor, pero no ha\-cer un uso comercial de ella.

\vfill

\noindent
%Más información sobre este libro:\\
http://www.juanjoconti.com/carne\\

\noindent
%Más libros del autor:\\
http://www.juanjoconti.com/libros

\cleardoublepage

\noindent
\begin{flushright}
\emph{
\emph{Carne de los dioses}\\
está dedicado a mis compañeros del taller\\
\emph{El brillo de la palabra},\\
testigos del nacimiento de muchos de estos cuentos.}
\end{flushright}

\cleardoublepage

\renewcommand*\contentsname{Índice}

\tableofcontents

\chapter*{Prólogo}

\begin{em} Armar un libro de cuentos es cansador y doloroso. Y no me refiero a
escribir los cuentos. Me refiero al trabajo posterior a esto.

Pasó un año o dos. Estuviste escribiendo. Asististe a un taller y cumpliste con
muchas de las consignas. Intentaste escribir una novela por tu cuenta, o dos, o
tres. Y llega el verano y, como además de escribir te gusta editar, te decís:
“Este verano imprimo un libro”.

Más o menos por el mismo tiempo, pero unos meses antes, te enteraste de un
concurso de cuentos. Tema: el agua. Hacés una búsqueda en tus archivos y
resulta que sí, que tenés un cuento, en el cual el agua, si bien no tiene un
papel principal, es el telón de la historia. \emph{Carne de los dioses}. Lo
escribiste hace ¿un año? como una broma para un grupo de amigos. Después le
recortaste las partes específicas y lo hiciste circular como un cuento más,
pero nunca lo imprimiste. Lo arreglás y lo mandás al concurso. No te gustan los
concursos por la burocracia que encierran: imprimir tres copias, enviarlas por
correo postal... Pero este es distinto. La entrega es por correo electrónico;
así que lo mandás. Una de las condiciones es que el cuento sea inédito. Este
seguro lo es... al menos, al momento de enviarlo.

Hay una imagen en el cuento que no recordabas, pero al releerlo, te impacta. Un
tigre. Un tigre con un collar de flores. Te gusta la imagen y le pedís a una
amiga que te haga un dibujo. Ella es tan amable que lo hace y el resultado es
genial. Tanto que te inspira a armar un libro de cuentos solo para que ese
dibujo esté en la tapa. El único problema es que el dibujo es en blanco y negro
y vos querés una tapa a color. La artista se niega a pintarlo, por lo que
decidís colorearlo por tu cuenta. Según tu esposa, parece pintado por un nene
de tres años, pero a vos te gusta. Le pedís a tu amigo diseñador que arme una
tapa con ese dibujo.

Después, llega  el momento de seleccionar los cuentos que integrarán el libro.
Reunís los textos; tu correctora amiga los corrige. Los mirás de un lado y del
otro. ¿Cómo agruparlos? ¿Cómo lograr un conjunto, un orden? Ninguna selección
reúne suficientes ejemplares. La de ribetes fantásticos tiene cinco textos. La
“costumbrista”, siete. Ensayás distintos índices; sacás, ponés, reordenás. Cada
vez que regenerás el libro, intentás mirarlo con ojos nuevos, pero no podés, ya
estás viciado.

Entonces, lo abandonás, lo dejás “macerar”. Te olvidás del proyecto por unas
semanas. Unas semanas que parecen meses. Luego volvés. Volvés y terminás el
trabajo. Te sentás en la silla y terminás el libro, porque yo quiero recibirlo
en mis manos.

\end{em}

\input{romantico.txt}

\input{amor.txt}

\input{guido.txt}

\input{carta.txt}

\input{explicacion.txt}

\input{carne.txt}

\input{dieta.txt}

\input{reyes.txt}

\input{lavalle.txt}

\input{matecocido.txt}

\input{fin.txt}

\input{restaurador.txt}

\input{vikingo.txt}

\input{dijkstra.txt}

\cleardoublepage

{\small
\section*{Algunos detalles sobre este libro}

\begin{enumerate}
\item El dibujo de la portada es una obra de Flor Soler.
\item El diseño de la portada es de Gonzalo “Kun” Vega.
\item Lis Gariglio tuvo la amabilidad de ayudarme a hacer una corrección integral del libro. Si quedó algún error, es mi culpa por no haber atendido a alguna de sus múltiples y pacientes devoluciones.
\item Escribí \emph{El primer romántico} con amor para mi esposa.
\item Creo que \emph{El viajero del cielo} es el más viejo de los textos incluidos. Los hechos que  relato en él transcurrieron en 2010.
\item Los detalles técnicos de carnicería que aparecen en \emph{El asado de los Reyes} son un aporte de Sebastián D. Criado, vestigios de otra versión de ese cuento.
\item Los protagonistas de \emph{Carne de los dioses} son Lucho, Martín y Charly.
\item Dos de los cuentos presentados son capítulos de novelas que aún no escribí.
\item Le agradezco a Sofía Storani su participación en \emph{La maravillosa vida de Henry Dijkstra}. Sin su valiosa opinión, el cuento no sería lo que es.
\item Los agradecimientos de siempre a César y Joel que se toman el trabajo de leer absolutamente todo lo que escribo.
\item Compraron este libro antes de que esté impreso Leonardo Martínez, Hernán Perotti, Cristian Bozzo, Pablo Cambria, Selva Mendoza y Leandro Rojas.
\item Agradecimiento tardío; Manuel Kaufmann compró otro de mis libros en preventa y no le di las gracias.
\end{enumerate}
}


\cleardoublepage

\setcounter{footnote}{0}

\section*{También publicado por Automágica}

\begin{itemize}
\item \emph{Xolopes} (novela)\footnote{http://www.juanjoconti.com/xolopes/}
\item \emph{Santa Furia} (cuentos)\footnote{http://www.juanjoconti.com/santa-furia/}
\item \emph{La prueba del dulce de leche} (cuentos)\footnote{http://www.juanjoconti.com/la-prueba/}
\end{itemize}

%IMPRENTA
\includepdf{empty.pdf}

\end{document}
