\documentclass[12pt,twoside,openright,a5paper]{book}

\usepackage[a5paper, hmarginratio={3:2}]{geometry}

\usepackage{times}
\usepackage[Lenny]{fncychap}
%\usepackage[Conny]{fncychap}

\usepackage[spanish]{babel}
\usepackage[utf8]{inputenc}
\usepackage[T1]{fontenc}

\newcommand{\personaje}{Nombre del personaje}

\usepackage{mathptmx}
\usepackage{etoolbox}

\usepackage[titles]{tocloft}

\usepackage{pdfpages}

\renewcommand{\cftchapleader}{\cftdotfill{\cftdotsep}}

% change the space before the titles
\makeatletter
\patchcmd{\@makechapterhead}{\vspace*{50\p@}}{\vspace*{0pt}}{}{}
\patchcmd{\@makeschapterhead}{\vspace*{50\p@}}{\vspace*{0pt}}{}{}
\makeatother

% change the space after the titles
\renewcommand{\DOTI}[1]{%
    \raggedright
    \CTV\FmTi{#1}\par\nobreak
    \vskip 10pt}% original: 40pt
\renewcommand{\DOTIS}[1]{%
    \raggedright
    \CTV\FmTi{#1}\par\nobreak
    \vskip 10pt}% original: 40pt


\title{Remembranzas de una familia peculiar}
\author{Juanjo Conti}
\date{}


% Evitar viudas y huérfanas
\widowpenalty=10000
\clubpenalty=10000

\begin{document}

\pagestyle{plain}

\maketitle

%\cleardoublepage

\thispagestyle{empty}
\noindent
Edición automágica, 2016.\\

\vspace{0.5cm}

\noindent
\emph{Remembranzas de una familia peculiar} lleva la licencia
\emph{Creative Commons Attribution - NonCommercial - ShareAlike 4.0 International}.
Esto significa que podés compartir esta obra y crear obras derivadas
mencionando al autor, pero no ha\-cer un uso comercial de ella.

\vfill

\noindent
%Más información sobre este libro:\\
http://www.juanjoconti.com.ar/pico\\

\noindent
%Más libros del autor:\\
http://www.juanjoconti.com.ar/libros

\cleardoublepage

\noindent
\begin{flushright}
\emph{
\emph{Remembranzas de una familia peculiar}\\
está dedicado a Adrián, Flavio, Joel y Tristan,\\
que no solo leyeron estos cuentos segundos después\\
de que terminé su primer borrador, sino que\\
proporcionarion datos para varios de ellos.
}
\end{flushright}

\cleardoublepage

\renewcommand*\contentsname{Índice}

\tableofcontents

\chapter*{Prólogo}

\input{pico0.txt}

\input{bromista.txt}

\input{lavalle.txt}

\input{pico1.txt}

\input{peluqueria.txt}

\input{explicacion.txt}

\input{pico2.txt}

\input{matecocido.txt}

\input{dieta.txt}

\input{camionero.txt}

\input{pico3.txt}

\input{reyes.txt}

\includepdf{empty.pdf}

\end{document}
