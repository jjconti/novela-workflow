\documentclass[11pt,twoside,openright,a6paper]{book}

\usepackage[a6paper]{geometry}

\usepackage{times}
\usepackage[Lenny]{fncychap}
%\usepackage[Conny]{fncychap}

\usepackage[spanish]{babel}
\usepackage[utf8]{inputenc}
\usepackage[T1]{fontenc}

\newcommand{\personaje}{Nombre del personaje}

\usepackage{mathptmx}
\usepackage{etoolbox}

\usepackage[titles]{tocloft}

\usepackage{fourier-orns}

\renewcommand{\cftchapleader}{\cftdotfill{\cftdotsep}}

% change the space before the titles
\makeatletter
\patchcmd{\@makechapterhead}{\vspace*{50\p@}}{\vspace*{0pt}}{}{}
\patchcmd{\@makeschapterhead}{\vspace*{50\p@}}{\vspace*{0pt}}{}{}
\makeatother

% change the space after the titles
\renewcommand{\DOTI}[1]{%
    \raggedright
    \CTV\FmTi{#1}\par\nobreak
    \vskip 10pt}% original: 40pt
\renewcommand{\DOTIS}[1]{%
    \raggedright
    \CTV\FmTi{#1}\par\nobreak
    \vskip 10pt}% original: 40pt


\title{Xolopes}
\author{Juanjo Conti}
\date{}

\hyphenation{per-te-ne-cien-tes}
\hyphenation{ín-di-ce}
\hyphenation{co-ra-zón}
\hyphenation{via-je-ro}
\hyphenation{vi-kin-go}
\hyphenation{a-tra-ve-só}
\hyphenation{dio-ses}
\hyphenation{fe-rry}
\hyphenation{ca-mi-nan-do}
\hyphenation{e-le-va-do}
\hyphenation{ba-ta-lla}
\hyphenation{ber-me-llón}
\hyphenation{bo-te-llas}
\hyphenation{pro-ta-go-nis-ta}
\hyphenation{ar-chi-pié-la-go}
\hyphenation{li-bre-rí-a}
\hyphenation{em-pe-za-ba}
\hyphenation{a-cos-tar-me}
\hyphenation{re-co-rrer}
\hyphenation{ma-yas}
\hyphenation{re-po-llo}
\hyphenation{per-dien-do}
\hyphenation{me-tros}
\hyphenation{di-si-mu-lar}


\begin{document}

\pagestyle{plain}

\maketitle

\cleardoublepage

\thispagestyle{empty}
\noindent
Edición automágica. 2013.\\
La tapa de este libro es material reciclado.

\vspace{0.5cm}

\noindent
\emph{Xolopes} lleva la licencia 
Creative Commons Attribution - NonCommercial - ShareAlike 3.0 Unported License.
Esto significa que podés compartir esta obra y crear obras derivadas de la misma
mencionando al autor, pero no ha\-cer un uso comercial.

\vfill

\noindent
Más información sobre este libro:\\
http://www.juanjoconti.com.ar/xolopes\\

\noindent
Más libros del autor:\\
http://www.juanjoconti.com.ar/libros

\cleardoublepage

\noindent
\begin{flushright}
\emph{
\emph{Xolopes} está dedicado a mi papá y a mi mamá\\
que me enseñaron a leer y a amar.\\
Amo, por eso escribo.
}
\end{flushright}

\cleardoublepage


\vspace{0.5cm}
\hrulefill\hspace{0.2cm} \decofourleft\decofourright \hspace{0.2cm} \hrulefill
\vspace{0.5cm}

Conté una ficción en lugar de la realidad para proteger a las personas.


\vspace{0.5cm}
\hrulefill\hspace{0.2cm} \decofourleft\decofourright \hspace{0.2cm} \hrulefill
\vspace{0.5cm}

Algunos un día antes no saben a dónde van a viajar. Entran a un sitio
en Internet, elijen la mejor oferta, manejan hasta el aeropuerto y buen
viaje. Otros, más previsores, nos pasamos tres meses mirando mapas y
tenemos el viaje pagado hace cinco.


\vspace{0.5cm}
\hrulefill\hspace{0.2cm} \decofourleft\decofourright \hspace{0.2cm} \hrulefill
\vspace{0.5cm}

SFE - ROS - EZE - MEX - CUN - PLA


\vspace{0.5cm}
\hrulefill\hspace{0.2cm} \decofourleft\decofourright \hspace{0.2cm} \hrulefill
\vspace{0.5cm}

¿Y si vamos a un hostel? Estás loco, me dijo mi amiga, si vamos a Cancún,
vamos a un All Inclusive.

Hay ocasiones en las que los términos medios parecen no existir. O frío
o caliente, a los tibios los escupiré de mi boca, me citaba a Jesús un
amigo testigo de Jehová cada vez que me ponía duditativo entre ir o no
a jugar a la pelota.


\vspace{0.5cm}
\hrulefill\hspace{0.2cm} \decofourleft\decofourright \hspace{0.2cm} \hrulefill
\vspace{0.5cm}

\hyphenation{a-pa-rez-ca}
\hyphenation{lue-go}
Auto, bondi o transporte, de una forma u otra, todos llegamos al
aeropuerto. Una vez ahí, a armarse de paciencia. Esperar que el vuelo
aparezca en las pantallas, que no esté retrasado, ir al mostrador,
despachar las valijas, ¿no me habré excedido en el peso? 16 kg, me sobró
lugar. Luego a esperar nuevamente, preembarque, embarque y ahora sí. Estoy
sentado dentro del avión. Una mole de miles de toneladas de acero alrededor
mío. Yo sentado en un reducto ínfimo, incómodo. Miro para los costados y
los demás parecen estar en otro mundo. Algunos juegan con sus teléfonos,
otros miran la pantalla o leen una revista. ¿Cómo pueden estar tan
tranquilos? ¿No se dan cuenta de que en menos de un minuto el capitán va
a encender los motores o ya los tiene encendidos pero los va a utilizar y
vamos a pasar todos de estar en la seguridad de tierra firma a estar en la
nebulosa, en un limbo? Ahí pasó una azafata y me pidió que me ajuste el
cinturón y ponga derecho mi asiento. Me empiezan a transpirar las manos. El
avión se ubica en la punta de la pista de despegue. Empieza a carretear. No
puedo evitar sentirme en el lomo de un pterodáctilo que corre por las
pista. Aguanto la respiración. Repaso las oraciones de un rosario como si
quisiera ametrallar al pelado de adelante con mis palabras. El corazón me
late. Y ahora, de un momento para otro, dejo de sentir el rugoso asfalto
bajo las ruedas del avión que ya se han despegado y lentamente, imagino,
vuelven a formar parte de la mole de acero. Suena una campanita y se apaga
el cartel luminoso que indicaban que nadie se podía levantar. Hemos pasado
los 10000 metros de altura, anuncia el capitán.


\vspace{0.5cm}
\hrulefill\hspace{0.2cm} \decofourleft\decofourright \hspace{0.2cm} \hrulefill
\vspace{0.5cm}

Where are you from, girls?

Argentina.

Where is that, somewhere in Mexico?


\vspace{0.5cm}
\hrulefill\hspace{0.2cm} \decofourleft\decofourright \hspace{0.2cm} \hrulefill
\vspace{0.5cm}

Al final fuimos a un All inclusive. Una pulcerita, comida y bebida gratis, en
abundancia, todo el día, todos los días. Yo me consolaba solo diciéndome
que así ya tenía resuelto el tema de la comida. Soy celíaco, intolerante
a la lactosa, astemio y tengo colon irritable.


\vspace{0.5cm}
\hrulefill\hspace{0.2cm} \decofourleft\decofourright \hspace{0.2cm} \hrulefill
\vspace{0.5cm}

Vas caminando por un bosque oscuro, sentís una presencia a tu espalda
y en lugar de darte vuelta empezás a correr y a correr. Te tropezás,
caés, te levantás y seguís corriendo. La presencia atrás se hace cada
vez más presente. Adelante ves un precipicio, no tenés tiempo siquiera
para pensarlo, seguís corriendo hasta que saltás. Mientras caés te
despertás. Estás en el asiento del avión. Otra vez te dormiste y otra vez
te despertaste. Miras el reloj, es la tercera vez en la última hora. Notás
movimiento unas filas más adelante, un hombre grande y rubio se mueve  como
si lo estuvieran electrocutando. Las personas de los asientos contiguos
le levantan asustadas. Más movimiento. Una azafata te pasa por al lado
corriendo. Escuchás al capitán preguntar si hay un médico a bordo. El
viejito sentado dos lugares más allá se levanta. Ahora el hombre rubio
está tirado en el piso del avión, alguien le practica técnicas de
reanimación. Ahora ves como arrastran al hombre rubio hasta detrás de
unas cortinas.


\vspace{0.5cm}
\hrulefill\hspace{0.2cm} \decofourleft\decofourright \hspace{0.2cm} \hrulefill
\vspace{0.5cm}

Lo lindo de tomar excursiones es que te sale más caro que si vas por tu
cuenta, te buscan a las 7 am pero hasta las 10 no dejan la ciudad (hay que
buscar al resto de los excursionistas) y una vez en el destino no tenés
tiempo de recorrer tranquilo porque el guía te va ametrallando datos que
no vas a recordar a la salida del lugar. No.


\vspace{0.5cm}
\hrulefill\hspace{0.2cm} \decofourleft\decofourright \hspace{0.2cm} \hrulefill
\vspace{0.5cm}

Leído en Wikipedia: \emph{Chichén Itzá (maya: (Chichén)
Boca del pozo; de los (Itzá) brujos de agua) es uno de los principales
sitios arqueológicos de la península de Yucatán, en México, ubicado en
el municipio de Tinum, en el estado de Yucatán.}


\vspace{0.5cm}
\hrulefill\hspace{0.2cm} \decofourleft\decofourright \hspace{0.2cm} \hrulefill
\vspace{0.5cm}

Ayer de pasada escuché a dos mozos hablar sobre el agujero de la cerradura de
una de las habitaciones. No alcancé a escuchar exactamente a qué habitación
se referían, pero decodifique un “setecientos cuarenta y...” y vi que uno
de los dos hacía gestos obscenos mientras el otro se reía a carcajadas. Sin
nada mejor que hacer un día nublado y con lloviznas en toda la península de
Yucatán, dejé el área de desayuno y me fui caminando hasta el bloque siete.


\vspace{0.5cm}
\hrulefill\hspace{0.2cm} \decofourleft\decofourright \hspace{0.2cm} \hrulefill
\vspace{0.5cm}

Volvés a escuchar la voz del capital. Les habla su capitán, hay una
persona descompuesta y no estamos pudiendo ayudarla por lo que vamos a
lanzar el combustible al mar y aterrizar en Recife. Sentís la fuerza del
avión regresando. Media hora después aterrizan. Más movimiento. No pasan
cinco minutos cuando volvés a oír al capital. Les habla su capitán,
lamentablemente tengo que informarles que el pasajero murió.


\vspace{0.5cm}
\hrulefill\hspace{0.2cm} \decofourleft\decofourright \hspace{0.2cm} \hrulefill
\vspace{0.5cm}

Una vez en el bloque siete, subí las escaleras hasta el piso cuatro y,
mirando el amplio pasillo, me puse a pensar cuál de las puertas sería el
custodio del entretenimiento de los mozos. Sin nada que me de una señal,
empecé a apoyar mi oreja en cada una de las puertas.

Estaba apoyado bajo un lustroso número 744 cuando la superficie que me
sostenía se esfumó. La puerta se había abierto y yo caí, despatarrado
y haciendo mucho ruido, a los pies de una señorita de zapatos rojos y
taco aguja.


\vspace{0.5cm}
\hrulefill\hspace{0.2cm} \decofourleft\decofourright \hspace{0.2cm} \hrulefill
\vspace{0.5cm}

\hyphenation{bra-si-le-ño}
\hyphenation{e-xi-ge}
Pasó una hora y el clima en el avión es tenso. Te preguntás qué mierda
estará pasando cuando volvés a escuchar la voz del capitán. Les habla
su capitán, tengo que informarles que se ha encontrado cocaína en las
pertenencias del pasajero y la policía local no nos permite continuar el
viaje hasta realizar los peritajes necesarios en el avión. Ves a un padre
con su bebé en brazos a las puteadas. Lo mandan a primera clase, igual que
hicieron con los que estaban en la misma fila que el hombre rubio. Suben
oficiales de policía al avión, sacan fotos al asiento del pasajero muerto,
toman muestras y por último se llevan la funda del asiento. Pasan dos
horas más y el capitán anuncia que aún no han conseguido combustible para
reponer el volcado en el océano. Es probable que tengan que dejar el avión,
pasar la noche en un hotel y continuar al día siguiente. Un brasileño se
ríe. Es que, te cuenta, Brasil le exige visa a los norteamericanos igual
que Estados Unidos le exige visa a los brasileños. Los van a hacer renegar
un buen rato, te dice.


\vspace{0.5cm}
\hrulefill\hspace{0.2cm} \decofourleft\decofourright \hspace{0.2cm} \hrulefill
\vspace{0.5cm}

\hyphenation{in-sig-ni-fi-can-te}
\hyphenation{re-cep-cio-nis-tas}
\hyphenation{be-ach}
\hyphenation{mo-chi-la}
De toda la fauna humana que se puede ver rebotando por Playa del Carmen, uno
de los especímenes más curiosos son los mochileros. Hombres o mujeres,
en su mayoría de entre 20 y 30 años (aunque hay excepciones) que se
trasladan de una ciudad a otra, de un país a otro o de un continente a
otro, cargando en sus espaldas todo lo que necesitan para subsistir. Estos
caracoles humanos se caracterizan por intentar gastar la menor cantidad de
dinero posible a la vez que maximizan las experiencias vividas. Su lugar
predilecto para pernoctar son los hostels en el centro de la ciudad, pero
si alguien les ofrece techo a cambio de trabajo o algún tipo de acuerdo
similar, no dudan en aceptar.

Si los clasificamos por su modo de viajar, podemos distinguir dos grupos
principales: los ahorristas y los buscavida. El primer grupo suele viajar
durante periodos de tiempo más cortos y cuando se les termina el dinero
ahorrado para el viaje, vuelven a su lugar de origen, trabajan por un
tiempo determinado y vuelven a salir a la aventura. El segundo grupo está
formado por personas que alguna vez fueron ahorristas pero se cansaron
de tener que interrumpir constantemente el viaje de la vida por algo tan
insignificante como es el dinero. Es por eso que en un punto del camino
deciden no regresar y encuentran formas alternativas de vivir, bajan sus
expectativas de confort y sus travesías duran años. Los hay malabaristas,
vendedores, recepcionistas...

¿Yo? No, yo estoy parando en el Ocean Beach Caribean Resort. No, nunca
salí con la mochila, lo que te estoy contando lo vi en un documental.


\vspace{0.5cm}
\hrulefill\hspace{0.2cm} \decofourleft\decofourright \hspace{0.2cm} \hrulefill
\vspace{0.5cm}

La chica que me miraba desde su altura más los veinte centímetros de
los tacos me empezó a hablar en una lengua que no reconocí. Me levantó
del hombro (era muy fuerte, sumado a la pista del idioma, adivino rusa o
ucraniana, medallista olímpica de lanzamiento de martillo, ¿por qué
no?). Evidentemente se pensó que yo era personal de maestranza o algo
parecido, porque con unas palabras que adiviné como insultos me señalaba
su reloj y a los empujones me llevó hasta el baño de la habitación donde
cataratas de agua emanaban por todos los elementos de grifería.


\vspace{0.5cm}
\hrulefill\hspace{0.2cm} \decofourleft\decofourright \hspace{0.2cm} \hrulefill
\vspace{0.5cm}

Les habla su capitán, les informo que hemos podido reabastecernos
del combustible necesario para continuar el viaje, en cuarenta minutos
iniciaremos el despegue. Ya pasaron seis horas desde el aterrizaje y es
de noche. La mayoría de los pasajeros están dormidos y vos, desvanecido,
te dormís cuando el avión está despegando.


\vspace{0.5cm}
\hrulefill\hspace{0.2cm} \decofourleft\decofourright \hspace{0.2cm} \hrulefill
\vspace{0.5cm}

\hyphenation{chan-gas} \hyphenation{des-en-ros-can-do} Acostumbrado a
arreglármelas y a hacer changas, me arremangué para revisarle la cañería a
la señorita. Por favor absténganse de malos pensamientos que esta no es ese
tipo de historias. Estuve escarbando, destrabando, soplando, desenroscando,
girando, palpando, doblando, estirando, sacudiendo, estrujando y reensamblando
alrededor de media hora hasta que en una de las secciones de la cañería
encontré, atascado, cien kilos de papel higiénico y descartables de aseo
personal varios condensados en una pelota de diez centímetros de diámetro.

Como un héroe que rescata una mascota, se la mostré a la ucraniana (ya la
había bautizado así en mis pensamientos) y me hizo cara de asco. La tiré
en el cesto de la basura y procedía a lavarme las manos, los antebrazos,
los codos, los brazos y hasta la nuca, era una limpieza sin fin. Entendí a
los pelados cuando se lavan la cara. Una vez presto y rechinando de limpio,
volví mi atención a la señorita. El agua ya no brotaba a borbotones y su
pesadilla cloacal parecía llegar a su fin. Me tomó de las manos y vi en
sus ojos el deseo de agradecerme. Casi lágrimas le provocaba la emoción
y la impotencia de no poder hablarme en mi idioma natal. Miró con el
rabillo del ojo la cama matrimonial y a mi se me aceleró el corazón. En
ese momento escuché la llave, el picaporte y la puerta.


\vspace{0.5cm}
\hrulefill\hspace{0.2cm} \decofourleft\decofourright \hspace{0.2cm} \hrulefill
\vspace{0.5cm}

\hyphenation{pa-sa-je-ro}
El Folha de Brasil informa. Africano muere durante vuelo con 104 cápsulas
de cocaína en el estómago.  El sudafricano Louis Hendrik Smith, de
34 años murió esta tarde durante un vuelo entre Argentina y México,
supuestamente por sobredosis de cocaína.  Según la Policía Federal, Smith
había ingerido 1,7 kg de droga, divididos en 104 cápsulas. Una de ellas,
con cerca de 16 g de sustancia, se rompió en el estómago del pasajero.

Luego de un aterrizaje de emergencia, el pasajero fue atendido por dos
médicos pero estos no pudieron hacer nada porque ya estaba muerto.

Las autoridades policiales retiraron el equipaje del pasajero del avión y
liberaron a los restantes 308 pasajeros y 17 tripulantes para que continúen
su viaje a las 12 de la noche.  Según la superintendencia de la Policía
Federal, todo indicaría que la víctima compró e ingirió la droga en
Argentina.


\vspace{0.5cm}
\hrulefill\hspace{0.2cm} \decofourleft\decofourright \hspace{0.2cm} \hrulefill
\vspace{0.5cm}

\hyphenation{hu-ma-ni-dad}
La zona arqueológica de Chichén Itzá fue inscrita en la lista del
Patrimonio de la Humanidad por la Unesco en 1988. El 7 de julio de 2007, fue
reconocida como una de las Las nuevas siete maravillas del mundo moderno, por
una iniciativa privada sin el apoyo de la Unesco, pero con el reconocimiento
de millones de votantes alrededor del mundo.

Desde ese día no permiten más subirla.


\vspace{0.5cm}
\hrulefill\hspace{0.2cm} \decofourleft\decofourright \hspace{0.2cm} \hrulefill
\vspace{0.5cm}

No miento si digo que podría ser doble de riesgo de Schwarzenegger. El
marido de la ucraniana eran tan ancho de espalda como yo de alto. Tenía
una malla celeste con flores verde fluor y un toallón colgado de la
espalda. Cuando me vió, el rostro se le reconfiguró y empezó a gritarle a
su mujer. Puteadas en ucraniano, supongo, obolonka tvoyei sestry, ty suka,
podyvit'sya na mudak Ty shcho , zhartuyesh. Después me miró a mí. Tenía
la piel roja por el calor del momento y me mostró su dentadura lobuna. Voy
a resumir lo siguiente que pasó diciendo que me hizo todo lo que yo le
hice a las cañerías de su habitación, pero con menos delicadeza. Cuando
pude escaparme y salir de la setecientos cuarenta y cuatro me encontré a
los mozos espiando por la cerradura, muertos de risa.


\vspace{0.5cm}
\hrulefill\hspace{0.2cm} \decofourleft\decofourright \hspace{0.2cm} \hrulefill
\vspace{0.5cm}

\hyphenation{ca-ri-be}
\hyphenation{su-pues-to}
\hyphenation{si-llón}
\hyphenation{se-ñas}
\hyphenation{in-su-pe-ra-ble}
\hyphenation{li-mi-tó}
\hyphenation{a-gra-da-ble}
\hyphenation{la-ti-no-a-me-ri-ca-no}
\hyphenation{quie-nes}
\hyphenation{mar-tín}
\hyphenation{co-la-te-ra-les}
Hace unos años leí esta historia en un post en el blog de Fede Heiz,
¿lo tenés? el de barba y pelo largo. Bueno, te decía que me acordé de
esta historia, te la paso porque te puede servir para la novela:

En el primer día de sus vacaciones en el Caribe, Martín salió a matar. De
abajo para arriba llevaba ojotas, bermudas rojas, camisa floreada, amplia
como para disimular la zapán y abierta para que se viera la cadena de oro,
anteojos negros con marco de metal dorado y la gorra del Club Atlético
Talleres, que no por seductor iba a renegar de sus sentimientos más
profundos. Entró a la playa con el pecho henchido, consciente de que las
minas lo miraban. Con su andar recio, tirando arena para arriba con el
talón de las ojotas a cada paso, se dirigió a un sillón de playa sobre
el que se recostó con un profundo suspiro de placer.

Tanto despliegue de masculinidad, por supuesto, no podía pasar desapercibido,
que al fin y al cabo las mujeres tampoco son de palo, y menos por esas
latitudes en las que el sol y la naturaleza mantienen la sangre siempre
efervescente. Había estado escasos minutos sobre el sillón cuando escuchó
una voz melódica que le preguntaba “would you mind spreading some sun
lotion on my back?” Martín no hablaba una palabra de inglés, lo que no
le impidió comprender la pregunta, porque pocas cosas hay más cercanas a
un lenguaje universal que la imagen de una morena pechugona, apenas cubierta
por una minúscula tanga, extendiendo un frasco de bronceador con la mano
derecha al tiempo que, con la izquierda, se frota demostrativamente el
hombro opuesto. Haciendo un enorme y futil esfuerzo por mantener la calma,
Martín se levantó y procedió a esparcir el bronceador sobre el dorso de la
ninfa. Ella festejaba con leves gemidos de placer cada uno de sus movimientos,
que iban haciéndose cada vez menos enérgicos y amplios a medida que él
se daba cuenta de que, a ese ritmo, la espalda se le acabaría enseguida.

Como muestra de agradecimiento, y por señas, ella le ofreció
compartir con él un trago en el bar de la playa. Allí, en virtud de la
insuperable barrera idiomática, la conversación se limitó a un constante
intercambio de sonrisas que, ocasionalmente, estallaba en inexplicables
carcajadas. Brindaron, bebieron el uno de la copa del otro, se hicieron caras,
y sin previo aviso Martín se encontró con un pie de ella suavemente apoyado
en su entrepierna, lo que cambió de inmediato la naturaleza de su sonrisa.

No habían aún terminado el segundo trago cuando se acercó a la mesa
otra muchacha, más hermosa, exhuberante y desnuda que la primera, si tal
cosa cabía, quien la saludó con un beso en la boca. Intercambiaron breves
frases en inglés, lanzando de vez en cuando risitas y miradas pícaras en
dirección a Martín. De alguna manera, se las arreglaron para preguntarle
al oído si no quería acompañarlas a la habitación del hotel, donde las
estaba esperando otra amiga.

El hotel no estaba lejos, y la puerta se abrió para revelar a la más bella,
más voluptuosa y más desnuda de las tres, quien los esperaba con tragos
servidos. Martín tardó un momento en darse cuenta de la presencia de otro
hombre en la habitación, y pese a que su aspecto era agradable sintió
cierto temor. El extraño resultó ser un centroamericano muy cordial,
de nombre Carlos, quien al ver que Martín no hablaba inglés, inició una
charla con él para que no se sintiera tan aislado.

Comenzó hablándole de la belleza del hotel, y de la comodidad de las
habitaciones. Mientras una de las chicas abrazaba a Martín por detrás,
metiendo su mano bajo la camisa y pellizcando sus tetillas, Carlos alababa
las artes de los cocineros y observaba las ventajas de estar tan cerca de
la playa. Tironeado entre sus sentidos de urbanidad y solidaridad, que le
impedían dejar hablando solo a un hermano latinoamericano, y su libido
que constantemente le hacía desviar la vista hacia las otras dos amigas,
quienes se besaban y acariciaban sobre el sofá, Martín simulaba atender
las apreciaciones de Carlos acerca del impecable servicio de seguridad y
la altísima calidad de la atención, comparable sólo a la de los hoteles
cinco estrellas.

Martín pasó todo el día siguiente tratando de reconstruir cómo y por qué
había salido de la habitación sin siquiera haber tocado a las niñas, pero
sobre todo no conseguía recordar en qué momento había firmado el contrato
que lo hacía acreedor, por las siguientes dos décadas, a una habitación
del hotel en tiempo compartido, diez días al año, del 3 al 13 de junio,
por una módica cuota mensual más servicio de mucama, seguros, gastos
comunes, administrativos y colaterales, sellados, prima de mantenimiento,
comisiones e impuestos.


\vspace{0.5cm}
\hrulefill\hspace{0.2cm} \decofourleft\decofourright \hspace{0.2cm} \hrulefill
\vspace{0.5cm}

\hyphenation{Mé-xi-co}
Algo que no te conté, del tipo del avión, es que cuando me subí estaba
sentado en mi asiento, el 23B. Se confundió, el suyo era el 32B. Cuando
me desperté con todo el movimiento no me di cuenta, pero antes de bajarme
la vi, abajo del 23B estaba su billetera. Me di cuenta por las fotos en
las credenciales. Ya estabamos en México, ¿qué iba a hacer? ¿llevarla
a la policía? ¿la policía de México se la iba a llevar a la policía
de Brasil? ¿a la de Sudáfrica? Mucha logística y yo estaba arrancando
las vacaciones. Me la guardé en la mochila para después tirarla. Si,
es verdad, no suelo tomar las mejores decisiones.


\vspace{0.5cm}
\hrulefill\hspace{0.2cm} \decofourleft\decofourright \hspace{0.2cm} \hrulefill
\vspace{0.5cm}

¿Lo viste al flaco ese? Un solo machucón. Para mi que se cayó de la
pirámide de Cobá.


\vspace{0.5cm}
\hrulefill\hspace{0.2cm} \decofourleft\decofourright \hspace{0.2cm} \hrulefill
\vspace{0.5cm}

One dollar. Vengo caminando desprevenido y ataca un vendedor.

No hay plata. Lo corto en seco y sigo caminando.

¿Argentino? Mitad de precio. Algo para la suegra. Un recuerdito. Me grita
para que lo escuche.

No se lo merece. Me doy vuelta y contesto.

El tipo sonríe mostrando los dientes que le faltan y remata. También
tenemos cuchillos.


\vspace{0.5cm}
\hrulefill\hspace{0.2cm} \decofourleft\decofourright \hspace{0.2cm} \hrulefill
\vspace{0.5cm}

\hyphenation{a-dre-na-li-na}
En mi quinta noche en México, solo en la habitación del hotel, me acordé
de la billetera del sudafricano, la mula. Cualquier persona la ubiese
dejado ahí, en el piso del avión, o se la hubiese dado a una asafata
y listo. Señorita, encontré esto en mi asiento, tome, gracias, no, de
nada. Listo, chau, vacaciones. Pero yo no, siempre tuve esa enfermedad de
creerme detective o policía o a veces ladrón. Esa necesidad de adrenalina,
teórica, porque si veo un arma o sangre, seguro me meo encima.


\vspace{0.5cm}
\hrulefill\hspace{0.2cm} \decofourleft\decofourright \hspace{0.2cm} \hrulefill
\vspace{0.5cm}

\hyphenation{mo-ne-da}
En el último grupo que traje había un hombre que me mostraba contento
unas monedas que le habían dado de cambio en uno de los puestos de
artesanías. Dolares maya, le dijeron. 1 dolar americano, 1 dolar maya. Eran
Quetzales, la moneda de Guatemala. Tengan cuidado.


\vspace{0.5cm}
\hrulefill\hspace{0.2cm} \decofourleft\decofourright \hspace{0.2cm} \hrulefill
\vspace{0.5cm}

El mar y la playa son como un libro inmenso, un libro infinito, que día
a día va pasando sus páginas. Cada ola sobre la arena que trae algo y se
lleva un poco puede verse como una hoja de ese libro que alguien va pasando.

Los glifos con los que se escribe son marcas en la arena. Un cuadrado que
delimita una cancha, un corazón que dice J y C, un castillo.


\vspace{0.5cm}
\hrulefill\hspace{0.2cm} \decofourleft\decofourright \hspace{0.2cm} \hrulefill
\vspace{0.5cm}

Chica, elija el que le gusta, el chico paga.


\vspace{0.5cm}
\hrulefill\hspace{0.2cm} \decofourleft\decofourright \hspace{0.2cm} \hrulefill
\vspace{0.5cm}

\hyphenation{di-re-cción}
En la billetera del sudafricano, al que le explotaron las cápsulas
de cocaína adentro, encontré 400 dolares, identificaciones y una
tarjeta. La tarjeta era blanca y en letras de molde negras tenía tatuada
una dirección. Del otro lado tenía el dibujo de ángel con las alas
rotas. La dirección era en Cancún, en un lugar alejado de la línea de
hoteles. Una tarde me tomé el colectivo y me acerqué al lugar. No pensaba
tocar el timbre con cara de feliz cumpleños, ni siquiera pisar al cuadra,
solo dar una vuelta, pasear, ver el barrio e imaginarme historias. Siempre
estoy imaginando policiales.


\vspace{0.5cm}
\hrulefill\hspace{0.2cm} \decofourleft\decofourright \hspace{0.2cm} \hrulefill
\vspace{0.5cm}

Esto es obsidiana. Piedra volcánica. Los mayas la obtenían a través
del comercio con otros pueblos. Tiene propiedades energéticas. Toquenla,
toquenla. Vean como se vuelve dorada al sol. Yo tengo una de estas en
mi casa dentro de un recipiente con agua para que atrape las energías
negativas. En luna llena tienen que cambiar el agua.

Los reyes la usaban como ornamenta por su belleza y los guerreros como punta
de flecha por su dureza. Hoy es usada en los centros de belleza para hacer
masajes y se hacen cuchillas de obsidiana para usar en cirugías de ojo o
corazón porque su filo es mucho más delgado que el de los escalpelos de
acero. Los cortes hechos con las cuchillas de obsidiana son más finos y
causan menos daño al tejido permitiendo que el cuerpo sane antes. ¿Cuantas
van a llevar? 50 dólares las grandes, 10 dólares las más chicas.


\vspace{0.5cm}
\hrulefill\hspace{0.2cm} \decofourleft\decofourright \hspace{0.2cm} \hrulefill
\vspace{0.5cm}

¿Cuál le gusta? Buen precio. Más barato de este lado. ¿Habla español? Le
hago un descuento si habla español. ¿España?

No.

¿Portugal?

No.

¡Argentina! Amigos de Francisco.


\vspace{0.5cm}
\hrulefill\hspace{0.2cm} \decofourleft\decofourright \hspace{0.2cm} \hrulefill
\vspace{0.5cm}

Ese día tomé lo que en perspectiva parece una desición muy tonta, pero
juro que en el momento me sentí muy listo. Dejé mi billetera en el hotel
porque no quería que me la vayan a robar. Maldecí la desición recién
en el momento en que sentí un arma en la espalda y una voz áspera que
al costado del oído me preguntó por qué miraba tanto hacia la casa del
patrón. Dos matones vestidos con traje negro y camisa color vino tinto
me llevaron a los empujones hasta dentro de la mansión emplazada en la
direccion grabada en la tarjeta guardada en la billetera que esa mañana
puse en el bolsillo trasero izquierdo de mi pantalón.



\vspace{0.5cm}
\hrulefill\hspace{0.2cm} \decofourleft\decofourright \hspace{0.2cm} \hrulefill
\vspace{0.5cm}

Una mañana vimos un chancho corriendo por la playa, paralelo a la línea del
mar. Bueno, paralelo y línea son formas de decirle a las curvas en movimiento
que va describiendo el agua sobre la arena. Es un movimiento oscilatorio,
senoidal casi te podría decir. Es que soy profe de matemáticas y siempre
estoy buscando nuevos ejemplos para los chicos. Bueno, te decía que una
mañana vimos un chancho pasar corriendo. Y atrás, no me lo vas a creer,
blandiendo la hoja metálica de su cuchilla, el cocinero del hotel.


\vspace{0.5cm}
\hrulefill\hspace{0.2cm} \decofourleft\decofourright \hspace{0.2cm} \hrulefill
\vspace{0.5cm}

\hyphenation{gor-di-to} En la playa no hay Doctor Mora, escribano Peretti
o ingeniero Zimmerman. Ese gordito que ves corriendo en patas hasta llegar
al agua y se tira de panza pudo haber estado la semana pasada haciendo una
operación a corazón abierto en el Hospital Cullen. El flaco huesudo que se
pone protector solar factor 40 en todo el cuerpo, con un sombrero pescador,
caminando en puntitas de pie para no quemarse con la arena pudo haber sido
el intermediario en un negocio millonario con una empresa china. Ese que
salta y remata pero deja la pelota contra la red en el volley playero puede
ser un empleado de Techint en sus vacaciones.


\vspace{0.5cm}
\hrulefill\hspace{0.2cm} \decofourleft\decofourright \hspace{0.2cm} \hrulefill
\vspace{0.5cm}

\---What happened with Mr. Louis Hendrik Smith?

\---¿He...? ¿Dónde estoy? ¿Quienes son ustedes?

\---¿Qué pasó con el Sr. Louis Hendrik Smith? Lo estoy esperando hace
unos días ya. El tenía que traerme un paquete muy importante y estoy a
punto de perder la paciencia. Por suerte, parece, que lo mandó a usted a arreglar
cuentas con nosotros.


\vspace{0.5cm}
\hrulefill\hspace{0.2cm} \decofourleft\decofourright \hspace{0.2cm} \hrulefill
\vspace{0.5cm}

Esta noche las costillitas de cerdo están menos grasosas que ayer. Pasame
otra.


\vspace{0.5cm}
\hrulefill\hspace{0.2cm} \decofourleft\decofourright \hspace{0.2cm} \hrulefill
\vspace{0.5cm}

Estábamos desayunando cuando nos llegó el comentario. A la mañana, en la
arena, bajo un árbol de ceibas apareció muerto un turista. Tenía marcas
en todo el cuerpo, como de animal. Y estaba blanco como un papel. Escuché
a alguien decir que era el argentino de la habitación 189. Nosotros estamos
en la 188 y lo recordé. Alto, 50 años, bigote recto. Estaba bronceado
porque hacía más de una semana que estaba aca. Todos los días en la
playa. Se cocina vuelta y vuelta, había bromeado con mi hijo. Profesor de
matemáticas me dijo que era.


\vspace{0.5cm}
\hrulefill\hspace{0.2cm} \decofourleft\decofourright \hspace{0.2cm} \hrulefill
\vspace{0.5cm}

La primera vez que escuché el nombre Playa del Carmen me imaginé a un
montón de viejas gordas con malla enteriza tiradas en la arena. Te lo
juro. Con ese nombre no pude hacer otra cosa que pensar en mi tía. Carmen.


\vspace{0.5cm}
\hrulefill\hspace{0.2cm} \decofourleft\decofourright \hspace{0.2cm} \hrulefill
\vspace{0.5cm}

\hyphenation{guí-a} Cancún. Can-Cún, dos palabras maya. El guía nos
hablaba despacio como si no habláramos todos español. Can significa nido
y cún serpiente. Hace veinte años este lugar estaba lleno de serpientes,
no había medios de comunicación, no llegaban proveedores, faltaba la
comida y la gente vivía de forma muy precaria. Luego llegaron los hoteles,
las cadenas e hicieron la ciudad a su imagen y semejanza. Cancún, nido de
serpientes. No lo repitan en sus países. No es buen nombre para el turismo.


\vspace{0.5cm}
\hrulefill\hspace{0.2cm} \decofourleft\decofourright \hspace{0.2cm} \hrulefill
\vspace{0.5cm}

En el almuerzo, el tema del hombre que apareció muerto en la playa ya era
\emph{vox populi} y se habían elaborado al menos diez hipótesis diferentes
acerca de lo que le había pasado: robo y muerte con arma blaca, borracho
dormido comida de animales salvajes, atropellado y abandonado, asesinado
en otro lugar y dejado ahí para despistar a la polocía, y algunas otras
que ahora no me acuerdo.

Estábamos justamente repasando estas hipótesis (debo admitir que con cierto
entretenimiento, ya que soy aficionado a las novelas policiales) con otra
pareja cuando el mozo que nos servía la bebida se quedó mirándonos,
como congelado. Qué le pasa, hombre, le pregunté. Y el tipo seguía
absorto. Se volvió hacia mi y me lo dijo: Xtabay.


\vspace{0.5cm}
\hrulefill\hspace{0.2cm} \decofourleft\decofourright \hspace{0.2cm} \hrulefill
\vspace{0.5cm}

Las comunidades Maya que pueden ver al costado de los ruta aún conservan
muchas de sus costumbres. Por ejemplo, duermen en hamacas en lugar de en
camas. Esto es porque son más frescas y por los bichos. Usan la hamaca
para todo, para dormir, descansar, reproducirse... Es la famosa hamaca de
San Andrés, donde se acuestan dos y amanecen tres.


\vspace{0.5cm}
\hrulefill\hspace{0.2cm} \decofourleft\decofourright \hspace{0.2cm} \hrulefill
\vspace{0.5cm}

En la biblioteca junto a la piscina hay unos doscientos libros.  Una señora
repasa los títulos moviendo los labios, impronunciables.

Yo tampoco encontré ninguno en español.

Es que esos libros son regalos de huéspedes para los otros huéspedes,
interviene el muchacho que cambia las toallas, y como buenos latinos ninguno
regala nada.

Vuelva a buscar mañana señora, que voy a dejar uno mío.

Se refiere a uno escrito por él, aclara mi esposa y me sonrojo.

Lo va a distinguir fácil, porque es finito.

¿Y qué escribís?, se interesa el marido de la señora.

Cuentos.

Ah... hay que tener imaginación para eso.

No tanta, me la paso recogiendo voces de otros.


\vspace{0.5cm}
\hrulefill\hspace{0.2cm} \decofourleft\decofourright \hspace{0.2cm} \hrulefill
\vspace{0.5cm}

Todas las mañanas a las 9:15 hs los miembros del equipo de animación del
hotel se aparecen por el comedor donde se está sirviendo el desayuno, toman
un plato, lo cargan de fruta y se sientan a desayunar con algunos de los
huéspedes. Charlan amigablemente en español, inglés, alemán o francés.

Una señora que come sola y con quien no se ha sentado nadie en toda la
semana resopla y se le escucha decir, despechada, les pagan para eso.


\vspace{0.5cm}
\hrulefill\hspace{0.2cm} \decofourleft\decofourright \hspace{0.2cm} \hrulefill
\vspace{0.5cm}

{\em Hace muchos años, antes de que el hombre blanco llegue a Yucatán,
en un poblado de nombre olvidado vivían dos mujeres. Una se llamaba Xtabay
y era llamada Xkeban (que significa prostituta, mujer mala o dada al amor
ilícito) y la otra se llamaba Utz-Colel, mujer buena, decente y limpia.

Xtabay estaba enferma de amor y de pasión y prodigaba su hermoso cuerpo
a cuanto hombre lo deseara. Utz-Colel, en cambio, jamás había cecido a
ningún amor carnal, nunca había cometido ningún pecado amoroso.

Xtabay era de gran corazón. Generosa, siempre estaba ayudando a los
demás. A los ancianos, a los niños e incluso a los animales que otros
desechaban. Vendía las joyas y finas túnicas que sus enamorados le regalaban
para poder ayudar a quién la necesitaba. Utz-Colel, por su parte, bajo su
apariencia dulce y cándida, era fría, orgullosa y de corazón duro.}


\vspace{0.5cm}
\hrulefill\hspace{0.2cm} \decofourleft\decofourright \hspace{0.2cm} \hrulefill
\vspace{0.5cm}

Hola, para reservar está noche en uno de los restaurantes temáticos. La
Hacienda, ese es el de carne argentina, ¿no?


\vspace{0.5cm}
\hrulefill\hspace{0.2cm} \decofourleft\decofourright \hspace{0.2cm} \hrulefill
\vspace{0.5cm}

Paseando con su camiseta de la selección Argentina de fútbol se jacta de
gran regateador por haber pagado dos dólares por un imán que le ofrecían
a cinco y pagó sesenta por un estuche impermeable para la cámara de fotos
que se lleno de agua del mar Caribe en la primer inmersión.


\vspace{0.5cm}
\hrulefill\hspace{0.2cm} \decofourleft\decofourright \hspace{0.2cm} \hrulefill
\vspace{0.5cm}

Hansel y Gretel. 300 kg entre los dos. Socios vitalicios del bar de la
piscina. Cómo ballenas encalladas, destinadas a morir en la orilla, nunca
cruzaron los metros de arena que los separaban del mar. Es que estamos de
vacaciones, dijo ella.


\vspace{0.5cm}
\hrulefill\hspace{0.2cm} \decofourleft\decofourright \hspace{0.2cm} \hrulefill
\vspace{0.5cm}

{\em Un día la gente dejó de ver a Xtabay y las habladurías dijeron
que había ido a otros pueblos a ofrecer su amor a los hombres que lo
solicitaran. Sus vecinos no tardaron en darse cuenta de que un dulce aroma
a flores salía de la casa que ella habitaba. Cuando entraron la encontraron
muerta, hermosa y sola en su lecho emanando un riquísimo perfume. El pueblo
entero se llenó del delicioso aroma a flores que emanaba de su casa y el
prodigio estuvo en boca de todos.

Cuando Utz-Colel se enteró, enojada dijo que de un cuerpo vil y corrupto solo
puede salir pestilencia, si brota perfume es obra de los malos espíritus
que ayudan a las mujeres sucias para seguir provocando a los hombres aún
después de muertas. La gente le creyó y dejaron el cadaver de Xtabay. Unos
pocos, por lástima, se encargaron de enterrarla y al día siguiente,
sobre su tumba, en la tierra aún fresca, crecieron cientos de flores con
el mismo perfume del día anterior. Con el nectar de estás flores se hace
una bebida que embriaga tanto como el amor de Xtabay.}


\vspace{0.5cm}
\hrulefill\hspace{0.2cm} \decofourleft\decofourright \hspace{0.2cm} \hrulefill
\vspace{0.5cm}

Mónica, sentada en la arena, escribió unos versos: Tengo el sol de Cozumel
en mi espalda. / Tengo el mar Caribe en mis oídos. / El agua turquesa que
me baña. / Los granos de arena que no son míos.


\vspace{0.5cm}
\hrulefill\hspace{0.2cm} \decofourleft\decofourright \hspace{0.2cm} \hrulefill
\vspace{0.5cm}

\hyphenation{aun-que}
Los cortes de carne vacuna que se pueden encontrar en una carnicería
argentina y que un argentino usaría para preparar un asado son, aunque
no se limitan a, costilla, marucha, falda, matambre. Cuando nos sentamos
a comer en el comedor llamado La Hacienda nos ofrecieron T-Bone y Rib Eye
Steak. Me quedé mirando.

Es que son cortes americanos para barbacoa, nos dijo el amable mozo.

La guarnición por excelencia es la ensalada de lechuga y tomate. Por
supuesto, otras ensaladas son también muy bien recibidas, zanahoria y huevo
duro, tomate y huevo duro, repollo, achicoria. En el plato que me sirvieron
había una papa de dudosa cocción, un cuarto de choclo y una cebolla frita.

Después de media hora esquivando la brecha cultural me fui al buffet y me
comí media pizza.


\vspace{0.5cm}
\hrulefill\hspace{0.2cm} \decofourleft\decofourright \hspace{0.2cm} \hrulefill
\vspace{0.5cm}

{\em Poco después de la muerte de Xtabay, murió Utz-Colel. Al entierro
acudió el pueblo etero para llorarla y despedir a tan noble mujer. Pero de su
tumba brotó el olor pestilente de un cadáver putrefacto. Sobre su sepulcro
nació un ceto espinoso. Intocable, da una flor bella pero pestilente.

Ya convertida en esta planta, Utz-Colel reflexionó erroneamente que como
Xtabay se entregaba amorosamente a los hombres, los dioses la habían
convertido en una flor de dulce aroma. No pensó en todas las buenas obras
que esta hacía y que eran, en verdad, la razón del milagro. Entonces pidió
ayuda a los espíritus malignos para  regresar al mundo cada vez que quisiera
convertida en mujer para seducir a los hombres y entregarles un amor nefasto,
el único que su duro corazón le permite. Peinando su larga cabellera, con
una flor de espinas como adorno, sigue a los hombres hasta conseguir atraerlos
y seducirlos, para al final asesinarlos en el frenesí de un amor infernal.}


\vspace{0.5cm}
\hrulefill\hspace{0.2cm} \decofourleft\decofourright \hspace{0.2cm} \hrulefill
\vspace{0.5cm}

Cozumel. Rent a car.

¿Cuál es el más barato?

Este, el Chevy rojo. ¿Les gusta? ¿Tiene carnet de conducir y tarjeta de
crédito? Muy bien, cuando regresan de dar la vuelta a la isla le llenan
el tanque de gasolina.


\vspace{0.5cm}
\hrulefill\hspace{0.2cm} \decofourleft\decofourright \hspace{0.2cm} \hrulefill
\vspace{0.5cm}

\hyphenation{as-ti-llas} \emph{Escribir es sondear y reunir briznas o
astillas de experiencia y de memoria para armar una imagen.} Juan José Saer.


\vspace{0.5cm}
\hrulefill\hspace{0.2cm} \decofourleft\decofourright \hspace{0.2cm} \hrulefill
\vspace{0.5cm}

Ahí hay una estación de servicio, frená.

¿Cuanto le echamos?

Llenalo, pero no tanto, es para devolverlo.

...

Che, van 200 pesos ya, córtalo porque me dijeron que iba a gastar 100 en
dar la vuelta a la isla.

Lo enciendo a ver si ya marca lleno. Todavía no,
no puede ser...


\vspace{0.5cm}
\hrulefill\hspace{0.2cm} \decofourleft\decofourright \hspace{0.2cm} \hrulefill
\vspace{0.5cm}

¿A dónde va, buen hombre? Le habló con una voz dulce y seductora.

La mujer se encontraba sentada bajo un frondoso árbol que él no
conocía. Tenía cabellos largos y negros y una flor espinosa peinada entre
ellos. Sus labios eran gruesos y su tez cobriza. Sus ojos eran negros y
profundos y su belleza no tenía igual. Lo miró con firmeza y le extendió
su mano.

El hombre, dócil, esclavo, falto de voluntad, caminó como hipnotizado hasta
los brazos de la hermosa mujer que lo llamaba. Bajo la sobra de ese árbol
lo acarició y lo besó. Luego empezó a desvestirlo. Incapás de oponer
cualqueir resistencia frenete a los favores que la mujer le proporcionaba,
el hombre no se dió cuenta que la violencia del ritual iba en aumento. Las
caricias se convirtieron en arañazos, pero las uñas clabadas en su carne
le daban igual o más palcer. Los besos se convirtieron en mordiscos, pero
su exitación iba en aumento. Ninguno de los dos se podía contener y
los gritos eran como de animales apareándose. La mujer lo mordía, pero
el también hacía lo suyo. Podía sentir el sabor a sangre en su boca,
transpiración, pelos. Garras, pezuñas, aullidos.


\vspace{0.5cm}
\hrulefill\hspace{0.2cm} \decofourleft\decofourright \hspace{0.2cm} \hrulefill
\vspace{0.5cm}

¿Qué es ese charco abajo del auto?

Le falla la bomba señor.

¿Del agua?

No, la bomba, la bomba, está mal su máquina, está perdiendo gasolina.


\vspace{0.5cm}
\hrulefill\hspace{0.2cm} \decofourleft\decofourright \hspace{0.2cm} \hrulefill
\vspace{0.5cm}

Maia.

¿Si?

¿Te acordás lo queme dijiste ayer? Que tenías la sensación de que había alguien en tu habitación, después de ducharte y del miedo te vestiste rápido y saliste corriendo.

Si.

Bueno, hoy leí este cuento en uno de los libros de la biblioteca que está al lado de la piscina y me acordé de vos. Se llama El departamento:

Estoy nuevamente en el departamento. De vez en cuando vuelvo por la noche. Todavía tengo mis llaves y no cambiaron la cerradura. Abro sin hacer ruido, subo despacio las escaleras y prendo la luz de la cocina. La del living no, porque se ve desde afuera por la ventana. Enciendo una hornalla de la cocina y pongo la pava con agua a calentarse. Mientras reviso los impuestos a pagar y alguna carta abierta, como algunas masitas del tarro o pan en rodajas. Si en el paquete quedan pocas rodajas, digamos menos de seis, no como. Se me hace que sería fácil que me descubran.

A la noche el departamento está vacío porque la nueva inquilina va a la universidad. Estudia alguna ingeniería. Lo se por sus apuntes. A veces los leo, pero me resultan bastante aburridos. Después de darme un rutinario paseo por el dormitorio, empiezo a limpiar los rastros de mi presencia. Lavo la taza en la que me tomé café, limpio, guardo. Me causa gracia, en mi actual departamento no soy tan prolijo.

Entonces se me ocurre, ¿por qué no ir más lejos hoy? Vuelvo al dormitorio y me acuesto bajo la cama. El cubrecama llega hasta el suelo y eso hace que mi escondite sea perfecto. Y me quedo ahí, esperando. Una hora. Dos.

Entonces escucho su llave en la cerradura, sus pasos en la escalera, la tecla que enciende la luz. Enciende una hornalla, tal vez la misma que elegí yo hoy. Escucho que abre un paquete. Tal vez fideos o arroz. Presto atención a cado uno de los sonidos. No se si pasó media hora o tres horas cuando escucho que cierra la canilla luego de lavar los platos.

La luz se enciende y vuelvo a verme las manos después de varias horas de oscuridad. Están transpiradas.


\vspace{0.5cm}
\hrulefill\hspace{0.2cm} \decofourleft\decofourright \hspace{0.2cm} \hrulefill
\vspace{0.5cm}

Sáquelo de aquí.

Y... ¿es seguro encenderlo?

Pues no, pero no puede dejarlo aquí perdiendo. Lléveselo, lléveselo.


\vspace{0.5cm}
\hrulefill\hspace{0.2cm} \decofourleft\decofourright \hspace{0.2cm} \hrulefill
\vspace{0.5cm}

Leído en Wikipedia: \emph{La ciudad recibía en la
antigüedad el nombre maya de Zamá (que significa en maya amanecer) y
el actual, Tulum (que significa en maya muralla), que parece haber sido
utilizado para referirse a la ciudad cuando ya se encontraba en ruinas.}


\vspace{0.5cm}
\hrulefill\hspace{0.2cm} \decofourleft\decofourright \hspace{0.2cm} \hrulefill
\vspace{0.5cm}

Taxi.

¿Cuánto sale de acá a Tulum?

La cartilla dice 505 pesos, pero yo los llevo por 450.

Precio para Xolopes, acotó el taxista por lo bajo.

¿Qué te dijo? No se, 450 pesos, un regalo. Vamos, vamos.


\vspace{0.5cm}
\hrulefill\hspace{0.2cm} \decofourleft\decofourright \hspace{0.2cm} \hrulefill
\vspace{0.5cm}

¿Y?

Y, ¿qué?

¿Cómo y qué? No te hagás la tota. ¿Qué te pareció el cuento El departamento?

No tiene nada que ver. Primero, lo de que me sentía observada lo inventé porque había salido toda desarreglada y vos estabas ya cambiada para la noche. Segundo, acá en el hotel eso no puede pasar porque cada habitación tiene una tarjeta distinta y cuando te vas cambian la clave.


\vspace{0.5cm}
\hrulefill\hspace{0.2cm} \decofourleft\decofourright \hspace{0.2cm} \hrulefill
\vspace{0.5cm}

\hyphenation{re-pa-ra-da}
¿Hay muchos accidentes?

Solo cuando llueve. Está carretera fue reparada, pero la que viene no. Es
de concreto hidráulico. Patina. Está es de asfalto. Cuando se moja es
como gramilla.


\vspace{0.5cm}
\hrulefill\hspace{0.2cm} \decofourleft\decofourright \hspace{0.2cm} \hrulefill
\vspace{0.5cm}

\hyphenation{pu-e-de}
\hyphenation{ob-te-ner}
\hyphenation{jue-go}
Esta historia la leí una vez en un libro, una vez me la contaron y dos
veces la viví. Cambian las nacionalidades y las profesiones, pero la idea
es la misma. Cierto es que cuando la viví por segunda vez, sonreí tanto
que se me acalambraron los músculos de la cara.

Un empresario norteamericano estaba mirando el mar en la costa de un
pueblito mexicano cuando vió a un pescador acercarse a la orilla con
su barco cargado con algunos pescados. El empresario le perguntó cuanto
tiempo al día dedicaba a la pesca. Solo un par de horas, respondió el
pescador. El empresario le preguntó por qué no le dedicaba más tiempo,
ya que así podría sacar más peces. El mexicano respondió que con eso
le alcanzaba para mantener a su famialia. Pero... qué hace el resto del
tiempo, quiso saber el empresario. Me levanto tarde, juego con mis hijos,
duermo la siesta con mi mujer y a la tardecita voy al pueblo a tomar vino
y tocar la guitarra con mis amigos. Como verá, tengo una vida muy ocuapda.

El norteamericano sonrio con sorna. Soy experto en negocios y puedo
ayudarlo. Usted debería pasar más tiempo pescando, así tendría más
pescados y con el dinero extra se podría comprar un bote más grande. Un
bote más grande le permitirá aumentar aún más su pesca y con el dinero
extra podrá comprar más botenes. Con una flota podrá aumentar aún más su
pesca y con tanto volumen podrá negociar directamente con los que procesan
el pescado en lugar de venderle a un intermediario. Ya no podrá vivir en
este pueblito, tendrá que ir a México DF, luego a Los Ángeles y más tarde
tal vez a New York donde podrá desarrollar todo el potencial de su empresa.

El pescador mexicano lo seguía rascándose la cabeza. Pero señor, cuanto
tiempo me va a llevar hacer eso. El empresario le explicó que unos 15 o
20 años. Y luego qué, pregunto el pescador. Ahí viene la mejor parte,
le respondió el norteamericano. Su empresa podrá salir a la bolsa,
vederá acciones y se hará rico. Podría obtener millones. Millones,
repitió el mexicano. Y luego qué, volvió a preguntar. Luego se retira,
puede mudarse a un pequeño pueblo pesquero, levantarse tarde, juegar con
sus hijos, dormir la siesta con su mujer y a la tardecita ir al pueblo a
tomar vino y tocar la guitarra con sus amigos.


\vspace{0.5cm}
\hrulefill\hspace{0.2cm} \decofourleft\decofourright \hspace{0.2cm} \hrulefill
\vspace{0.5cm}

A la vuelta llueve. Bienvenido a la pista de patinaje.


\vspace{0.5cm}
\hrulefill\hspace{0.2cm} \decofourleft\decofourright \hspace{0.2cm} \hrulefill
\vspace{0.5cm}

\hyphenation{des-pués} Otro día en la playa con el cuaderno, después de
flotar por horas en el agua, Mónica escribe: Floto / boca arriba en las
aguas / profundas. / O floto / boca abajo / en el cielo celeste.


\vspace{0.5cm}
\hrulefill\hspace{0.2cm} \decofourleft\decofourright \hspace{0.2cm} \hrulefill
\vspace{0.5cm}

Soy México reza el cartel del parque más visitado y el actor que posa,
vestido como maya, dando un caderazo en medio de un tradicional juego es
pelota, es guatemalteco.


\vspace{0.5cm}
\hrulefill\hspace{0.2cm} \decofourleft\decofourright \hspace{0.2cm} \hrulefill
\vspace{0.5cm}

Carretera 307. Carretera principal. El cinturón de seguridad salva
vidas, úselo. Snorkel 500  m. Retorno 1 km. Carril izquierdo solo para
rebasar. Maneje con precaución, su familia lo espera. Compre lotes en Tulum,
desde 500 metros cuadrados.


\vspace{0.5cm}
\hrulefill\hspace{0.2cm} \decofourleft\decofourright \hspace{0.2cm} \hrulefill
\vspace{0.5cm}

Ninguna cámara va a capturar lo que ves con los ojos, así que dejá de
sacar fotos y ponete a mirar.


\vspace{0.5cm}
\hrulefill\hspace{0.2cm} \decofourleft\decofourright \hspace{0.2cm} \hrulefill
\vspace{0.5cm}

\hyphenation{por-que}
¿De donde sos?

Israel.

¿Y qué estás escribiendo?

Con una pronunciación que no podía disimular su origen nos contestó. Anoto
mis pensamientos. Es mucho mejor que las fotos. Necesito recordar lo que
siendo en este momento porque es todo muy lindo.


\vspace{0.5cm}
\hrulefill\hspace{0.2cm} \decofourleft\decofourright \hspace{0.2cm} \hrulefill
\vspace{0.5cm}

Podemos acortar camino por acá. Lo que no está prohibido está permitido.


\vspace{0.5cm}
\hrulefill\hspace{0.2cm} \decofourleft\decofourright \hspace{0.2cm} \hrulefill
\vspace{0.5cm}

Vamos a descansar un rato porque todavía tenemos que caminar del dicho
al hecho.


\vspace{0.5cm}
\hrulefill\hspace{0.2cm} \decofourleft\decofourright \hspace{0.2cm} \hrulefill
\vspace{0.5cm}

Timados, engañados, seducidos, robados, golpeados, arruinados, burlados, xolopes.


\vspace{0.5cm}
\hrulefill\hspace{0.2cm} \decofourleft\decofourright \hspace{0.2cm} \hrulefill
\vspace{0.5cm}

El mirador.

Don Cafeto.

Playa de los pescadores.

Playa paraíso.

Basta, no camino más.


\vspace{0.5cm}
\hrulefill\hspace{0.2cm} \decofourleft\decofourright \hspace{0.2cm} \hrulefill
\vspace{0.5cm}

InfoMex. Noticias de Internet. Todos los que vamos de vacaciones a hoteles
cinco estrellas hemos utilizado las tarjetas magnéticas que permiten
abrir y cerrar las puertas de los cuartos y les hemos confiado nuestra
intimidad. Después de leer el artículo que salió en la revista Forbes,
la paranoia y desconfianza se apoderará de más de un lector viajero. En
el artículo se cuenta la historia de un hacker que puede abrir la puerta
de cada una de las habitaciones de la mayoría de los hoteles en minutos
y con un equipo que no cuesta más de 50 dólares.

Según el experto en seguridad informática Cody Brocious, si el sistema
de puertas del hotel está conectado a la red eléctrica, él puede abrir
cualquier puerta usando un dispositivo basado en una mini computadora llamada
Arduino. El aparato que construyó es capáz de abrir cinco millones de
habitaciones en todo el mundo.

El aparato funciona como si fuera una de las tarjetas que el hotel entrega
y abre todas las puertas que utilicen ese tipo de cerradura. Consigue su
objetivo porque el código necesario para abrir una puerta se encuentra
tanto en la tarjeta de los huéspedes como en la cerradura. El invento de
Brocious lee la memoria de la cerradura y luego un programa es capaz de
abrirla. El hacker tiene 24 años y ha presentado su descubrimiento en la
conferencia Blackhat.


\vspace{0.5cm}
\hrulefill\hspace{0.2cm} \decofourleft\decofourright \hspace{0.2cm} \hrulefill
\vspace{0.5cm}

\hyphenation{pa-ra-í-so}
Panza arriba tirado en la arena de Playa Paraíso esperaba que mi esposa se
vista en uno de los baños. Diez años de casados y para festejar me la traje
a la bruja al Caribe. Dejamos a los chicos con una tía y emprendimos el
viaje. En eso estaba pensando cuando se me acercan dos señoritas vestidas
con túnicas negras y las caras tapadas. Musulmanas o algo así, y con
señas me piden que les saque una foto con su teléfono. Perdido me quedé
mirando en la pantalla la sopa de letras en glifos árabes. Cuando levante
la cabeza di un salto para atrás. En el lugar donde estaban las mujeres de
Alá ahora había tremendo par de señoritas en bikini frotándose contra
una palmera. Ahí entendí el verdadero sentido de la frase: liberación a
través del Corán. Cuando mi mujer salió del baño las turcas me tiraban
besitos en agradecimiento. No me volvió a hablar hasta que pisamos suelo
argentino. No voy más a México.


\vspace{0.5cm}
\hrulefill\hspace{0.2cm} \decofourleft\decofourright \hspace{0.2cm} \hrulefill
\vspace{0.5cm}

Vocabulario de Indigenismos en Las Crónicas de Indias. Editado por  Manuel
Alvar Ezquerra. Biblioteca de filología hispánica. Consejo superior de
investigaciones científicas. Xolope: bobo, tonto. Pues no traigan más
acá otra vez a estos xolopes (que así llamaban a los españoles, desde
que vieron a los primeros comer anonas, que es fruta de tierra caliente).


\vspace{0.5cm}
\hrulefill\hspace{0.2cm} \decofourleft\decofourright \hspace{0.2cm} \hrulefill
\vspace{0.5cm}

Y entonces ves pasar una parejita cargando sus bolsos y caés en la
inevitable, fatídica, insoslayable, infernal realidad de que vos también en
uno, dos, a lo sumo siete días, vos también te vas a tener que ir. Y
el lunes vas a tener que volver a prepárale informes a Fernández.


\vspace{0.5cm}
\hrulefill\hspace{0.2cm} \decofourleft\decofourright \hspace{0.2cm} \hrulefill
\vspace{0.5cm}

\hyphenation{pla-yas}
\hyphenation{cá-li-das}
El Heraldo de Quintana Roo informa. Playa del Carmen. Cifra récord de
asistencia a playas. En esta temporada de verano más de 50 mil personas
han acudido a disfrutar del mar Caribe.

El número se contabiliza en visitas a las playas de Cozumel y Playa
del Carmen en lo que va de la temporada de verano, y hasta el momento el
saldo ha sido blanco aunque se han dado 53 rescates, indicó el titular de
Protección Civil, Camilo Díaz de la Fuente.

De acuerdo con el funcionario municipal, entre el sábado y domingo pasados
acudieron más de 10 mil personas a disfrutar de las cálidas aguas de
nuestro mar, por lo que sumado a las visitas anteriores la cifra supera,
haciendo la plancha, los 50 mil bañistas en lo que va de la temporada
veraniega tan sólo en las playas públicas.  En temporada alta los niños
son uno de los problemas a los que se enfrentan los guardavidas, porque
los adultos se quedan dormidos tomando sol o se meten a nadar y descuidan a
los menores y estos se alejan o se pierden. Hasta el momento han rescatado
a seis niños y dos niñas.


\vspace{0.5cm}
\hrulefill\hspace{0.2cm} \decofourleft\decofourright \hspace{0.2cm} \hrulefill
\vspace{0.5cm}

\hyphenation{bue-ní-si-ma}
\hyphenation{con-ti-nua-ción}
\hyphenation{di-ga-mos}
¿Te conté de la novela que quiero escribir? Mientras
estaba de vacaciones en México leí una novela que transcurría en Villa
Gesell y me pareció buenísima. Si bien es un policial atrapante, lo que
más me llamó la atención no fue el contenido sino la forma. Esta novela,
en lugar de ser una sucesión de pocos capítulos largos, se forma con una
multiplicidad de textos cortos, uno a continuación del otro, una multitud
de voces que van ingresando al texto para formarlo. Así que voy a intentar
hacer una novela con ese formato. Un estilo aggiornado a los tiempos que
corren digamos... la gente tiene lapsos de tiempo cortos para leer (en el
baño, en el colectivo, en el auto mientras esperan a la hija que salga del
colegio) y a mi me pasa lo mismo para escribir (el domingo a la mañana,
alguna noche, un recreo en el trabajo). De hecho la empecé a escribir en
el celular mientras viajaba.


\vspace{0.5cm}
\hrulefill\hspace{0.2cm} \decofourleft\decofourright \hspace{0.2cm} \hrulefill
\vspace{0.5cm}

\hyphenation{ti-mi-dez}
\hyphenation{a-pa-guen}
Y pensar que hay gente que llega y nosotros
nos vamos. Si pudiera. Si pudiera vencer la timidez y acercarme
a hablarles. Decirles que apaguen el celular, que dejen de leer las
noticias. Política. Economía. Que se desconecten de todo eso que los está
contaminando y que por una semana o dos, lo que les dure la estadía, solo
se dediquen a descansar, a compartir con su familia y a disfrutar de sus
vacaciones, cómo yo no hice. Maldito localizador satelital que me manda
reportes y estadísticas de la oficina cada media hora.

Entonces me decidió y voy. Voy a hablarles. Me explayo, gesticulo,
grito. Ellos me miran. Vuelvo fracasado, son alemanes y no me entienden
ni jota.


\vspace{0.5cm}
\hrulefill\hspace{0.2cm} \decofourleft\decofourright \hspace{0.2cm} \hrulefill
\vspace{0.5cm}

Ahí está el transfer. Disculpe, ¿donde pega menos el aire?

Atrás. ¿Se enfermó?

No, soy asmática. Si puede ponerlo no muy fuerte, le agradezco.

Cebollina. Es como la cebolla blanca pero en miniatura. Aquí se usan las
hojas picaditas como complemento de una comida típica, el mondongo. Mi
hijita tenía ese demonio y una comadre la ayudó. Bueno, tiene que agarrar
dos cebollinas, la parte que está debajo de la tierra, picarla, hacer
un té y tomarlo frío. Mi hija tenía asma, ahora tiene trece años y no
volvió a tener problemas. Pruébelo, se va a acordar de mí.


\vspace{0.5cm}
\hrulefill\hspace{0.2cm} \decofourleft\decofourright \hspace{0.2cm} \hrulefill
\vspace{0.5cm}

Conté la realidad en lugar de una ficción para proteger a los personajes.


\vspace{0.5cm}
\hrulefill\hspace{0.2cm} \decofourleft\decofourright \hspace{0.2cm} \hrulefill
\vspace{0.5cm}

\hyphenation{pres-tar}
¿Qué vas a encontrar en este libro? Este libro recoge un puñado
de historias que tienen, en su mayoría, lugar en la península de
Yucatán. Más precisamente en la fracción de esa geografía dedicada a
los turistas extranjeros. Le voy a prestar mi voz a muchos de ellos y es
por eso que muchas partes de la trama podrán parecer la historia de uno,
pero son, en realidad, la historia de muchos. Personas que observé, seguí,
espié. Dueños de cuartos en los que husmee. Entrevistados para medios
que inventé. Algunas mujeres que soñé. Protagonistas de historias que
robé. Me metí en la piel de todas estas personas y, como un psicópata
que usa el pellejo de su víctima como tapado, me paseé con un atuendo
que era su atuendo, con una mirada que era su mirada, diciendo palabras
que eran sus palabras.



\end{document}
