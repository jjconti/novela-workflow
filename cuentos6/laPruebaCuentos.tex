 \documentclass[12pt,twoside,openright,a5paper]{book}

\usepackage[a5paper, hmarginratio={3:2}]{geometry}

\usepackage{times}
\usepackage[Lenny]{fncychap}
%\usepackage[Conny]{fncychap}

\usepackage[spanish]{babel}
\usepackage[utf8]{inputenc}
\usepackage[T1]{fontenc}

\newcommand{\personaje}{Nombre del personaje}

\usepackage{mathptmx}
\usepackage{etoolbox}

\usepackage[titles]{tocloft}

\usepackage{pdfpages}

\renewcommand{\cftchapleader}{\cftdotfill{\cftdotsep}}

% change the space before the titles
\makeatletter
\patchcmd{\@makechapterhead}{\vspace*{50\p@}}{\vspace*{0pt}}{}{}
\patchcmd{\@makeschapterhead}{\vspace*{50\p@}}{\vspace*{0pt}}{}{}
\makeatother

% change the space after the titles
\renewcommand{\DOTI}[1]{%
    \raggedright
    \CTV\FmTi{#1}\par\nobreak
    \vskip 10pt}% original: 40pt
\renewcommand{\DOTIS}[1]{%
    \raggedright
    \CTV\FmTi{#1}\par\nobreak
    \vskip 10pt}% original: 40pt


\title{La prueba del dulce de leche}
\author{Juanjo Conti}
\date{}


% Evitar viudas y huérfanas
\widowpenalty=10000
\clubpenalty=10000

\begin{document}

\pagestyle{plain}

\maketitle

%\cleardoublepage

\thispagestyle{empty}
\noindent
Edición automágica, 2014.\\

\vspace{0.5cm}

\noindent
\emph{La prueba del dulce de leche} lleva la licencia 
\emph{Creative Commons Attribution - NonCommercial - ShareAlike 3.0 Unported License}.
Esto significa que podés compartir esta obra y crear obras derivadas
mencionando al autor, pero no ha\-cer un uso comercial de ella.

\vfill

\noindent
%Más información sobre este libro:\\
http://www.juanjoconti.com.ar/la-prueba\\

\noindent
%Más libros del autor:\\
http://www.juanjoconti.com.ar/libros

\cleardoublepage

\noindent
\begin{flushright}
\emph{
\emph{La prueba del dulce de leche}\\
está dedicado a mi amigos Ale y Juan,\\
protagonistas de este libro.
}
\end{flushright}

\cleardoublepage

\renewcommand*\contentsname{Índice}

\tableofcontents

\chapter*{Prólogo}

Juanjo Conti nos presenta una nueva selección de relatos que definen su estilo.
Todos, escritos durante 2013 y 2014, llevan la marca 
que distingue la escritura y la creatividad del autor de las de otros. Esa marca es un 
conjunto inequívoco de características, tales como un tiempo por general impreciso (\emph{Visita 
al dentista}, \emph{La prueba del dulce de leche}); el uso del presente que sumerge al lector en el 
desarrollo inmediato de la historia (\emph{Bicicleta}); el uso de la segunda persona que interpela 
aún más a quien está leyendo y lo hace identificarse con el protagonista y reconocer que sus 
vicios son sus vicios (\emph{La caja}); la combinación extraña de ciertos adjetivos (\emph{obtusa 
meditación}, \emph{voracidad congelada}) que describen con eficacia cada situación e incluso, 
descripciones precisas y exactas que contienen la palabras necesarias para pintar el 
escenario o los personajes de la historia (\emph{El regalo}, \emph{La permanente}); algunos párrafos 
plagados de imágenes retóricas y de gran valor poético (el párrafo introductorio de \emph{El 
último vuelo}); la mezcla de oraciones largas con oraciones cortas que definen un ritmo 
particular de cómo narrar; las enumeraciones tanto de verbos como de sustantivos que 
detallan aún más el contexto; y, principalmente, los finales contundentes que, diciendo 
poco, dicen mucho y abren juego a la imaginación y a misterio.

En síntesis, otra obra que nos hace partícipes del entramado del texto, que nos atrapa y, 
como siempre, nos sorprende.

\vspace{0.5cm}

\begin{flushright}\emph{L.G.}\end{flushright}


%\input{carta.txt}

\input{bicicleta.txt}

%\input{garganta.txt}

\input{mosquito.txt}

\input{dentista.txt}

\input{paredon.txt}

\input{notebook.txt}

\input{regalo.txt}

\input{caja.txt}

\input{permanente.txt}

%\input{restaurador.txt}

\input{prueba.txt}

\input{pinocho.txt}

\includepdf{empty.pdf}

\end{document}
