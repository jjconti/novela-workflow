\documentclass[11pt,twoside,openright,a5paper]{book}

\usepackage[a5paper]{geometry}

\usepackage{times}
\usepackage[Lenny]{fncychap}
%\usepackage[Conny]{fncychap}

\usepackage[spanish]{babel}
\usepackage[utf8]{inputenc}
\usepackage[T1]{fontenc}

% inter parrafo
\setlength{\parskip}{10pt plus 1pt minus 1pt}


\newcommand{\personaje}{Nombre del personaje}

\usepackage{mathptmx}
\usepackage{etoolbox}


%\renewcommand{\cftchapleader}{\cftdotfill{\cftdotsep}}

% change the space before the titles
\makeatletter
\patchcmd{\@makechapterhead}{\vspace*{50\p@}}{\vspace*{0pt}}{}{}
\patchcmd{\@makeschapterhead}{\vspace*{50\p@}}{\vspace*{0pt}}{}{}
\makeatother

% change the space after the titles
\renewcommand{\DOTI}[1]{%
    \raggedright
    \CTV\FmTi{#1}\par\nobreak
    \vskip 10pt}% original: 40pt
\renewcommand{\DOTIS}[1]{%
    \raggedright
    \CTV\FmTi{#1}\par\nobreak
    \vskip 10pt}% original: 40pt


\title{El brillo de la palabra}
\author{Escritos 2012}
\date{}

\hyphenation{o-pe-ra-cio-nes}
\hyphenation{ves-tir}
\hyphenation{nues-tros}
\hyphenation{irrum-pen}
\hyphenation{co-rres-pon-di-do}
\hyphenation{di-fe-ren-cias}
\hyphenation{diá-lo-go}
\hyphenation{res-pe-tar}
\hyphenation{tro-pe-zar}
\hyphenation{es-cri-tu-ra}
\hyphenation{la-bo-ral}
\hyphenation{es-tu-diar}
\hyphenation{li-te-ra-tu-ra}
\hyphenation{cre-an-do}
\hyphenation{tro-pe-zar}
\hyphenation{o-pe-ra-ti-vo}
\hyphenation{va-rios}
\hyphenation{e-vi-den-te-men-te}
\hyphenation{des-co-no-ci-das}
\hyphenation{bri-llo}
\hyphenation{ber-me-llón}
\hyphenation{gla-mour}
\hyphenation{ma-es-tro}
\hyphenation{co-ti-dia-na}
\hyphenation{nues-tro}
\hyphenation{des-fi-gu-ró}
\hyphenation{a-cep-ta-bas}
\hyphenation{ca-rac-te-ri-za-ba}
\hyphenation{a-po-ya-da}
\hyphenation{cer-te-ro}
\hyphenation{ins-truc-cio-nes}
\hyphenation{co-rrió}
\hyphenation{su-mi-da}
\hyphenation{su-bli-me}
\hyphenation{ló-gi-ca}
\hyphenation{cho-che-an}
\hyphenation{per-ge-ñó}
\hyphenation{Ma-nuel}
\hyphenation{pa-re-cía}
\hyphenation{ve-a-mos}

\begin{document}

\pagestyle{plain}

\maketitle
\cleardoublepage
\emph{El brillo de la palabra} es un Taller literario que nació en el mes de agosto de 2012 en la calle Piedras 7370 del Barrio de Guadalupe, de la ciudad Capital de Santa Fe.

La idea y coordinación está a cargo de Marilina Renk (Profesora de Letras) y Norma Kenda (Psicopedagoga).

\vspace{1cm}

Los Talleristas dicen ser:

\vspace{0.5cm}

Ana Krawchuk

Belkis  Ramírez

«Constanza Luna»

Dolores Valverde

Gustavo Renk

Hernán Renk

José «Willy» Cámara

«Juanjo» Conti

Nelly López

\vspace{1.5cm}

%Sitio web: www.elbrillodelapalabra.blogspot.com.ar

%\begin{center}
%http://www.elbrillodelapalabra.blogspot.com.ar
%\end{center}
\begin{center}
http://www.elbrillodelapalabra.com/
\end{center}

% inter parrafo
\setlength{\parskip}{0pt plus 1pt}


\tableofcontents

\chapter*{Palabras para eliminar}
\addcontentsline{toc}{chapter}{Palabras para eliminar}

En este trabajo de compilación y edición que ofrecemos al lector privilegiado, sólo nos hemos preocupado por alcanzar una unidad. 

Aquella unidad dispersada en el aire, que con cada lectura en el Taller se delineaba trazo  a trazo. 

Ése fue y seguirá siendo nuestro plan de trabajo: la escucha incesante y respetuosa de quienes nos congregamos en un lugar común, con intereses  semejantes.

Porque escribir y leer implican procesos muy diferentes y sólo en la entrega al «otro» es que se alcanza el verdadero sentido.

Palabras sueltas, algunas que escasean, otras que no consiguen fieles compañías. 

La palabra se muestra rebelde ---rebelde y mezquina, como alguien supo escribir en alusión al idioma---, y está en cada uno el saber tratar con ellas.

Si elegimos darle \emph{brillo} es porque sostenemos que, aún, con la palabra más gastada y opaca del lenguaje, puede producirse el milagro de nuevos significados.

Fundamos un encuentro semanal, con una lógica propia y nos dimos el gusto de movilizarnos y de movilizar personas, tanto desde lo físico como desde  lo emocional.

En cada reunión, vibramos  junto  a ellos, los vimos crecer, tropezar, acordar o rechazar ideas, con la complejidad que define a cada ser humano, en una postura particular de ver y explicar la vida. Una síntesis del oficio, que comprueba que a la apertura mental se  llega con la ejercitación y el dominio de la palabra.

Por todo lo expresado es que estamos convencidas que, aquí,  las palabras «no se las llevó el viento», sino  que quedaron apresadas en cada hoja de este libro, como testimonio fiel de un esfuerzo personal y colectivo.

\begin{flushright}Norma y Marilina\end{flushright}

\part*{Agosto\\«El mes para volver a ser niños»}
\addcontentsline{toc}{part}{Agosto}
\chapter*{La caperucita}
\addcontentsline{toc}{chapter}{La caperucita}

%Marilina
\vspace{0.5cm}
\emph{Las palabras no podían ni querían esperar más. Necesitaban brotar de cada Tallerista como el retoño de una nueva planta. Es que celebrábamos el inicio de un nuevo espacio de lectura y escritura y «ellos», los protagonistas, deseaban liberarse de esas \emph{ataduras} de palabras. Es por eso que ideamos una excusa para hacer rodar la imaginación, ésta vez, cuesta abajo al pasado. Y fue el tema de la infancia el que desencadenó una variedad de escritos que, ahora, apreciaremos en contraste.}

\emph{La primera en llegar al «bosque del taller» fue \textbf{Caperucita}. Aunque, la pequeña conservaba muy poco de los rasgos que los hermanos Perrault idearon para ella. En nuestro ámbito, Caperucita se vio multiplicada en otras versiones y con otros caprichos, propios del hombre moderno:}

\section*{Caperucita}
\addcontentsline{toc}{section}{Caperucita (Por Belkis)}

\begin{flushright}\begin{flushright}Por Belkis\end{flushright}\end{flushright}

Antes que nada, me voy a presentar. Me llamo Florencia, tengo 6 años, voy a primer año y mi apodo es «Flor». Vivo muy cerca de una playa y junto a mi casa vive mi abuela Carolina. ¡Cuánto la quiero! Siempre estamos juntas, con ella me siento muy querida, protegida. Lo que más me gusta de mi «abu», porque así la llamo, es que me vive contando y leyendo cuentos, de hadas, de princesas. Pero, el que más se me grabó fue el cuento de Caperucita Roja. No sé el motivo, tal vez sea porque uno de los protagonistas principales era la abuelita.

Un día recibo un llamado por teléfono, era mi «abu» que me decía: 

---«Florcita, te espero en la playa, hoy es un hermoso día y podemos jugar juntas en la arena; te haré el castillo que tanto te gusta. Trae todos los elementos de playa y, de paso, nos sentaremos a tomar unos mates». ¡Qué contenta me puse! En seguida le pedí permiso a mi mamá y ella me dijo: «Justo estoy terminando esta torta de manzana, así que la llevas para acompañar el mate»
Preparé todas las cosas y salí corriendo. ¡Qué hermosa estaba la tarde, en el aire se sentía el perfume de las flores de algunos jardines, a medida que pasaba junto a ellos. Ya faltaba poco para llegar cuando veo en la playa muchísima gente amontonada, algunos llorando y otros que decían: 

---«¡Pobre abuela cómo quedó!»

Yo quería correr ligero, y pensé: ¿Será mi abu? ¿Le habrá pasado algo?

A mi mente se me vino el cuento de Caperucita y el lobo:¿Habrá algún animal que la lastimó? Quería llegar rápido, pero no podía, qué desesperación. «A mi abuelita no quiero que le pase nada» y comencé a llorar y llorar. Cuando me faltaban centímetros para ver lo que había pasado, siento que mi mamá me decía: «¡Flor, Flor, levantáte, que tenés que ir al colegio» 
Y cuando me desperté tenía toda la cara mojada por mis lágrimas. Pero agradecí tanto a Dios. ¡Todo había sido un sueño! 

\section*{La pimienta de la loba}
\addcontentsline{toc}{section}{La pimienta de la loba (Por Gustavo)}

\begin{flushright}Por Gustavo\end{flushright}

Corría el siglo XXI. Don Lobo, como lo conocían sus pacientes, tenía un consultorio odontológico medianamente reconocido en una pequeña ciudad. Con los años, las exigencias hicieron que éste contratara a una secretaria para que organice su humilde agenda. 

Se trataba de Caperucita, una joven de curvas exuberantes y con enorme eficiencia en tareas administrativas.

Don Lobo comenzó a observar que ahorraba mucho tiempo con el desprendimiento de los salvajes papeles a manos de su ayudante. Podía de esta manera tomar más turnos diarios. Su satisfacción laboral era casi plena.

En cambio su esposa Doña Loba estaba enfurecida con los cambios acaecidos en el entorno laboral de su marido. Su baja autoestima sumado a la belleza externa que veía en Caperucita, acrecentaron a pasos agigantados sus sospechas de engaño por parte del lobo.

Todas las mañanas se despertaba y salía expulsada hacia el espejo del baño a mirar su cabeza pese a haber corroborado previamente de modo táctil la curvatura de su cráneo. La no detección de ellos no detenía su desconfianza.

Pasaron unos meses y los celos de la esposa se hicieron insoportables. Así Don Lobo pese a su total amor y fidelidad marital y a la eficiencia de su secretaria terminó despidiéndola.

No conforme con la determinación, Caperucita se dirige hacia la casa de su ex jefe para pedir mayores explicaciones. Toc-toc ¿Don Lobo está? ¿Quién es?, responde Don Lobo,  Caperucita. Adelante está abierto.

La esposa no estaba pero al llegar y ver a su enemiga en su propio living cree corroborar su desconfianza.

Se genera una feroz discusión: «pero sos una roba maridos, pero cállate, vieja cornuda». El esposo intenta calmarlas pero con una rapidez «matrixiana», la loba toma el gas pimienta de su cartera y lo dirige a los  ojos de su marido y de Caperucita (puede ser la joven pero caperucita tiene mas impacto). Luego toma una olla con agua hirviendo y se la arroja a la joven, para acto seguido clavarle un cuchillo en el estomago. 

Al recuperar la visión Don Lobo, la muerte de su ex secretaria y el estado de shock de su esposa eran hechos consumados.

\section*{Caperucita descontenta}
\addcontentsline{toc}{section}{Caperucita descontenta (Por Dolores)}

\begin{flushright}Por Dolores\end{flushright}

Había una vez una Caperucita Roja, que descontenta dejó el bosque y se fue a la ciudad.

Allí conoció autos, casas con todo el confort, aviones, trenes velocísimos, ricos helados, la TV y las computadoras. Ella, también, se dejó ganar por la novedad y la comodidad. No iba al cine porque en la tele veía películas, no visitaba a la gente, porque total chateaba con sus amigas.

Pasado el tiempo, empezó a sentirse mal, perdió el dorado de la piel y se le apagó el brillo de sus ojos.

Un día recibió una extraña invitación para encontrarse con un señor, fue un encuentro raro, no debía decirle a nadie a dónde iba. Pero, la cita resultó un fracaso. El hombre mostró sus garras.

Caperucita no veía niños jugando en las veredas y menos personas grandes, salvo los apurados que hacían algún trámite. Todos estaban en casa frente a ese aparato, los había atrapado como el lobo, nada más que la gente no se daba cuenta y era un lobo sofisticado. 

Y Caperucita decidió volver al bosque entre la magia del viento, entre los árboles y los pájaros que la despertaban a la mañana. Caperucita volvió a tener, así, la lozanía y tranquilidad, allí donde los hombres son hombres y los lobos son lobos.

\section*{Caperucita Negra}
\addcontentsline{toc}{section}{Caperucita Negra (Por Norma)}

\begin{flushright}Por Norma\end{flushright}

A Caperucita Negra, que estaba creciendo a pasos agigantados, le gustaba caminar. Pero, sabía, porque era muy intuitiva, que andar sola era muy peligroso.

Ya su abuela, con la que hablaba constantemente, le había relatado viejas historias de lo mal que lo habían pasado las otras caperuzas, a través de milenios.

Su Pueblo, que así se llamaba, «Pueblo», tenía la atractiva característica de que todas sus mujeres, desde que nacían, eran iniciadas en el uso de la caperuza.

Sólo se les permitía elegir el color, atributo que no dejarían hasta su muerte.

¡Qué injusticia! ---pensaba la joven, ya que uno de sus mayores deseos era lucir su esplendida cabellera.

La bella comenzó a entristecer y creyó necesario cambiar los caminos que había recorrido hasta entonces.
Su astuta cabecita pergeñó una idea, pero, ¡oh!, se dio cuenta que estaba sola para cargar con semejante destino. ¿Era igualmente aceptado por todas las mujeres que habitaban Pueblo?

La tarea de golpear, puerta por puerta, no se hizo esperar y cuando emprende la atrevida búsqueda, siente la sensación de que alguien la persigue, la espía. Era evidente, por el murmullo que producían los arbustos y enredaderas que encontraba a su paso. Hasta que la súbita presencia de un lobo le aseguró que no estaba equivocada. Una sonrisa, con dientes amenazantes, no impidió que continuara con la tarea propuesta.

Por otro lado, la abuela, al ser consultada por la joven, aprobó con orgullo la valentía de caperucita Negra.

La convocatoria se iba a llevar a cabo en los próximos días en la Plaza Pueblo, donde asistirían las mujercitas que comenzaban a entender el motivo de la protesta y la posibilidad de que sus melenas quedaran liberadas para siempre.

Cuenta la historia escrita por leñadores que, el primer Sindicato de Caperuzas Liberadas estaba naciendo con la fuerza de todos. Esa energía acumulada desde muy antigua data, se había transformado, por la gestión de nuestra Caperuza, en un palpitante deseo que comenzaba a ver la luz.


\section*{¡¿A dónde vas Caperucita?!}
\addcontentsline{toc}{section}{¡¿A dónde vas Caperucita?! (Por Marilina)}

\begin{flushright}Por Marilina\end{flushright}

Corría el año 2001 cuando en una biblioteca de Buenos Aires (La Mariano Moreno) sucedió este hecho mágico que habré de relatar. Sepa el lector armarse de ingenio y no sucumbir ante la maniobra que realidad y fantasía tramaron para él:

La acción se desarrolló muy temprano cuando bibliotecóloga y secretario recién se desemperezaban en sus respectivos domicilios con el poco ánimo de ir a trabajar.

En el salón de la biblioteca comenzaron a escucharse ruidos extraños. Parecía que las maderas de los armarios crujían  en complicidad con la víctima. Y en un abrir y cerrar de puertas, los libros emprendieron la fuga, entre ellos la colección de los hermanos Grim.

Al llegar al umbral del edificio, muchos fueron los ejemplares que, atónitos, se miraron y dijeron para sí: 

---Qué haremos allí afuera, fueron tantos años en este sitio, que no habrá otro lugar mejor, ni estante más mullido para velar nuestros sueños.

Y al aceptar esto, poco a poco y con «el lomo a cuestas» emprendieron la retirada. Una procesión de libros regresaron; cada uno a su sector, escalando posiciones y haciendo el menor bullicio, para que allí nadie se diera cuenta del «paseo de palabras» 

Pero aunque la clama volvía a reinar, no todo estuvo igual. Una de las historias decidió escapar: «¡Caperucita, Caperucita! ---dijo el hada madrina--- ¡A dónde se ha ido, la pequeña no está!» Y el lamento invadió a todos, incluso a la princesa, su principal rival.

Mientras tanto, Caperucita ganó la ciudad, atravesó avenidas larguísimas y se preguntaba:

---¿Qué bosque tan gris es éste? ¿Se habrán cansado del verde tal vez? ---Caperucita caminaba y caminaba, cuando empezó a ver una multitud de gente que golpeaba utensilios de cocina entre sí. «¿Tendrán demasiado hambre quizá?», se dijo y comenzó a seguir a la muchedumbre, hasta que ésta se detuvo en una plaza gigante. Caperucita estaba algo asustada, oía unos truenos terribles, pero al mirar al cielo, el sol estaba radiante.

De repente, las personas comenzaron a gritar y un sonido ensordecedor la hizo actuar. Justo detrás de ella, un hombre era metido en un vehículo con alas. Al ver eso, Caperucita (que no había perdido su espíritu de aventurera) le pidió al señor que por favor la llevara con él. El hombre muy apurado y pensativo, le dijo que sí sin vacilar y, en cuestión de segundos, empezaron a elevarse.

---¿Cómo te llamas? ---lo interrogó la niña.

---Eh, soy Fernando, un hombre cualquiera

---Qué raro se vive allá bajo Fernando, me sentí acorralada por la gente. ¡No sabés cómo  te entiendo querida! ---contestó el señor.

---Es más ---dijo Caperucita--- creo que ya no podré regresar más al lugar que tenía, ya nada será igual para mí ---se lamentó la pobre.

---¡Ya lo creo! ---agregó Fernando, sintiendo que en algún punto ambos compartían las mismas sensaciones.

---¿Tú saliste de algún cuento de hadas, Fer?

---No hija mía. Yo escapé de una triste realidad, en la que creí ser un lobo para defender a mi nación y terminé siendo un gran patán.

El helicóptero tomó altura y Caperucita sacó de su canasta el pastel un tanto añejo, que su abuelita olvidó comer. Nada más se supo de ella, y del paradero del tal Fernando muchos son los que quieren olvidar.

\section*{La convención}
\addcontentsline{toc}{section}{La convención (Por Juanjo)}

\begin{flushright}Por Juanjo\end{flushright}

El 22 de agosto de 2012 se llevó a cabo en la ciudad de Santa Fe una reunión muy particular. Esta tuvo lugar en el salón de convenciones del hotel Los Silos, ubicado en el puerto de la ciudad, junto al casino.

Si se lo miraba desde lo pisos superiores, el lugar era un campo de frutillas. Las caperuzas cubriendo las cabezas de todas las invitadas casi no dejaba ver el gris del mármol que pisaban.

Había caperucitas de alma y caperucitas de profesión. Había caperucitas quinceañeras y caperucitas que ya eran abuelas.
Estaba, por ejemplo, la reconocida actriz Lucía Vertucci, protagonista del éxito Caperucita y el motochorro. E incluso estaba Caperucita Rosa, cuyo nombre de bautismo era Carlos.

En total eran más de 200 personificaciones del legendario personaje de rizos dorados y sonrisa inocente.

El objetivo de la convención era dividirse las rutas para ir a visitar a las correspondientes abuelitas, ya que en una ciudad tan chica, muchas se encontraban cuando tenían que perderse y eso no era ninguna gracia en los cuentos. No sabían quien era la organizadora de semejante reunión, pero todas coincidían en que era una estupenda idea.

Cuando el maestro de ceremonia se disponía a hablar, la música bajó, las luces se atenuaron y todos hicieron silencio. El orador se aclaró la garganta tosiendo y su tos retumbó en toda la sala. Se acomodó las gafas sobre las puntiagudas orejas y tomó entre sus manos cuatro o cinco hojas de papel. No había empezado la primer oración de su discurso cuando caperucita, una, notó que las puertas se habían cerrado y que de la espalda del maestro de ceremonia asomaba una horrible, enorme y peluda cola de lobo.


\chapter*{Carta a la maestra}
\addcontentsline{toc}{chapter}{Carta a la maestra}

%Marilina
\vspace{0.5cm}
\emph{Infancia, tiempo de sonrisas detenidas, de caricias prolongadas, de golosinas deseadas y deseantes. La infancia de los abuelos, de los padres, nuestra infancia. Juntas se entremezclan y traen a la memoria los recuerdos desdibujados de esos años inolvidables. Allí, también emergen rostros y personas que fueron marcando éste, nuestro sendero propio. Entre ellos, los maestros: grandes hombres y mujeres, portadores de la palabra y de la acción. Como no los teníamos «a mano» decidimos enviarles una carta imaginaria que recompense, en parte, tanta gratitud:}

\section*{Recordando a mi maestra}
\addcontentsline{toc}{section}{Recordando a mi maestra (Por Nelly)}
                                                                                \begin{flushright}Por Nelly\end{flushright}

Hay una mujer que permanece en mi recuerdo iluminando mi interior. A esa mujer todo mi agradecimiento; dedicada, trabajadora, abnegada, mi querida maestra de la primaria.

Cómo olvidarla si cuando llegué  el primer día a la escuela «Manuel Belgrano» del barrio Sur todo me llamaba la atención, el patio muy grande, los salones, los bancos. Todo iba descubriendo con mucho asombro.

Llegué apretadita de la mano de mi hermano que no abandonó la mía hasta el talán-talán de campana que escuchaba por primera vez. Él  aflojó su mano y se fue. Quedé solita, en medio del barullo de otras niñas, sentí miedo. Sonó fuerte una voz «Nelly López» A mí me llamaban.  Mi pequeño corazón saltaba y saltaba, no sabía qué hacer. Me tomaron del brazo, «¿Sos Nelly López?»

---Sí ---contesté con voz muy bajita.

---Ven querida, soy tu maestra, vamos a formar una filita, desde hoy yo te cuidaré. 

Qué lindo lo que escuché, en ese instante comencé a quererla, cómo no quererla si cuando miré sus ojos me miraban con amor. Miré su boca que sonreía y sus manos que me contenían, había comprendido mi timidez, ¡cómo se parecía a mi mamá!

Señora Isabel Pazeth, en tantos años es la primera vez que en letras la recuerdo. Sepa que nunca se fue de mi corazón; lloré cuando me enteré que joven aún, se elevó a la casa eterna. Seguramente estará rodeada de angelitos a los que con su amor tan especial les dirá: eme-a- ma, eme-a- ma las dos juntas, «mamá».

Beso su recuerdo, me abrazo a la ausencia.

\section*{Querida maestra Sofía}
\addcontentsline{toc}{section}{Querida maestra Sofía (Por Dolores)}

                                                                                 \begin{flushright}Por Dolores\end{flushright}

Esta carta es para que se entere que yo la recuerdo con mucho cariño. Usted fue buena conmigo, supo enseñarme con dedicación y paciencia.

Aquella tarde que salté la ventana de la vieja Escuela Nº 417, para ir a comprar caramelos, usted con sabias palabras me hizo entender que yo había hecho algo fuera de lugar.

Luego,  lo comentó en la libreta de fin de mes, para que mis padres se enteraran. Ellos también, me retaron. Aunque no muy enojados, así que la cosa quedó así nomás.

Después de muchos años, cuando nos encontramos aquí en Santa Fe, usted tuvo la delicadeza de no comentar lo sucedido. La saludo con mucho afecto. Dolores.

%Marilina
\vspace{0.5cm}
\emph{Dolores alterna prosa y poesía, tratando de abrigar con sus estilos tanto amor dedicado a la enseñanza:}


\section*{Maestros}
\addcontentsline{toc}{section}{Maestros (Por Dolores)}

\begin{flushright}Por Dolores\end{flushright}

\begin{center}
La maestra toma la tiza

y escribe en el pizarrón:

la fecha,

un dato a consignar.

Y así, día por día, 

Matemática, Geografía,

el placer de la Historia Argentina.

El cuento, la poesía

y el reino natural.

No importa el guardapolvo a cuadrito,

si es blanco o es azul.

Lo que importa es la mano que posa.

La palabra, la caricia,

el llanto que calma,

o el límite que pone

cuando intenta educar.

Cura, cocina, aconseja.

Camina entre piedras y barro,

para poder llegar.

Es una vocación,

Es un apostolado.

Y así es su sueldito

con el que come, educa,

compra útiles y hasta,

vive ella de su magro mensual.

Tiene tiempo de jugar, 

de reír, de cantar.

Hay música en su alma.

De tanto derrochar caricias,

se le ha vuelto rítmico su hablar.

Y en el corazón, una clave de sol

le da ritmo a sus oídos .

Y cuando va la bandera a enarbolar,

todos quedan unidos,

bajo la azul y blanca.

¡Adelante, siempre adelante,

siempre a enseñar!
\end{center}

\section*{A mi maestra}
\addcontentsline{toc}{section}{A mi maestra (Por Belkis)}

\begin{flushright}Por Belkis\end{flushright}

Busco en mis recuerdos alguna anécdota, de mi paso por las aulas y se me presentó tu imagen, querida maestra reemplazante. Estaba en cuarto grado, ese día nuestra Señorita Claudia había faltado, porque se encontraba enferma y vino la Directora a comunicarnos que, por algunos días, íbamos a tener otra docente.

Cuando la veo llegar y presentarse ante todos nosotros, grande fue mi sorpresa... ¡era mi prima! Estaba recién recibida, me puse muy contenta y pensé para mis adentros: «con ella la pasaré requete bien». Pasó ese día, todo fue normal y tranquilo. Pero, para el día siguiente ya nos dio bastante tarea y de estudiar. Ni les cuento... ¡era todo sobre Belgrano!

A la tarde, cuando fui a su casa (éramos vecinas), empecé a preguntarle si le había gustado el grupo y cuántos días se iba a quedar. Me contestó que éramos muy revoltosos y, según parecía,  iban a ser bastantes días. Y fue ahí que me recalcó: 

---Así que espero que seas una de las más estudiosas, porque mañana la primera que va a pasar a dar la lección, vas a ser vos.

Yo me reía y le contesté: «no te creo nada»

---Te conviene que estudies para mañana ---dijo finalmente.

Cuando llegó la hora de dar la lección, ¿adivinen a quién llamó? Sí, fue a mí y yo ni siquiera había abierto el libro, no sabía nada de nada. Bueno, enojada no estaba. Pero, por primera vez la vi muy seria. Cuando llegó el recreo, me apartó y me preguntó: 

---¿Por qué no estudiaste, si vos estabas avisada de que te iba a llamar? ---Yo bajé la cabeza, no sabía qué contestarle.

Pasaron los días y cada vez que preguntaba algo, ¿a quién llamaba? Sí, a mí...

Pero, desde ese papelón que pasé el primer día, empecé a estudiar. No fui la mejor, aunque estaba muy interesada en crecer cada día más. Con ella aprendimos el significado de las palabras amor, compañerismo, responsabilidad, solidaridad, confianza, respeto...

Y a pesar de que ya no estés en este mundo, sé que siempre estarás presente en nuestros corazones. ¡Mil gracias, prima querida, por todas tus enseñanzas!
\clearpage
\section*{Carta al maestro}
\addcontentsline{toc}{section}{Carta al maestro (Por Gustavo)}

\begin{flushright}Por Gustavo\end{flushright}

Profe, te escribo para informarte que, aún, no he podido concretar el que señalaste como el más importante de tus consejos. Sí terminé la secundaria con esfuerzo, pero con mucho goce y un puñado de muy buenos amigos. Luego empecé la Universidad y hace 29 días me recibí, después  de largos, pero rápidos años. Y ni te cuento de las peleas constantes con los profesores. Como lo sabés, siempre fui un alumno complicado, tal vez no para vos. Pero sí para el que no tenía ganas de escuchar al alumno o creía sabérselas todas. No me arrepiento de eso, me recibí pero nadie me pasó por encima.

Por último, te cuento que la mujer llena de «guita»  que aconsejaste como la ideal, ya va a aparecer. Si querés darme una ayuda desde arriba, no vendría nada mal.

Un beso. Gus.


\section*{Carta para la maestra}
\addcontentsline{toc}{section}{Carta para la maestra (Por Norma)}
                                                                          \begin{flushright}Por Norma\end{flushright}

Hablar, escribir, leer y todo lo que se relaciona con la palabra fueron mis barreras para comunicarme con los otros, ya que mi madre lo hizo por mí hasta llegar a la adolescencia y comenzar la escuela secundaria.

No aprobar Lengua o Castellano marcaron el desafío mayor con el que tuve que luchar para, prácticamente, aprender desde cero a musitar con mucha vergüenza mis primeras propias palabras.

Fuiste, tú, maestra particular, la que me acompañó con una pasión desconocida por mí,  a encarar la hoja en blanco.

Desde ese estreno hasta ahora, no he parado de expresarme en los mil ámbitos que he
encontrado a lo largo de mi vida. ¡Gracias, querida maestra!

%Marilina
\vspace{0.5cm}
\emph{Tinta que brota con nostalgia. Aunque hay otras que emanan de la pura ficción y convierten al lenguaje en un hecho solemne, para el acto de escritura y para regocijo del propio lector:}

\section*{Carta a mi maestra}
\addcontentsline{toc}{section}{Carta a mi maestra (Por Juanjo)}

\begin{flushright}Por Juanjo\end{flushright}

\begin{flushright}
Durante los años que vivió en Santa Fe,\\
Vladimir Medina participó de un taller literario llamado El brillo de la palabra.

El siguiente es uno de sus trabajos.
\end{flushright}

Querida maestra de tercer grado, te escribo en medio de un taller de escritura. El título de la consigna es «Carta a mi maestra» y el objetivo es practicar la prosa epistolar recordando hechos del pasado. Si el resultado tiene forma, probablemente te la haga llegar.

Nos conocimos el primer día de clases del año 1993. Recuerdo tu pelo negro atado con un moño, tus ojos celestes y tus dientes como perlas. Tu sonrisa, cuando se dejaba ver, era muy linda.

Con el correr de los días, los demás alumnos te empezar a referir con el mote de «la loca»,  por la cantidad de tarea que nos dabas. Yo nunca hice caso a esos juegos de niño y, muy por el contrario, siendo por esos tiempos un sabelotodo insufrible, eras mi preferida. Creo no equivocarme al suponer que la preferencia era mutua.

Recuerdo un día en particular en el que era tanta la tarea de matemáticas para el hogar que el salón de clases se amotinó y exigió hablar con la directora. Flojo de personalidad en esos años de mi vida, no hice otra cosa que unirme al grupo para pedir tu cabeza. Uno de mis compañeros le dijo a la directora que lo habías golpeado y otro, antes, le había propinado una paliza para que la historia tenga pruebas que la sustenten. 

Por supuesto que lo negaste todo, pero un aula llena de chicos entrenados para la mentira no es fácil de doblegar. Durante los días que siguieron, distintos miembros de aquel clan demoníaco repitieron la treta. Ojos en compota, cortes en los brazos e incluso dientes caídos formaban el parte médico del aula.

El niño de la mentira original nos hizo prometerle que haríamos saber de esta historia clínica en nuestras casas y tu expulsión del establecimiento se concretó cuando un grupo de padres tomó cartas en el asunto.

Todavía recuerdo tu mirada. Había decepción en esos ojos celestes. No mirabas a los demás, solo me veías a mí. A ese pequeño traidor.

Años después, siendo ya adulto, buscando en una hemeroteca me enteré que había sido de tu vida. No pudiste soportar el acoso de la prensa y el repudio de tus vecinos y tu sobrenombre se volvió realidad. Con un poco más de investigación encontré la dirección del lugar donde estabas internada y, como una forma de acallar mi culpa, tomé el turno de la noche. Como enfermero en el manicomio me encargo de que tomes a tiempo tus pastillas y te doy de comer. Para la cena hay puré de calabaza, espero que te guste.

Tu alumno preferido.

\part*{Septiembre\\«Tiempo para revelar el Tiempo»}
\addcontentsline{toc}{part}{Septiembre}

\chapter*{Apresando a Tiempo}
\addcontentsline{toc}{chapter}{Apresando a Tiempo}

%Marilina
\vspace{0.5cm}
\emph{Nos preguntamos a diario por qué se va la vida en un suspiro, por qué hay instantes que se nos asemejan a la eternidad. ¿Será porque hay un tiempo propio, que no sabemos o queremos gobernar? ¿Tiempo que nos pertenece, o pertenecientes a un tiempo? En esa encrucijada echamos a rodar palabras, queriendo así aprisionar al tiempo:}

\section*{Al ritmo del tiempo}
\addcontentsline{toc}{section}{Al ritmo del tiempo (Por Dolores)}
                                                                                    \begin{flushright}Por Dolores\end{flushright}

Ayer vino mi nieto a visitarme. Al verme vestida para salir preguntó:

---¿Salís abuela?

A lo que respondí: 

---Sí, voy a un té bingo con unas amigas.

---¿Y ayer, saliste?, te llamé y no contestaste.

---Sí, ayer fui a jugar canasta. Y mañana tengo Taller de literatura. El jueves voy al cine y el viernes a la peluquería. Como vez, tengo todos los días ocupados... Pero, sentáte, charlemos un ratito, todavía es temprano.

---Abuela, te olvidaste del sábado y el domingo.

---¡Ah, sí!, el sábado voy al teatro y el domingo almuerzo con ustedes, y a la noche voy al club de los abuelos. Como veraz, este es mi tiempo. Mientras pueda salir y gozar de la vida, de las amistades, me siento bien. Algunos males me aquejan, pero puedo sobrellevarlos. No quiero ser una abuela quejosa, encerrada en la casa, malhumorada, sin ver que todas las mañanas sale el sol para alumbrar el día y llenarte de esperanzas de otro día mejor.

\section*{Es tiempo}
\addcontentsline{toc}{section}{Es tiempo (Por Ana)}
                                                                                                        \begin{flushright}Por Ana\end{flushright}

Es tiempo de primavera, mi amor...

Es tiempo de amar.

Es tiempo de cosecha.

Es tiempo de siembra.

¿Te duelen las piernas, la cintura?

«Es el tiempo», te dicen.

El tiempo-Sí, ¡el tiempo que hace que has nacido!

¡Todo tienen su tiempo!

Dale tiempo al tiempo, y marchará sobre ti, 

lanzando sin compasión sus huestes de años,

que harán añicos tu juventud, dejando

Sólo devastación a su paso.

¡El tiempo! Tiempo corto de los años vividos,

largo tiempo de los años perdidos.

¡Todo lo puede el tiempo! Amo y Señor.

¿Quién puede detenerlo?

¿Qué muralla?

¿Qué fuego puede demorar su paso?

Él prosigue su marcha implacable,

llevando en sus alforjas tus vivencias.

No camina solo, marchan a su lado

sus obsecuentes cómplices: la vida y los años.

Marchan siempre juntos, tomados de la mano.

\section*{Sr. Tiempo}
\addcontentsline{toc}{section}{Sr. Tiempo (Por Willy)}
                                                                                           \begin{flushright}Por Willy\end{flushright}

En el día de ayer fue detenido el Sr. Tiempo, viejo conocido de la zona que merodea por todos lados pero no está en ninguno en particular. Está acusado de varios delitos: el de envejecer a la gente, en hacer creer que el olvido algún día llegará con su paso; también que el perdón llega con el transcurso de su lento pero seguro caminar.

Llegó el día del juicio contra Tiempo, se lo veía algo abatido pero erguido, con la mirada siempre vivaz, como un niño queriendo descubrir todo lo nuevo a su alrededor. 

Tras un pequeño silencio de espera, el juez dio por iniciado el juicio y paso siguiente comenzó la alocución del abogado defensor, quien expresó lo siguiente: 

«Todos culpamos al tiempo... ¡es tan fácil hacerlo! La humanidad siempre buscó una explicación a sus limitaciones y debilidades. Por eso inventaron el tiempo. Los griegos consideraron al tiempo como algo cíclico que se repite indefinidamente: estamos destinados a vestir un cuerpo diferente después de cada reencarnación. El alma va ocupando otro cuerpo y debe cumplir un castigo o una vida mejor a la que tuvo en sus vidas anteriores. Siempre en forma cíclica.

Con el cristianismo el tiempo pasó a ser lineal: hubo un antes de Cristo y un después de Cristo. Esa forma de ver el tiempo dominó por dos mil años en nuestra civilización. San Agustín consideró que el tiempo fue creado por Dios cuando creó el mundo, antes de eso no había tiempo, sólo eternidad.

Con Einstein comenzó otra forma de ver las cosas y relativizar el espacio y el tiempo, todo es relativo, incluso el tiempo. El tiempo no es igual en cualquier punto del universo, varía según donde estemos, y eso nos llevó a pensar las cosas de otra manera. 

En el Siglo XX un filósofo francés, Henri Bergson, consideró que no medimos el tiempo sino el espacio que recorre la aguja en un reloj. El tiempo no pasa, pasamos nosotros: el yo vive el presente con el recuerdo del pasado y la anticipación del futuro, que sólo existen en la conciencia que los unifica. Los instantes valen de diferente modo, un momento penetra en otro y queda ligado a él. Es inútil ir a la búsqueda del tiempo perdido: no hay reversibilidad del tiempo. El tiempo es nuevo a cada instante y requiere un método específico.

Por eso señores ---continuó el abogado--- ¿podemos culpar al tiempo? Le dimos una entidad real para explicar ciertas cosas que no tienen explicación. Dejo en su conciencia su dictamen final, y más allá de lo que digamos y hagamos, el tiempo continuará su marcha tan lejos del bien como del mal... porque él, vaya paradoja, no tiene tiempo, no tiene edad.

Muchas gracias señores del jurado»
\clearpage
\section*{Tiempo}
\addcontentsline{toc}{section}{Tiempo (Por Norma)}
                                                                                                         \begin{flushright}Por Norma\end{flushright}

El Señor Tiempo Libre tomó por asalto un camping privado, para hacer de las suyas.

El hecho de girar y girar sobre sus pies con movimientos que recuerdan las danzas circulares, le otorgó un poder diferente al que había tenido hasta entonces. Se preguntó con interés con quién podría compartir la novedad.

De esta manera, busca introducirse dentro de los hogares habitados. Pero, mirando hacia atrás, observa el evidente deterioro sufrido después de su apresurado paso y queda perplejo.

Los vecinos alertan a la autoridad máxima, quien envía a la policía, para detener a Sr. Tiempo Libre. Éste había demostrado una extraña conducta erosiva y había perdido el sentido de su vida.

Por esta circunstancia, las más afectadas fueron las cuatro estaciones que, junto con Vivaldi, solicitaron una entrevista. De esta forma, le recordarían a Señor Tiempo Libre cómo producían cuando se respetaba el paso de los años.

Es en ese momento que Sr. T. Libre solicita apelar a la Sabiduría del Mono Relojero, para que contemporice con la institución policial su detención. Así, continuarían con su trabajo de siglos y siglos. Es que la escapada del mencionado le había tendido una trampa.
\clearpage
\section*{El tiempo}
\addcontentsline{toc}{section}{El tiempo (Por Hernán)}
                                                                                                \begin{flushright}Por Hernán\end{flushright}

Se puede decir que el tiempo es una dimensión. Para ser mas precisos, es la cuarta dimensión  en la cual nos movemos y en la cual somos participes. En cierta medida, el tiempo nos condiciona, por ello hay que saber manejarlo para no convertirnos en «esclavos» de él.  

En algunas personas se convierte en algo bueno, porque conocen su propio tiempo y lo valoran. Por otra parte, no es conveniente adaptarse siempre  al tiempo del otro, de ahí que pueda resultarnos agotador. La tranquilidad y la paciencia -a mi entender -son dos factores importantes para convivir en armonía con el tiempo.

Haciendo hipótesis, podríamos pensar que el tiempo se ha detenido como consecuencia del ritmo vertiginoso de la sociedad, creando entonces, controversias entre las personas que la habitan

Si tendría que describir su forma de actuar, diría que se caracteriza por su paso fugaz, siempre  en el presente. Ésta es la razón por la cual en un mismo momento no se puede ni ir para atrás ni para adelante, porque algo ya pasó o está por venir. 

En síntesis, hay que disfrutar el «aquí y ahora», que es lo que pasa. Es decir, el «momento a momento» como si fuera el último y disfrutarlo, que es muy importante.

Puede ser que actué en complicidad con Dios y a su vez con la sociedad en general. De manera que yo recomiendo hacerse amigo de él, en definitiva, compartimos juntos nuestros más lindos momentos; esos donde conocimos un montón de personas hermosas o vivenciamos cosas con ellos. Se trata de saber que el tiempo está y permanece con nosotros en nuestros recuerdos, como ese amigo fiel que conquistamos en nuestras vidas

\section*{Tiempo}
\addcontentsline{toc}{section}{Tiempo (Por Nelly)}
                                                                                                            \begin{flushright}Por Nelly\end{flushright}
\begin{center}
Señor tiempo has pasado a mi lado.

Siempre con tu eterno apuro,

continuabas tu rumbo.

\vspace{0.5cm} 

He querido aferrarte,

mas con sabia costumbre

dibujaste surcos en mi vida

y te alejaste como si nada.

\vspace{0.5cm}

Eres por demás prófugo, 

no tienes alma,

te acuso por dejarme abandonada.

\vspace{0.5cm}

Sabías de mis lágrimas

y no te detenías,

de  mis temores incluso.

\vspace{0.5cm}

Momentos que ansiaba que te vayas,

mas lo hacías cuando a ti te parecía.
 
Cuánto tiempo has pasado al lado

de esa niñita frágil,

que sin palabras te pedía:

«quédate, juguemos un ratito más»

\vspace{0.5cm}

Pero, tu prisa no te lo permitía.
 
La candorosa adolescente te quería

y no te importaba.

\vspace{0.5cm}

La espléndida mujer 

rodeada de amores,

te rogaba «¡no te vayas!»

y le dabas la espalda.

\vspace{0.5cm}

Hoy esa niña, esa adolescente,

esa mujer de ayer,

no te pide, no te ruega.

\vspace{0.5cm}

Te ordena, te detiene, 

te interroga: «¿purgué esa prisa?  

¿Por qué corres?

¿Por qué no atiendes mi demanda?»

\vspace{0.5cm}

En este tiempo que, aún río, 

tengo ilusiones.

Tiempo en que me abrazo a mi amado,

mis labios tienen un destino,

mi mirada se hunde en unos ojos queridos.

\vspace{0.5cm}

Éste es el momento de que quedes preso,

para que no te vayas.

Quédate un tiempo más.

Quédate Señor Tiempo.
\end{center}

\clearpage
\section*{Los captores del tiempo}
\addcontentsline{toc}{section}{Los captores del tiempo (Por Juanjo)}
                                                                               \begin{flushright}Por Juanjo\end{flushright}

---El trabajo está hecho.

Don Mescola asintió casi imperceptiblemente con la cabeza y se retiró de la oficina. No sabía si la noticia de su lacayo llegaba a tiempo. Con su chalina y sombrero calzados, bajó por las escaleras y salió a la calle. Tenía el cuerpo de un oso, pero había aprendido a moverse con agilidad en la vida. En la puerta del edificio lo esperaba su chofer frente al coche. Se dirigieron a una clínica privada en el centro de la ciudad.

Mescola caminó hasta la habitación 101. No necesitaba que ninguna enfermera lo guiase porque se había pasado yendo y viniendo de allí, toda la semana. Dentro estaba su esposa casi dormida en una silla. Cuando lo escuchó entrar, reaccionó asustada y miró enseguida hacia la cama. Allí, en el centro de la habitación, entubada y conectada a una máquina, estaba Clara, su hija. Parecía la versión moderna de un Cristo en la cruz. Los pitidos constantes de la máquina indicaban que aún vivía. Clara trabajaba en la pastoral carcelaria visitando reclusos. Siete días atrás se había visto envuelta en el medio de un motín. Hubo disparos y dos balas la alcanzaron en órganos vitales.

---¿Lo encontraron? ---preguntó la mujer, ansiosa---. Todo parecía indicar que el plan había salido según lo dictado.

\vspace{0.5cm}

Mientras tanto, a 46 Kms. de la ciudad, en las afueras de un poblado, donde se levanta el establecimiento penitenciario Santa María, la banda de Lechuga espera la puesta en libertad de su líder. El calor se hacía sentir y los cuerpos transpirados de los hombres brillaban contra el opaco paisaje. Lechuga estuvo guardado cuatro años por un error de cálculo en cierto trabajo. El error de cálculo consistió en que la policía llegó antes de tiempo. A las 12 en punto del mediodía el reo volvería a caminar entre los hombres libres. No sucedió así. 11:59, hora local, las manecillas de todos los relojes quedaron congeladas. El sol se quedó donde estaba y las nubes detuvieron su viaje. Los guardia cárceles se negaron a soltar a Lechuga y, luego de permitirle hablar por teléfono, lo devolvieron a su celda.

---¿Me podés decir qué pasó, Cartucho?

---No tengo idea, jefe. Está todo muy extraño acá afuera. Es como si hubiesen detenido al Tiempo.

---Eso no puede ser. El Tiempo es el tipo más astuto que hay. El Tiempo es escurridizo. Cuando lo estás por atrapar, el Tiempo se te escapa entre las manos. Se dice que el tiempo es un gran maestro; lo malo es que va matando a sus discípulos.

---Vos quedáte tranquilo, Lechuga. Nosotros nos vamos a averiguar qué está pasando.

---Escucháme bien, Cartucho. Necesito salir hoy sí o sí.

\vspace{0.5cm}

La noticia no tardó en llegarles a los oídos. En la ciudad el clima estaba enrarecido por la falta del Tiempo. Aparentemente los hombres de Don Mescola le habían hecho una cama y el Tiempo había caído como el mejor. No se sabía cómo lo habían logrado, pero se aseguraba que habían recibido ayuda externa. La cuestión era que el Tiempo, o Sr. Tiempo, como lo llamaban quienes tenían el gusto, estaba detenido un sector especial de la comisaría primera de la ciudad. Según comentaban, en una fosa de tres metros de profundidad y con oficiales haciendo guardia a su alrededor. La orden era tenerlo allí por tiempo indefinido.

\vspace{0.5cm}

La banda de Lechuga se reunió en un bar a planear el atraco y posterior liberación del Tiempo. La moza que los atendió estaba bastante malhumorada. Hacía tres horas que se tendría que haber ido a su casa, pero como el Tiempo estaba detenido, su jefe no la dejó marcharse hasta completar el turno.

---¿Qué pudiste averiguar, Lechuza? ---habló Cartucho, que parecía haber tomado el rol de jefe, en ausencia de Lechuga.

---El rumor parece cierto. Aparentemente la orden vino de arriba. Alguien importante quiere ver al Tiempo detenido por un buen rato y se encargó de pagar muy bien su capricho. No va a haber juicio, ni nada. Derechito adentro.

---¿Algún dato de interés sobre los oficiales que lo custodian?

---El cabo Rodríguez se casa mañana. ¿Sirve?

---Sirve. Sabueso, vamos a necesitar algunas cosas para el operativo. Anotá. Una camioneta, cotillón, dos bailarinas hawaianas, una comparsa (con bombos y platillos), un barman con tragos para veinte personas,  sandwichito de miga (de jamón y queso) y un mono. ¿En cuánto podés tenerlo listo?

---Dos horas.

---Ok. Como los relojes no funcionan vamos a hacer lo siguiente. El Abuelo tarda diez  minutos en dar una vuelta a la manzana. Se va a poner a caminar y a contar las vueltas diciéndonos el número por radio. Cuando llegue al número doce, nos encontramos acá.

\vspace{0.5cm}

Cuando El Abuelo completó las vueltas y cantó el doce, todos los miembros de la banda de Lechuga se encontraron en el bar, se subieron a la camioneta y enfilaron para la comisaría.
Cuando llegaron, Cartucho fue el encargado de anunciar la visita. Tocó el timbre y una mirilla en la puerta de metal se corrió para dejar ver el par de ojos de uno de los oficiales. Cartucho, que ya se había puesto una guirnalda como collar y tenía un trago largo en una mano que equilibraba con el sandwichito de la otra, anunció que estaba todo listo para la despedida de soltero del cabo Rodríguez. El aluvión de personas tomó por sorpresa al ocasional portero y todos entraron velozmente a la comisaría: Cartucho, Lechuza, El Abuelo, Sabueso, el del bombo, el de los platillos, las bailarinas, el mono y algunas personas que pasaban por ahí y les apeteció formar parte del festejo.

\vspace{0.5cm}

Entonces aconteció lo más fantástico de esta historia. Mientras todos se estaban divirtiendo, comiendo, tomando y bailando, en la profundidad de la fosa algo mágico pasó. El Sr. Tiempo sintió un fuerte dolor en la espalda y le crecieron un par de alas de blanco plumaje. Solo con aletear un par de veces logró elevarse por sobre la boca de lo que era su prisión y ver desde arriba a sus captores. Sucede que el tiempo vuela cuando uno se divierte.

Al verlo aletear e irse, los miembros de la banda de Lechuga brindaron por el éxito de su operativo y se fueron raudos en la camioneta a encontrarse con su jefe. El cabo Rodríguez tuvo que aceptar la realidad de que en menos de veinticuatro horas se casaría pero de todas formas seguía abrazado a una bailarina hawaiana. En una clínica privada, un jefe de la mafia lloraba amargamente mientras pedía la cabeza del jefe de policías. Comprendió que lo que el Tiempo quita, el dinero no lo puede comprar.

\vspace{0.5cm}

Cuando Lechuga estuvo afuera, pidió las llaves de la camioneta y salió disparado. No veía la hora de reencontrarse con Clara Mescola, su enamorada, pero esta vez, afuera.

\section*{Detención del tiempo}
\addcontentsline{toc}{section}{Detención del tiempo (Por Gustavo)}

\begin{flushright}Por Gustavo\end{flushright}

Se escucha un silencio y Sr. Juez ingresa a la sala. Una vez en la misma, el fiscal acusa de falso testimonio al Sr. Tiempo. Parece ser que luego de un accidente automovilístico, un testigo asevera haber visto al inculpado al volante del auto que se dio a la fuga el día del hecho, hace un mes, a las 19:56 hs. Pero, el acusado sostiene que en ese instante se encontraba en el estadio de fútbol de su equipo del alma, viendo a su amiga Ansiedad torturar a cuanto rival se oponía en su camino.

Un segundo testigo asegura haber visto hace un mes a la hora en cuestión, al Sr. Tiempo en el cine América, en la sala número dos. Frente a esto, el imputado asevera que es verídico, que estaba viendo junto a su prima Historia su film favorito: Viaje al futuro»

El juez un poco desconcertado pregunta al acusado: 

---Su supuesta presencia al volante, ¿cómo se explica?

---Es sencillo Sr. La ansiedad del fiscal por construir una historia creíble le generó una confusión habitual: yo no utilizo medios de transporte, soy un medio de transporte. Acaso... ¿usted no me utiliza?

\vspace{0.5cm}

\emph{Qué es el tiempo, nos seguimos preguntando. Tal vez, este escrito pueda explicar el Tiempo, según el paso de los años:}

\clearpage
\section*{Tiempos}
\addcontentsline{toc}{section}{Tiempos (Por Constanza Luna)}

                                                                                              \begin{flushright}Por Constanza Luna\end{flushright}
\begin{center}
Tiempos de infancia,

Siestas en el campo,

Rodillas gastadas

Y juntar mariposas.

Cuentos con la abuela,

Comer mandarinas

Y saltar la rayuela.

Tiempo de adolescencia,

Lágrimas sin motivo,

Poca paciencia,

Amores imposibles

Y bocas pintadas.

Tiempo de reveldía

Tiempo de adultos,

Compromiso, poco descanso,

Mucho trabajo, sueños parados.

Tiempo de abuelos,

Volver a juntar mariposas,

Contar cuentos

Y saltar una que otra rayuela.

Yo me pregunto:

¿El tiempo nos corre?

¿O Nosotros corremos al tiempo? 
\end{center}

%Marilina
\vspace{0.5cm}
\emph{Años, historias del tiempo que encierran cada vida. Cada uno sabe ponerle el color preferido, la nota musical que está más acorde con su realidad:}

\section*{La música en el tiempo}
\addcontentsline{toc}{section}{La música en el tiempo (Por Dolores)}

\begin{flushright}Por Dolores\end{flushright}

Cuando apareció el hombre, el mundo se pobló de sonidos. La primera manifestación pensada fue la voz humana, la risa y el llanto.

Luego, el canto de los pájaros hizo su maravilloso aporte, con su trinar melodioso.

Más tarde, la rudimentaria confección de instrumentos de viento y percusión.

Pasando el tiempo, el hombre se las ingenió para crear instrumentos de cuerda y perfeccionar los ya existentes.

Ahora, ya están los elementos, los instrumentos y la música que era pasada, de oído. Faltan los signos que nacerán de un verso de Guido de Arezzo.

Ya tenemos las notas, como la literatura la palabra y aparecen los genios Pergolesi, Bach, Mozart, Beethoven, Chopin, Albéniz, Falla, Chaikovsky y muchos otros, que engalanan ese arte.

Irrumpe con su grandeza e inspiración «La Sinfónica», que convoca a muchos músicos; nacen las orquestas que deleitan a la gente de todas las épocas.

La ópera tiene un lugar muy importante, ya que conjuga canto, drama, ballet, estableciendo huellas indelebles con constantes que movilizaron multitudes.

Pero, la Era dorada fue la del vals, con la familia Strauss; Erato deposita la creatividad absoluta. Son acompañados a donde van y hasta se cambia el aspecto edilicio de la ciudad. Es que Viena es la cuna del vals, donde se mecen al compás de melodías inolvidables.

Vuelan las polleras de las damas en sincronizado movimiento, cuando el caballero la lleva por la cintura.

Horas y horas, los salones son ocupados por parejas que danzan con paso valseado, dejando en el tiempo un testimonio gratísimo que no morirá jamás.

De todas las artes, la música es la más noble. No hay necesidad de comprenderla, solamente se debe escuchar con atención y crecerá, entonces, la espiritualidad.

\chapter*{Fiesta de disfraces}
\addcontentsline{toc}{chapter}{Fiesta de disfraces}

%Marilina
\vspace{0.5cm}
\emph{Desnuda la palabra, anestesiado el enojo. Quizás sellando un pacto de reconciliación, ahora se avecina un encuentro. Un festejo propuesto para acercarnos al Tiempo. En esta oportunidad, disfrazamos con el lenguaje nuestro cuerpo y nuestro ser:}

\section*{Camino a la fiesta}
\addcontentsline{toc}{section}{Camino a la fiesta (Por Gustavo)}

\begin{flushright}Por Gustavo\end{flushright}

---Holaaaaaaaa!... (bostezo) ¿A ver qué es esto?  «Buen día, estás invitado a la fiesta de disfraz a realizarse en el predio. «Es hora de ponernos la careta». La música estará a cargo de Dj. Tiempo. Se adjunta tu disfraz. No faltes!!!  Saludos de todo el staff de The Cosmos Fest Corporation»

Mmm...interesante, si está este Dj. no puedo faltar. Pero, qué hay en esta caja: ¡¡¡un disfraz de camino!!! 

Fue éste el punto de inicio de una gran fiesta en la que los complejos de la gente, no tuvieron cabida, ¿lo vemos juntos?

Era el gran día, todos los detalles trataron de ser tenidos en cuenta y sólo era cuestión de tiempo, los invitados comenzaron a llegar. Luz fue la primera en brillar, estaba desde los comienzos del día, su impaciencia era tal que hasta ayudó a ultimar todos los pormenores.

Al llegar Camino fija su mirada en Luz, la Reina de la fiesta, intenta acercarse pero no sabía cómo y ve a lo lejos un personaje que pasaba desapercibido a simple vista y al llegar a él le dice: 

---Hola, ¿cómo estás, te puedo hacer una pregunta?

---Sí, claro ---responde el intrigante sujeto. 

---Perdón por mi ignorancia, pero... ¿de qué estás disfrazado?

---De Túnel, pero no te preocupes es normal en la gente que no me registren. Yo me la paso soportando el peso de un río, de una ciudad, de toda una estructura que me tiene la columna a la miseria, ¿para qué? ¿Alguien me puede decir para qué? (grito).

---Tranquilo amigo ---sugiere un poco desconcertado Camino.

---No, qué tranquilo, todo para que venga el figureti de Puente y se ponga a bailar en la tarima, siempre es así, llamando la atención de todos: la fotito del nene al lado del puente, la imagen que representa a la ciudad...el puente, se cae y todos arriba de él a levantarlo. Y de mi quién se acuerda, total esa cosa tapada que la pisan día y noche nadie la escucha. Pero eso sí, bien calculado que tienen los años que me quedan, ja! Esperen cuando no esté más, ya van a ver, ya van a ver...

Camino sigue su recorrido por el predio se cruza con distintos personajes entre ellos uno bastante chistoso:

---Hola soy Camino, ¿y vos?

---Encantado soy el Eco, soooy el eco...

---Y Eco, qué te parece la fiesta.

---Y este Dj. tiempo medio que repite los temas, los teeemas... Che Camino te dejo, quedé en encontrarme con mi amigo Infinito, infiniiito...

A cada minuto el encuentro se ponía mejor. Pero la llegada de Oscuridad opacó a Luz, la rivalidad entre ambas hizo que poco a poco, esta última se vaya apagando, destello a destello. 

Igualmente intentaron intercambiar algunas palabras que, sólo quedaron en simples balbuceos. El episodio fue de tal magnitud que de pronto se hace un silencio absoluto, sólo se escuchaba a Eco, cuando con gran velocidad la voz del Dj. irrumpe diciendo: ¡¡¡lluvia de meteoritos!!! Era una nube de alegría que embellecía la hermosa noche.

Luego del suceso precedente, nuestro personaje fija su mirada en un grupo de planetas, se acerca y luego de presentarse ante Júpiter, queda deslumbrado por la hermosa Venus, intercambian algunas palabras frente a la mirada furiosa de Marte, que comenzaba a emitir chispazos. Entra en la conversación Saturno, quien se acerca a Camino y le dice al oído: «tené cuidado, primero te enamorás y después te someten al compromiso, si no miráme a mí, no tengo escapatoria».

Acto seguido, el protagonista se encuentra con Ruta, una ex novia, se ponen al día sobre sus vidas y la charla se hace extensa. En eso aparece Agujero Negro, un gran amigo que lo introduce nuevamente en la fiesta, juntos se desplazan a otra galaxia en busca de nuevos horizontes.

Las melodías iban lentamente anticipando el final del espectáculo y la voz del Dj. expresa lo siguiente:

---Siempre acompañando sus vidas me hacen pasar muy bellos momentos y esta vez, el anfitrión quería ser yo, no espero que este evento quede en sus recuerdos ni que se mortifiquen esperando a que se repita, si no que simplemente de la misma manera en que hoy musicalicé sus corazones, ustedes musicalicen día tras día a todo el universo.
\clearpage
\section*{El disfraz}
\addcontentsline{toc}{section}{El disfraz (Por Belkis)}

\begin{flushright}Por Belkis\end{flushright}

Sé que el Sr. Tiempo me conoce muy bien y, por ese motivo, estoy segurísima de que va a traer un disfraz, que sabe que me lo iré a poner gustosa. Y será de mariposa. Pero no una mariposa común, sino que sus alas serán muy largas, fuertes y rapidísimas, de varios colores. Así, podré volar muy alto e iré de un lugar a otro en cuestiones de segundos. 

Le pediré al Sr. Tiempo que retroceda muchos años y la fiesta sea en Navidad. Entonces, ahí, estarán todos mis amigos disfrazados de flores,  árboles, pastores, ovejas, Reyes Magos, pájaros, etc.

Me acercaré al árbol que esté más iluminado, el que sea más fuerte, con globos de cristal y luces brillantes y que, por supuesto, tenga una hermosa estrella de plata. Y yo andaré volando por todos lados, bajaré la mirada ya allí divisaré a la sagrada familia, con su niñito Dios recién nacido. Él junto a su mamá, la Virgen María dándole todo su amor y su papá terrenal, San José, adorándole. Mis alas bajarán despacito y sin despertarlo, depositaré un gran beso en su carita suave como terciopelo. Entonces, sí le diré al Sr. Tiempo: 

«Ya está, cumplí mi cometido. Ahora, llévame de vuelta a mi hogar».

\section*{Tiempo de máscaras}
\addcontentsline{toc}{section}{Tiempo de máscaras (Por Ana)}

                                                                                      \begin{flushright}Por Ana\end{flushright}

Diáfana, pletórica de sol, puedo tocar su transparencia, beberme el aire fresco cargado de aromas. Así de bella esta mañana que disfruto en el patio.

La ropa que voy tendiendo se agita feliz en los brazos del viento, que juega a inflar globos en las mangas... En ese momento, el Señor Tiempo me hace saber que: da una gran fiesta de disfraces y estoy invitada. Le contesto que nunca fui a una de esas fiestas y tampoco tengo disfraz. Me responde: ¡Sólo díme de qué quieres disfrazarte y será tuyo el más bello!

Sentada al borde de mi cama medito en voz alta: «esta esencia original e irreductible que hace 74 años me acompaña, no admite disfraces ni dobleces; y lo que me hubiera gustado ser ya no fue, y no puedo volver atrás.

Disfraces y máscaras a diario caminan por la vida. Seguro la mascarada del Sr. Tiempo será un éxito. Ojalá no le falte tiempo.

\section*{Un disfraz para mí}
\addcontentsline{toc}{section}{Un disfraz para mí (Por Willy)}

                                                                                   \begin{flushright}Por Willy\end{flushright}

El Sr. Tiempo, ese personaje que sigue su marcha a través de la eternidad, ha sido pensado por los hombres como un ser siniestro en algunos casos o como un ser desconsiderado en otros. 

Más allá de lo que piensen los demás yo sé que él me conoce muy bien. Sabe lo que pienso y lo que no pienso. Podría decir de él que es la mano invisible de Dios, que pone sus límites a la soberbia del hombre: el hombre siempre quiso parecerse a Dios, pero el tiempo frena su soberbia, cortándole su vida tarde o temprano. Por eso respeto al tiempo...cualquiera que no lo piense así, se irá de esta vida igual que todos...sin vida.

El tiempo se pondría delante de mí, me observaría y diría: 

«Sé como piensas y en qué piensas.Tu vida no ha sido ni fácil ni difícil, simplemente es vida. Aprendiste a no quejarte y a ver las cosas de otra manera. Los demás siempre le echan la culpa a otro o a lo que los rodea, debido a esto consideran que su vida es desgraciada.

Pero en tu caso ves las cosas de otra manera, son los mismos ojos que tienen todos, pero es distinta la mirada. A pesar de eso, igual tienes muchas personalidades en una, como toda la humanidad. ¿Pero cuál es la que te define? ¿Será la de aquél que trata de fingir muchas veces que no sufre, que siempre parece estar bien, optimista, pero, al igual que el león, se lame sus heridas en silencio y lejos de la vista de todos masticando su sufrimiento? ¿O de aquél que algunas veces se enoja y se enceguece y es capaz de pelear a muerte por lo que cree que es justo? ¿O será la de aquél que muchas veces finge ser fuerte, pero en el fondo lamenta algún amor no correspondido? Si tuviera que elegir un disfraz que te defina sería el de un pensador que se pregunta por todo y pregunta a todos por qué viven de esa manera. Te disfrazaría de Sócrates, aunque no seas parecido en su pensar ni tampoco quieras parecerte. Sócrates cuestionaba la sociedad que le tocaba vivir. Y por eso hablaba con todos preguntándoles si estaban de acuerdo con su forma de vida y si sabían definir muchas cosas presentes en su vida: qué es la justicia, la virtud, la felicidad, etc.

En tu caso preguntarías qué piensan del poder y si eso les trae felicidad. El poder, esa palabra que tanto te molesta, porque sabes que es un dulce veneno que embriaga al principio, pero mata después a todo el que lo pruebe.

Preguntarías a todos por lo que piensan sobre ética, moral, solidaridad, comprensión, autoestima, empatía... ¡en fin, tantas palabras que faltan en el diccionario de muchos! 

Palabras tan actuales por su carencia, por su ausencia en nuestro tiempo actual. Y el poder rodeando todo como un manto oscuro, envolviendo todas las demás palabras en su red de secretos que no pueden revelarse a los ojos de la mayoría. Pero esa mayoría no sabe que el poder no se tiene sino que se ejerce. Y eso lo sabes muy bien. Por eso tu pensamiento sabe que el poder no es de aquel que se cree superior sino de aquel que lucha por alguna idea o causa superior. Tu disfraz de Sócrates es el que mejor te queda. Aprovecharías para preguntarle a todos qué piensan y porqué piensan lo que piensan, sabes que con eso ellos se ayudarían a replantear  su modo de vida. Eso te haría tan útil como feliz.  Eso yo no lo dudo, te conozco a través de mí, a través de  siempre. Tu amigo, el Tiempo»

\section*{La fiesta de disfraces}
\addcontentsline{toc}{section}{La fiesta de disfraces (Por Juanjo)}
                                                                                    \begin{flushright}Por Juanjo\end{flushright}

La invitación de T llegó en papel acartonado color marfil y caligrafía perfecta. Con tinta negra me invitaba a su fiesta de disfraces. El lugar del evento sería la mansión del Maestro López Claro, el 9 de septiembre a las 21:00 hs. y los demás miembros del grupo recibirían una invitación similar.

La única condición que T imponía a sus visitantes era que él elegiría el traje con el que nos tendríamos que presentar.

Cuando llegó a casa, treinta minutos antes de la fiesta, me dijo que ya había estado con los demás invitados y que, por ser yo un invitado especial, podría elegir mi disfraz de entre los que tenía para ofrecerme. Con ademán señorial hizo un movimiento de su capa y ante mi aparecieron tres túnicas.

La primera se llama Ádalo, me dijo. Con esta túnica serás invisible y podrás divertirte a tus anchas en la fiesta. Póntela antes de tocar el timbre y, cuando la puerta se abra, entra rápido, antes de que el mayordomo piense que se trata de una broma de los niños del barrio y la vuelva a cerrar.

La segunda se llama Eufén, ofreció. Si vistes este disfraz, todos los que te vean usándolo verán exactamente lo que esperan que tú seas. Las profesoras te verán como un alumno ejemplar, los compañeros como un par solidario y esa nueva alumna te verá como... bueno, ya sabés cómo querés que te vea.

La última opción se llama Kafén. Si vistes con este traje, serás, a los ojos de los demás, una persona horrible y despreciable. Cada uno de los que te vea, verá en tí a su peor pesadilla. Es un traje arriesgado pero, al final de cuentas, se trata de una fiesta de disfraces y todo está permitido. Todo depende de tus intenciones. Más de uno, en el pasado, se ha divertido en grande usándolo.

Apabullado por la cantidad de cosas que T me decía, di un paso hacia atrás, como buscando aire, y me dejé caer en un mullido futón verde. No alcanzaba a ponderar sus tres ofertas cuando una idea me cruzó como un rayo la mente: yo quería disfrazarme de Batman. Evidentemente no era una opción disponible. Si mi superhéroe preferido hubiese sido El hombre invisible, habría tenido mejor suerte.

Un pensamiento llevó al otro y, todavía sin responder, me puse a pensar en los demás participantes de este curioso festejo. ¿Habría ido T a todos con la misma historia? ¿En verdad ya todos tendrían su traje o estaba T en ese mismo momento también con todos ellos haciéndoles la misma propuesta? ¿Qué pasaría si todos elegimos la túnica de invisibilidad? Llegaría a la fiesta y pensaría que no hay nadie más que yo, que todo fue una burla y que los demás estaban en sus casas mirando el partido de Colón. Tal vez todos estarían dentro del salón, sin verse, moviéndose con los brazos estirados, tratando de encontrar a otros con el tacto. Eso sería bastante divertido, pensé.

Si todos eligiéramos Eufén, el traje de la complacencia, la fiesta sería seguramente un aburrimiento sideral. Descarté ese traje luego de este pensamiento.

Finalmente arribé a la idea que me había estado rondando la cabeza desde que escuché sobre Kafén. Comiéndome las sienes, masticando mis neuronas, anudando las delgadas fibras que forman mi cerebro. Si todos eligiéramos a Kafén, la fiesta de disfraces no sería una fiesta de disfraces cualquiera. Sería una noche de brujas, sería Halloween, ese 9 de septiembre sería 30 de octubre.

Estaba perdido en mis cavilaciones y laberintos de la mente cuando el constante y seco sonido del golpe del taco de T en mi parqué recién lustrado me trajo de regreso a la realidad. Me miraba desde el otro extremo del salón, con el brazo extendido sosteniendo mis tres opciones y el rictus en su cara parecía decirme «apuráte que me acalambro». Por supuesto, un ser superior, como lo era T, no se expresaría de esa forma. De todos modos, como no quería incomodarlo con mi indecisión, le propuse un trato. Si me prestaba los tres trajes, se los tendría limpios y planchados para el lunes a la mañana. Sonrió con una mueca y me retrucó la propuesta. Tenía que lavar y planchar el de todos los invitados. Estrechándole fuertemente la mano, acepté.

Llegué al 73 de la calle Piedras, pasadas las 21:15 hs. Todavía no había decidido que disfraz ponerme cuando toqué el timbre. El sonido retumbó como si un monje chino hiciera sonar 100 gongs a la vez. Me resultó extraño, las otras veces que había tocado ese timbre había sonado «la cucaracha». Me encogí de hombros, T era de hacer esas cosas. El mayordomo llegó para abrirme y me miró algo contrariado a ver que, aparentemente, no había cumplido con la regla de ir disfrazado. Caminé la senda que, entre coníferas, lleva al salón principal de la mansión y empujé las pesadas puertas de madera para abrirme paso. Antes de entrar, me eché las tres túnicas sobre el cuerpo, dejando la capa de invisibilidad por sobre las demás. 

Mis ojos no daban crédito a lo que veía. Allí, en el medio del salón, me vi a mi mismo disfrazado de Batman. Me acerqué lentamente a escuchar que decía. Estaba hablando con la alumna nueva; ella estaba disfrazada de caperucita roja.

Le hablaba al oído por lo que no podía escucharme. Traté de acercarme lo más que pude, sin que notaran mi presencia. Había muchas personas a nuestro alrededor y la tarea se me dificultaba. Estaba ejecutando ese prodigio, estirando el cuello lo más que podía para acercar mi oído cuando me pisé la túnica y ésta, sin pedir permiso, se fué directo al suelo. Quedé parado ante mi mismo y caperucita luciendo a Eufén.

La cara de Batman se transformó. Yo no podía saber qué estaba viendo, porque el efecto de esta capa es subjetivo a quién observa. Batman estiró los brazos hacia mi e intentó besarme. La situación se me hacía terrorífica; no solo un hombre intentaba besarme, sino que ese hombre tenía mi cara. Arrinconado en una de las esquinas del salón y viendo pocas alternativas de escape, decidido a rifar cara mi hombría, me saqué de encima a Eufén y quedé luciendo Kafén. Cuando Batman me vio, dio un grito curiosamente agudo; era el grito de una mujer. El hecho me ayudó a comprender un poco lo que sucedía. Yo no era Batman; alguien estaba vistiendo algún mágico disfraz ofrecido por T con vaya uno a saber qué hechicería y desconocidas consecuencias para quien lo mirase. Estiré la mano y le arranqué su traje. La vestimenta de Batman ya no era tal en mi mano, la mágica tela volvió a su forma original y a la luz de los hechos apareció una compañera del taller. Me lanzó una mirada venenosa; evidentemente estaba enojada porque la había dejado sin disfraz.

Luciendo Kafén, la capa del terror, me moví por todos los puntos del salón asustando a los invitados. Ante cada una de mis apariciones, el desprevenido de turno se ponía a temblar, se caía desmayado o se infartaba. Me pasé una hora entera divirtiéndome a costa de los demás; T tenía razón. En un momento, cansado de tanto correr y saltar, me senté a tomar un vaso de gaseosa para recuperar energías. En ese momento me asaltó la duda. ¿Qué me mostraría Kafén si con esta túnica puesta me dirigía a un espejo y me miraba en él? ¿Con qué imagen me aterraría el prodigio? ¿Un temor del pasado? ¿La fatalidad de un futuro? ¿La imagen de un enemigo? ¿Un rostro desconocido? Con mucha curiosidad caminé ansioso hacia el espejo que me esperaba colgado en la pared del fondo. Cuando llegué ante él me miré y el reflejo que me devolvió era un rostro conocido. No me costó esfuerzo reconocerlo. Lo había visto esa mañana mientras me afeitaba.

\chapter*{César López Claro}
\addcontentsline{toc}{chapter}{César López Claro}
%Marilina
\vspace{0.5cm}
\emph{Abrir el mundo parece una utopía realizada, en tiempos en los que las distancias se acortan gracias a la intervención virtual.
Sin embargo, hay mundos próximos que ignoramos. Tesoros valiosos que adormecemos con nuestra indiferencia.
El Taller tiene su espacio y éste convive con otros lugares, no menos importantes.
«El brillo de la palabra» cohabita con un museo, el museo de Guadalupe, el de él, el de Don César López Claro. 
Fuimos hasta el lugar, bebimos de su encanto y re-significamos su obra, traducida por nosotros en simples, pero sentidas palabras:}

\section*{La niña}
\addcontentsline{toc}{section}{La niña (Por Constanza Luna)}
                                                                                                     \begin{flushright}Por Constanza Luna\end{flushright}

Estoy en mi atelier, cierro los ojos y visualizo en mi memoria, el río, sus orillas, la gente del lugar, y recuerdo la niña de rulos y grandes ojos negros.

Comienzo a pintar. Pareciera que la tengo frente a mí. Sentada en una silla pequeña hecha de paja; su vestido descolorido, pies descalzos y una sonrisa triste. Al fondo, la luna reflejada en el río y un perro tomando agua. Bueno, el paisaje casi está. Pero quiero volver sobre la niña.

El olor a café de la cocina me desconcentra. Viene María con su paso apresurado y deja una taza humeante en la mesa. Tomo un poco y vuelvo al cuadro. Quiero dibujar sus ojos negros. Me acuerdo que al verla quedé mudo delante de tanta belleza. Su mirada me decía cosas que, en el momento, no llegué a comprender. 

Creo que mientras viva no voy a olvidar esos hermosos ojos negros.
Hoy, muchos de esos cuadros duermen entre sombras y polvo. Sin embargo, cuando uno lo recorre siente como si un fantasma nos diera la mano y nos ayudara a conocer toda su obra.

\section*{Bermellón}
\addcontentsline{toc}{section}{Bermellón (Por Juanjo)}
                                                                                                        \begin{flushright}Por Juanjo\end{flushright}

Entorné los ojos para enfocar y entender lo que estaba viendo. Dos puntos luminosos, uno arriba del otro. Luego el campo de visión se amplió y aparecieron unos números en el panorama. El reloj digital indicaba las dos y diez. Al costado, sobre la misma repisa, mis herramientas. Pinceles, lápices y la cuchilla con la que saco punta a esos lápices. Me desvestí de las sábanas usando las piernas y con un movimiento que a mi edad podría calificarse de ágil, dos segundos después, tenía los pies enfundados en las pantuflas de paño. Arrastré las suelas de goma por el atelier, tomé un pincel con la mano derecha y continué donde había dejado al caer rendido ante el ataque sorpresivo del sueño.

Mientras trabajo no puedo dejar de pensar en María. Ella duerme en la habitación, silenciosa. El camastro en el taller me permite trabajar durante la noche, tomando pequeñas siestas de media hora sin molestar a mi esposa. Cuando los primeros rayos de luz entran por la ventana, dejo todo y voy a dormir a su lado. Cuando me despierto al mediodía, ella ya está terminando alguna clase. Almorzamos juntos y vuelvo a trabajar.

Estoy convirtiendo una de las paredes del taller en un nuevo mural. Me gustan los murales. Huelen a inmensidad, a sin frontera. Para completar un mural uno tiene que dedicarle semanas, en oposición a un cuadro chico, tal vez una naturaleza muerta, que se puede completar en a lo sumo dos días. Gracias a esta cantidad de tiempo requerida por la obra es que se logra desarrollar una onda sensación de pertenencia. En ambos sentidos. En el más clásico, la obra te pertenece, puesto que la creaste. Pero en uno más metafísico, es la obra la que te empieza a poseer. Te  pide más, dicta su desarrollo, expande sus límites. 

El mural en el que estoy trabajando ahora se llama Revuelta o tal vez termine llamándose distinto. Muchas personas se han juntado en una plaza a manifestarse. Llevan carteles y pancartas. Insignias y lemas. Rostros y banderas. Yo mismo me veo en la revuelta. Soy uno más y a la vez soy todos. Pinto horas enteras sin descansar. El olor a pintura fresca me llena y me vacía. Inflo mis pulmones y soy irrigado. A mi alrededor, el taller. Trapos sucios, latas, botellas. Olor a aguarrás y resinas. Pinceles y paños. Luces y sombras. Colores y engaños. La fuerza creadora me eleva. 

Y para materializar la metáfora me subo a un andamio y pinto la parte superior del mural. Puntas de lanzas que rasguñan el cielo. Gritos y plegarias que ascienden. Y ahí, desde arriba, escucho al gato de la vecina en la ventana. Maúlla y araña el vidrio como queriendo entrar. «No, ahora no puedo. No molestes, estoy trabajando». Pinto, delineo, coloreo. Ropa sucia, pinturas y otras obras. Todo, escenario de la actual concreción. 

El gato sigue maullando y me interrumpe. «No ahora, no. No te puedo dar de comer». Sigo pintando. Amarillos, bermellón. Negros, grises y marrones. Hay fuego. Multitudes. El pueblo grita, se exalta, canta. Y yo soy su voz. Tengo que pintar para que puedan gritar, exaltarse, cantar. Si no pinto no existen. El gato sigue molestando, ahora con más insistencia. Mezclo lo que queda de naranja con bermellón sobre la tapa de una lata. Cargo el pincel y sigo. No puedo detenerme. Ahora son estallidos. Columnas de fuego y humo circundan la escena. La parte derecha del mural explota en una batalla campal entre el orden y los que se manifiestan. 

Yo soy su arsenal, el que le carga las armas, el que fabrica sus balas. Sin mí no tienen con qué disparar y la batalla está perdida. Siguen los estallidos y las explosiones. Amarillo, naranja, bermellón. Sigo pintando. Y el gato de la vecina golpea el cristal con sus uñas. Y aprieto el pomo de bermellón y ya no queda. Lo exprimo, lo estrujo, lo estrangulo. No salen más que las últimas gotas. Pero el mural no está terminado. Me pide más, me interpela, me exige. El pueblo me grita, me necesita. Están perdiendo la batalla. El fuego también me reclama. Y el gato vuelve a maullar. Y por primera vez lo miro. Lo miro a los ojos. Desde el andamio. Dos, tres metros, elevado sobre el atelier. María duerme. Estiro el brazo y muevo el barral que abre la ventana. Y el gato entra. Corre. Entra corriendo y se para junto al platito que le hace a veces de comedor. Me bajo fatigado. Malhumorado. Quería seguir pintando, no ser interrumpido. El gato me mira, confiando que como siempre voy a abrir la bolsa de alimento balanceado. Sirvo una porción en el platito y lo dejo comer un rato. La cuchilla está al alcance de la mano y con un movimiento que a mi edad podría calificarse de ágil, dos segundos después, le separo la cabeza del cuerpo. Dejo la cabeza comiendo del platito y me llevo el resto arrastrado por la cola, chorreando gotas bermellón.

\section*{El reloj de César}
\addcontentsline{toc}{section}{El reloj de César (Por Gustavo)}

\begin{flushright}Por Gustavo\end{flushright}

---César, César, despierta, despierta no me dejes abandonado. Estoy aquí.

---¿Dónde estás, quién eres?

---Aquí César, a tu izquierda. Soy tu olvidado boceto, me ignoras hace días, ¿por qué?

---Pero no seas vanidoso, estaba despejando mi saturada mente para poder nadar por sus intrincados pasadizos.

---Discúlpame maestro, mi inmadurez...

---Es suficiente, no te justifiques más, que sea la última vez, yo elijo cuándo deslizar mis manos sobre ti y tú observas, ¿de acuerdo?

---De acuerdo, señor.

De esta manera, da comienzo un agitado día de César. Pero... ¿quién es César? Es una respuesta que ignoro. Aunque, sus bocetos y cuadros lo saben perfectamente. Igual que a los mismos los explique un curador o un buen crítico en la materia, yo simplemente quiero dejar constancia de la medición del tiempo, tempo, time, temps o como quieran pronunciar a ese nostálgico, fugaz e impredecible compañero del artista.

En el exterior está amaneciendo, es un día despejado, agradable, irresistible hasta para el más laxo seguidor de la divinidad del aire libre. En el interior es un día grisáceo, eclipsado por la lucha de razas que, lejos de amenazar el pulso de César, se plasman en un verdadero duelo de colores, que culminan en una armoniosa conceptualización artística.

Ha pasado algo de tiempo, el hambre comienza a aflorar en el pintor como sucede con la mayoría de los mortales que no quieren dejar de serlo y el boceto que ya ha madurado cede un instante en su apetito. Es la oportunidad, César se digna a almorzar.

La tarde está en pañales, César «en cortos» y su amada María irrumpe en el atelier, es hora de dar su clase de danza. La música clásica invade la sala, los giros generan pequeños tornados de glamour y los pasos son caricias que ese suelo tiene la fortuna de recibir. Este hermoso suceso trunca por unos minutos, extensos para el maestro, la continuación de su pintura.

Pasa un largo rato, la impaciencia se apodera de Don boceto y se escucha: «Señor tenemos algo pendiente»

---Si, discúlpame tú, esta vez tienes razón, ya estoy contigo.

El atardecer se impone, pero ahora es mutuo porque el día de lucha está llegando culminando y, por fin, el insolente bocetillo transmuta a un verdadero cuadro. De este modo, César será libre hasta la gestación de la próxima obra.

\section*{Pinceladas en el tiempo}
\addcontentsline{toc}{section}{Pinceladas en el tiempo (Por Marilina)}

\begin{flushright}Por Marilina\end{flushright}

\begin{center}
Cuesta imaginar el recuerdo 

de un César activo,

inspirado, meditativo.

Busco hallar en cada figura,

una síntesis de su obra,

una metáfora de sus principios.

El misterio de sus preocupaciones diarias.

Sus desvelos más persistentes.

Aplico el ejercicio de rastrear

su esencia en aquellas palabras,

que María prodiga al pasar.

Aquellos susurros de amor, 

que una mujer devota de César,

reitera a cada rato.

Mientras su memoria

desdibuja al amado.

César es hoy reparo 

para la cultura; admiración

para espectadores, espíritu

irrepetible para una compañera,

que eternaliza su vida,

con cada expresión de nostalgia

derramada al pasar.

Miro sus dibujos. 

Contemplo cada trazo.

Me proyecto a aquellos días

de labor incansable,

de sacrificio mental, 

de frutos cosechados.

Me regocijo como él,

en un presente que lo sigue interpelando. 

En un ahora, que redimensiona su arte.

César López Claro y su casa mundo.

Su historia en pinceladas.

Sus anhelos estampados por el óleo.

¡Así son los grandes que veneramos!
\end{center}

%Marilina
\vspace{0.5cm}
\emph{De esta manera, acercamos el mundo del artista a la vida cotidiana de los hombres; de esos hombres y mujeres que desconocían su arte o, en algunos casos, que lograron medir con palabras la cercanía de una amigable relación:}

\section*{Al César lo que es del césar}
\addcontentsline{toc}{section}{Al César lo que es del césar (Por Norma)}

                                                                              \begin{flushright}Por Norma\end{flushright}

Nada más atinado que recordar en un merecido homenaje, que existen personas que con un perfil diferente a lo convencional, son generadoras de otros espacios, en donde viven otras realidades  y dentro de las cuales han grabado un espíritu de pertenencia.

Nada más generoso en el hilo conductor de un pensamiento constante, que muestra a quien sabe ver esencias y energías, hacia dónde va todo el legado de una vida de producción ininterrumpida y donde se aprecia el deseo de ser transmitida a personas concretas.

Nada más oportuno, para estos tiempos anárquicos, en donde no hay lugar para concebir un trabajo como una bella fuente de satisfacción, que descubrir para quienes lo ignoran, la estética que llena ese Museo, generado cuidadosamente, día por día, hasta alcanzar grandes dimensiones que hoy tenemos al alcance de nuestras manos.

Lugar de consultas académicas en donde un arsenal de palabras se encuentra en cada centímetro cuadrado de pared.

Paredes que atrincheran el peligro de un desborde. Desborde de angustia y ansiedad, que luego quedaron trazadas en una infinitud de líneas y colores que antropomorfizaron mostrando lo multifacético de una ética.

Se leía en esa ética una transparencia comunicada a la obra, que debe ser interrogada, como un único lugar de respuesta a tanto misterio que rodea a todo ser humano.

Hablar del Museo es hablar de César.

\part*{Octubre\\«Cuando lo pensado es creído... ¡y nos viene dado!»}
\addcontentsline{toc}{part}{Octubre}

\chapter*{A dormir una siesta}
\addcontentsline{toc}{chapter}{A dormir una siesta}

%Marilina
\vspace{0.5cm}
\emph{Fui heredado, llegué a este mundo inmerso en consejos, costumbres y preceptos. Avanzo en el camino, librándome de algunas ataduras. Mientras me reconozco fiel en la elección de aquello que creo mejor para mi vida.}

\emph{Herencias y creencias, motivos suficientes para que la sinceridad confabule con la palabra y, juntas, nos hagan decir lo que ellas pretenden.}

\emph{Creer para vivir, para sostenernos, para fundamentarnos. Obligarnos y, por qué no también, relajarnos.}

\emph{En esta ocasión, es la poesía la que cede su encanto a un momento del día... ¿heredado? Quizás, pero por todos muy recomendado: ¡la siesta!}

\section*{Una siesta}
\addcontentsline{toc}{section}{Una siesta (Por Ana)}
                                                                                                             \begin{flushright}Por Ana\end{flushright}

Suavemente ha partido esta casi tangible mañana de septiembre...

El sol traspasa, con sus rayos ahora verticales, la copa del viejo limonero y un raro espectro de luz cae sobre la hierba cubierta de pétalos de azar.

Las azaleas, coquetas, reverberan al sol y dejan caer como encendidas brazas sus flores.

En éste, mí lugar preferido a la hora de la siesta, entorno los ojos y me dejo estar...

Una vibración extrañe invade de pronto mi alma confundiéndome: siento que yo, ya estuve antes aquí, y toda esta paz, toda esta belleza yo antes la viví. También esos pájaros cantaron antes y yo los oí. Estoy inmersa en una vieja postal y en esta hora mágica se ha detenido el tiempo. Todo está como entonces para mí. Las azaleas y los pájaros, el aire tibio cargado de aromas vegetales. La quietud reina expectante. De pronto, vocaliza un zorzal que prepara el concierto. Sus trinos me quitan de esta ensoñación...
Y ya está, no es una postal, ni es un sueño recurrente. Es mi vida que transcurre y ésta es una siesta más...

\section*{Ocio creativo}
\addcontentsline{toc}{section}{Ocio creativo (Por Dolores)}

\begin{flushright}Por Dolores\end{flushright}

\begin{center}
Me acomodé en un sillón,

la siesta era propicia.

El ocio me ganaba,

y a él me entregué.

\vspace{0.5cm}

Mientras soñaba despierta

de mi descanso vespertino,

pensaba: «Yo de ésta no salgo,

sin tener una idea brillante»

\vspace{0.5cm}

Al transcurrir del tiempo

que yo me había ganado,

feliz, dormí un ratito.

\vspace{0.5cm}

Mas al despertarme confundida,

no sabía dónde estaba.

Pensé: «arriba, ya dormiste bastante»

Y como siempre hago

apareció la idea brillante...

¡Fui a preparar el mate!
\end{center}

\section*{La siesta}
\addcontentsline{toc}{section}{La siesta (Por Belkis)}
                                                                                                             \begin{flushright}Por Belkis\end{flushright}

Cuando nombran la siesta, se me viene a mi mente mi niñez. ¿A qué chico le gusta dormir la siesta? A ninguno y nostras con mi hermana Marta no éramos la excepción.

Esperábamos que  mi mamá se durmiera y salíamos en punta de pie, si hasta aguantábamos la respiración para salir sin hacer nada de ruido. ¿Y a dónde íbamos? Por supuesto, a comer pizingallo. La verdad, no sé si ahora, esa delicia todavía existe, por lo menos hace largo tiempo que no lo veo en ningún jardín.

Estaba en una casa que daba a la calle, todo para nostras dos. ¡Cómo nos gustaba! Y comerla a la siesta nos parecía más dulce y rica. Siempre con Marta fuimos muy unidas y para romper alguna regla impuesta por mi mamá, las dos éramos especialistas.

Después de cansarnos de comer pizingallo, nos habían hecho con  unos alambres unos cazamariposas. Lejos estábamos de alcanzar alguna. Mientras duraba nuestra travesía, alrededor de una hora, éramos libres, corríamos bajo el sol, jugábamos. Al terminar nuestro itinerario siestero, volvíamos tan felices y contentas que, de la nada, nos salía la risa.

Entrábamos al dormitorio, despacito sin hacer ningún barullo y nos acostábamos a los pies de la cama de mamá. Cuando ella se despertaba, nosotras preguntábamos con aparente ingenuidad: «¿Podemos ir a jugar al patio?». «Sí, ya durmieron. Por supuesto que sí», nos contestaba.

Hasta el día de hoy, cuando nos ponemos a recordar con mi hermana, estamos seguras de que mi mamá sabía lo que hacíamos a la siesta. Pero, por las dudas, nunca se lo preguntamos.

\section*{Oda a la siesta\\(de un santiagueño que se fue a vivir a Capital)\\
 en tres sonetos}
\addcontentsline{toc}{section}{Oda a la siesta (Por Juanjo)}

\begin{flushright}Por Juanjo\end{flushright}

\begin{center}
Tenía una almohada resbalosa

que usaba yo feliz en la catrera.

No era de esas almohadas rastreras

era mullida, pero muy frondosa.

\vspace{0.5cm}

La mía era una piecita modesta,

yo en pijamas la encontraba acostada.

Ella sagaz en la funda arropada,

abrazados a la hora de  la siesta.

\vspace{0.5cm}

Y compartíamos todos, la pieza.

Me acuerdo de las siestas en Santiago.

Éramos como el anís: 8 hermanos.

\vspace{0.5cm}

La extrañé cuando conseguí un trabajo.

Jefe gruñón, horario de ocho a cuatro.

Sin ventanas, de secretaria: un gato.

\vspace{0.5cm}

Perdí la siesta al combatir la hambruna.

Se sabe que en Santiago es venerada.

Pero en la Capital, menospreciada.

¡Con qué ganas hoy me dormiría una!

\vspace{0.5cm}

Taparse con sábana si es verano,

o si es invierno con una frazada.

Imposible quedar embarazada,

Si a la siesta engañaba un hermano.

\vspace{0.5cm}

Recordando ese placer santiagueño,

me pasaba el horario laboral.

Lo mío era existencia intemporal.

\vspace{0.5cm}

¿Qué hacía encarcelado en ese infierno?

Eso era mucho peor que Alcatraz.

Pensé para adentro: no aguanto más.

\vspace{0.5cm}

Catorce meses duré en ese encierro.

Junté mis cosas y me volví al pago.

En la Capital dejé el cuarto impago,

un lunes puse fin a mi destierro.

\vspace{0.5cm}

Me recibieron todos mis hermanos,

prepararon locro, humita y tamales.

Había música, arreglos florales.

Si hasta me permití fumar habano.

\vspace{0.5cm}

Al otro día cumplí lo añorado.

Nuevamente, todos en la piecita,

listos, parecíamos carmelitas.

\vspace{0.5cm}

«Amanece dos veces quien la duerme»,

les dije tras sacarme el pantalón,

recordando el legado de Perón.
\end{center}

\section*{Siesta}
\addcontentsline{toc}{section}{Siesta (Por Norma)}
                                                                                                                   \begin{flushright}Por Norma\end{flushright}

\begin{center}

Siesta que te quiero siesta,

por el descanso que ofreces.

Si tú no estuvieras siesta,

a qué cosa acudiría.

\vspace{0.5cm}

Siesta, bendita siesta,

divides en dos mis días

y permites que los sueños

de siestas guíen mis alegrías.

\vspace{0.5cm}

Antiguas siestas durmiendo,

un verano interminable,

ya que las calles de siestas

se volvían peligrosas.

\vspace{0.5cm}

Aburrimiento de siestas,

dejó huellas rencorosas,

que  al llamado del recuerdo,

molestas acuden a la memoria.

\vspace{0.5cm}

Siesta que te quiero siesta,

destino de travesuras,

aprovechando el silencio.

Tiempo perdido en la infancia,

obligación caprichosa.

\vspace{0.5cm}

Leyendas, mitos, historias,

personajes endiablados,

para que la desgracia no toque

al osado que transgrede.
\end{center}

\section*{Entre las 12 y las 15 me entrego a ti}
\addcontentsline{toc}{section}{Entre las 12 y las 15 me entrego a ti (Por Gustavo)}

\begin{flushright}Por Gustavo\end{flushright}


\begin{center}
Al despertar te recuerdo.

Me llenas de fuerza

para salir expulsado de la cama.

Durante la mañana te anhelo.

Luego, eres el poder divisorio, 

la transmutación del mediodía en tarde.

Eres el divorcio, eres duelo del hoy, 

del ahora, del presente con el ayer, 

el antes, el pasado, para lograr el comienzo 

de la relación con el mañana, 

el después, el futuro.

Eres oportunista, esperas la debilidad del ego

para aprovecharte del ser, del yo.

Eres el no-ego, el anti-ego.

Pero tus antecedentes no me importan.

Cuando llegas sólo quiero disfrutarte.

No quiero ruidos.

No quiero interrupciones.

Sólo quiero fundirme en tus brazos.

Si lo logro, rápidamente pasarás y me relajarás.

Pero si no hay conexión entre ambos,

mis ojos serán como huevos fritos 

y mi cabeza como un bloque de concreto.

Pese a todo, sigues siendo una incógnita

que, vales la pena explorar.
\end{center}

\section*{¡A ti te alabo Siesta!}
\addcontentsline{toc}{section}{¡A ti te alabo Siesta! (Por Marilina)}
                                                                                                  \begin{flushright}Por Marilina\end{flushright}
\begin{center}
¡A ti te alabo Siesta!

Ángel de los desposeídos,

de los trasnochados, 

de los interruptores del tiempo.

A ti te imploro, madre del descanso, 

compañera de la digestión,

trago dulce del mediodía. 

A ti te clamo, seductora de mis desganos, 

réptil de mis tentaciones, 

cómplice de mi lecho.

Si te reconozco en mis 

predilecciones es por 

tu afán incansable de 

perseverar en mi antojo.

Tu nombre es como

un atajo de luz

en la sombra de mi trajinar.

Ven a mí, heroína

de todos los tiempos, 

clávame tu puñal, 

adorméceme como sólo tú

tienes el don de hacerlo.

En la lejanía del ocaso

Te llamaré recuerdo.

Y en la agonía del sueño

maldeciré  tu ego.

Aún así, bendeciré tu imagen

cual retoño anhelado.

¡A ti te alabo Siesta,

Ayer, hoy y siempre!
\end{center}

\chapter*{Leyendas}
\addcontentsline{toc}{chapter}{Leyendas}

%Marilina
\vspace{0.5cm}
\emph{Creer en espacio y en tiempo. Esa cosmovisión que ideamos en conjunto y con la cual damos sentido a nuestra existencia.}

\emph{Santa Fe es nuestro espacio y la leyenda urbana, la fuente de inspiración para creer que, en nuestro lugar, las historias pueden ser inventadas y recreadas hasta el cansancio:}

\section*{Textos del Manual de seres fantásticos de la ciudad de Santa Fe}
\addcontentsline{toc}{section}{Textos del Manual de seres fantásticos de la ciudad de Santa Fe (Por Juanjo)}

                         \begin{flushright}Por Juanjo\end{flushright}

Los siguientes textos fueron transcritos del Manual de Seres Mitológicos de la ciudad de Santa Fe, compilado por el prestigioso zoólogo, empleado público, corredor inmobiliario, abogado e hincha de Colón, Dr. Omar Peretti.

\subsection*{El monodrilo}

Cuenta la leyenda que el monodrilo es un ser nocturno que recorre las noches santafesinas en búsqueda de sus víctimas. De aspecto femenino y rostro aterrador, seduce con su voz a los hombres que, salidos del boliche sin ninguna conquista, están demasiado ciegos por el alcohol como para descubrir su verdadera naturaleza. Recién a la mañana siguiente la víctima es capaz de apreciar sus reales facciones. Peluda como un mono y jetona como cocodrilo, la sensación de horror es tal que ahí nomás, muchos mueren de un paro al corazón.

Se desconoce a ciencia cierta el origen de la criatura. Alguno aventuró que es el resultado de una maldición sobre su madre. Otros, más pragmáticos, aseguran que es una mujer cuyo rostro se desfiguró por la cantidad de productos de belleza que se ponía en el cutis.

Con mirada extraviada y dientes desencajados, la criatura recorre las zonas aledañas a recintos bailables en toda la escala de glamour. Desde el boliche más opulento hasta el tugurio más exiguo.

Se recogen testimonios diversos; hay quienes aseguran que solía pernoctar en una alcantarilla de avenida Gorriti y a la vez no faltó quien manifestó que, al encontrársela cara a cara, reconoció en ella a la esposa de cierto ministro.

Entre los varones santafesinos es común la expresión «que no termines con el monodrilo», como un grito de guerra o expresión de deseo con la cual saludar a los compañeros que salen a  la aventura.

Desgraciadamente no tenemos información de cómo mantener alejado a este temible ser o de qué hacer en caso de encontrarlo en la cama luego de una resaca.

\subsection*{El ciruja inmobiliario}

El ciruja inmobiliario es un personaje imaginario, o no, que recorre el barrio sur de la ciudad de Santa Fe. Con barba larga y cana, usa sandalias durante todo el año. Lleva un costal atado a un palo de escoba que recuesta sobre su hombre al andar.

Lo que distingue a este personaje de otros cirujas, crotos y mendigos que deambulan por la ciudad, es su modus operandi. El ciruja se presenta en las inmobiliarias y pide visitar alguna propiedad. Éstas son casas muy paquetas, en algún barrio patricio y siempre, pero siempre, habitadas. A pesar de su aspecto, logra convencer al corredor inmobiliario de turno sacando un billete de 100 dólares de su bolsa y guiñándole un ojo como diciendo: «de dónde salió ese, hay más»

Ante la mirada atónita de los dueños de la residencia, el anciano se pasea examinando cada rincón. Levanta las macetas, pasa el índice por las rendijas de las ventanas, olfatea la madera de las puertas y golpea con sus nudillos las paredes. Cuando termina, con una sonrisa estampada en el rostro y expresión de satisfacción, como un maestro ante un alumno que ha pasado su examen, les dice que les comprará la casa con la condición de que lo dejen bañarse. Según él, una casa digna de su persona tiene que proporcionarle un buen baño. Si los propietarios se muestran reticentes, recurre a su billete de 100 dólares y la codicia los hace concederle el deseo.

Una vez limpio y oliendo a jabón de lavanda, el buen hombre se retira con la promesa de que lo pensará y nunca más se tienen noticias de él.

Son variadas las especulaciones sobre el origen de esta leyenda. Algunos dicen que es contada por los viejos empleados de grandes inmobiliarias a los nuevos para reírse de ellos y otros aseguran haber vivido la escena que aquí se relata. Los más fantasiosos aseguran que el ciruja es en realidad un millonario disfrazado, muy excéntrico en sus formas de divertirse.

\section*{El espejo viviente}
\addcontentsline{toc}{section}{El espejo viviente (Por Gustavo)}

\begin{flushright}Por Gustavo\end{flushright}


Cuenta la historia que, hace un par de décadas en la ciudad de Santa Fe, se paseaba un espejo viviente, el cual reflejaba en toda santafesina que quisiera aspirar al modelismo, una imagen de cuerpo desproporcionado, no acorde a los estándares exigidos para ingresar en la prestigiosa agencia local de Pold.

Los allegados a las victimas afirman que era tal el trastorno infundado en las jóvenes que no le quedaban ganas de pretender dedicarse a esta profesión. Es más, en la mayoría de los casos terminaron agrupándose con las demás damnificadas, surgiendo de estos encuentros, la fundación de una organización Sui Géneris, denominada Organización con fines persecutorios del Espejo Viviente o más comúnmente conocida como OCFP del Espejo Viviente.

El fenómeno cobró fuerza y algunos relatos con la prensa local expresaban las siguientes palabras: «Es un horror, yo me vi enfrente mío primero y luego a los costados, después atrás....hay! No me da miedo, era yo toda deformada, oh! No soy así....no vasta, no puedo hablar más.... ayuda...!» Otro entrevistado declaraba: «Están todas chapitas, eso les pasa por no comer. Míreme a mí, con mi panza yo estoy orgulloso, no sabe el asado que me clave ayer».

Luego de más de un quinquenio de lucha contra este reflejo torturador, un brujo que prestó sus servicios a la OCFP descubrió que el victimario no era ni más ni menos que una modelo de Pold, ya entrada en años, dedicada a la parapsicología; medio por el cual engendraba el miedo y la ira de las santafesinitas. De esta manera, la renovación del plantel de Pold ya no tuvo trabas...

\section*{Bello fantasmita}
\addcontentsline{toc}{section}{Bello fantasmita (Por Nelly)}
                                                                                          \begin{flushright}Por Nelly\end{flushright}

Pequeño fantasma del barrio Sur, ¡qué alegría haberte conocido  y qué pena tu final inadvertido!

Todas las mañanas te veía cuando pasaba a hacer mandados. Yo era muy niña y tú pequeñita, siempre en el umbral de tu casa, sentadita recién levantada, con un gran chupete en la boca. Tus labios estaban ensanchados de tanto chupetear y tus rulos enmarañados, no tenías quién te los arregle, porque tus papis trabajaban.

Yo no quitaba los ojos de tu figura, te veía preciosa, una muñeca. Con qué ganas te hubiese hecho mía, mi muñeca para peinarte, quitarte el chupete y darte una rica leche como la hacía mi mamá.

Pasaron los años, creciste, te convertiste en  una bella señorita, muy asediada por los jóvenes del barrio Sur. Nunca nos saludamos. Yo te admiraba por bella y desenvuelta. Se murmuraba que aceptabas muy gustosa los galanteos, pero tu corazón se inclinó por un amor prohibido. Seguramente éste te deslumbró con sus palabras y experiencia. Según cuentan, no se quería separar de su esposa, sufriste, te arrancó tu joven corazón, tus padres se oponían a esta relación, mas tu juventud no supo, no quiso o no pudo romper el embrujo del primer amor.

Una noche hermosa, con su cielo estrellado titilando como nunca las estrellas, regresaste a tu casa, tus papá no estaban, Qué dolor habrás tenido, qué sufrimiento te envolvería, qué cegada estarías, que tomaste tan cruel decisión.

Tus padres te encontraron con un hermoso camisón blanco en la cama. Atravesando con tus grandes ojos verdes el techo de tu habitación, dos estrellas se habían pegado en tu pecho, que hacían brillar una gran rosa roja en él, justo en tu corazón herido con tu propia mano.

Te mataste por amor, tú dejaste este mundo y él continúa, envejeciendo. Y cuando lo veo pienso; «por un instante, ¿volverás en sus recuerdos?»

Yo me  alejé del ese lugar. Dicen que en la casa en que vivías se han construido unos hermosos departamentos, mas en las noches de verano cuando el cielo se tapiza de estrellas se ve la sombra de una niñita con rulos enmarañados y con un gran chupete en la boca, sentadita en el umbral. Es que quizás, vuelves al único lugar y en el único estado, en que fuiste feliz...

Los más ancianos que todavía perduran en el barrio dicen que te tiene miedo y si se paran para observarte, te pierdes entre los rosales rojos del jardín del edificio que crecieron sin que nadie sepa quién los plantó.

Para ti, dulce duendecito, mi recuerdo y deseos de paz.      

\section*{Reptilitus}
\addcontentsline{toc}{section}{Reptilitus (Por Belkis)}
                                                                                                     \begin{flushright}Por Belkis\end{flushright}

Esto que voy a contar le pasó a Julia, una vecina de al lado. Era un viernes de invierno y ella junto a su amiga salieron a caminar. Ambas llegaron hasta la Avenida Aristóbulo del Valle y decidieron seguir caminando hacia la zona norte. Recorrieron varias cuadras, charlando de todo un poco, hasta que estaba ya por oscurecer y sin darse cuenta se toparon con la Avenida Gorriti. En ese momento, observaron un tumulto de gente alrededor de una alcantarilla. Se acercaron y preguntaron ansiosas qué era lo que estaba pasando. Un señor mayor les contestó: «Y hoy es viernes, hay luna llena y en esa alcantarilla sucede siempre algo extraño.».

Los autos y colectivos comenzaron a detenerse en lugar. La luna posó sus rayos sobre un cuadrado enorme y allí asomó un ser de color verde oscuro, con ojos rojos y saltones, tenía una piel rugosa. El espécimen intentó salir, pero al ver tanta gente, se asustó y se metió de nuevo en la alcantarilla. Al girar, se le vio una enorme cola como de cuatro metros, una suerte de reptil gigante.

Algunos aplaudían, otros gritaban. Pero, mi vecina Julia y su amiga se asustaron muchísimo y no podían creer lo que habían visto. Ni bien llegaron al barrio, fueron a mi casa a contarme lo sucedido y yo medio en broma, después de escucharlas, les dije: «Y bueno, si Bariloche lo tienen a Nahuelito, nosotros aquí en Santa fe lo tenemos a Reptilitus»

Cada vez que sale la luna llena, en la alcantarilla de Gorriti se agrupan varios vecinos con una mezcla de esperanza y temor, ansiando poder verlo y así, transmitir a sus futuras generaciones, la tan misteriosa aparición de Reptilitus.

\section*{Leyenda}
\addcontentsline{toc}{section}{Leyenda (Por Norma)}
                                                                                                      \begin{flushright}Por Norma\end{flushright}

¡Vieja de los trapos, vieja de los trapos!, gritábamos y salíamos corriendo.

Todos los días a la misma hora, sucedía lo ya casi esperado. Luego, nos encerrábamos en nuestras casas para almorzar y dormir la siesta, sin culpas ni remordimientos.

Pero el paso del tiempo deja marcas que aparecen con el recuerdo, cuando uno menos lo piensa.

Pongo el cuerpo y el espíritu en este recuerdo que se repite cada vez que vuelvo con mi pensamiento al barrio donde nací.

La «vieja de los trapos», también llamada «vieja de las siete medias» o «la húngara», era una bella mujer, aunque para nosotros todo lo que tuviera un par de años más, lo considerábamos viejo.

Lo de «bella  mujer» fue la imagen que tuve que construir con el paso del tiempo, considerando al tiempo un transcurrir, que marca procesos, cambiantes en sus perspectivas.

Alta, delgada, muchísimo cabello largo enroscado en trenzas desprolijas. Vieja de los trapos, tenías los ojos claros y la ropa encimada, que terminaba en largas polleras, mostrando un trozo pequeño de tus piernas envueltas con demasiadas medias.

Dormía en el Parque Garay y salía muy temprano para hacer su cotidiana recorrida.

Era el mediodía. Ese día me ensañé con ella, todavía no recuerdo los motivos. Pero, comencé a provocarla con morisquetas y movimientos exagerados de manos y brazos.

La hice enojar, conducta que no le conocíamos. Comenzó a perseguirme por todo el barrio, preguntando a los vecinos por la chica de los ojos celestes.

Encontró mi casa y se quejó a mis padres. El resto, se lo dejo al lector para que imagine el desenlace

\section*{De cómo Juan de Garay dio nombre a Santa Fe}
\addcontentsline{toc}{section}{De cómo Juan de Garay dio nombre a... (Por Marilina)}

                                          \begin{flushright}Por Marilina\end{flushright}

Es arduo conocido por los santafesinos que un tal Juan de Garay fundó la ciudad un 15 de noviembre de 1573. Las diferentes versiones históricas dan cuenta de una conquista improvisada en la confluencia de los ríos Paraná y Salado o Saladillo.

Sin embargo, la cuestión nominal no fue un hecho aislado o mezquino, más bien responde a una leyenda olvidada con los años, pero que tiene su fundamento anecdótico muy bien establecido.

Cuenta la historia extraoficial que, en plena expedición, Juan de Garay (nombrado el alguacil mayor de las provincias del Plata) sufrió de problemas gastrointestinales muy agudos. Sus compañeros de travesía aseguraban que tantas leguas recorridas a caballo habían afectado el aparato digestivo de este hombre.

Si bien llevaban consigo semillas y plantas, en el camino sólo habían bebido aguardiente, después de la última cena en la que hubo pescado a la pimienta negra, junto con trigo fermentado traído desde la propia España. Garay era sí un conquistador, pero no por ello estaba exento de ciertas contingencias y necesidades fisiológicas comunes a todo ser terrenal.

De esta manera, Don Juan seguía su prometido de fundar ciudades. Aunque, su malestar iba en ascenso y, poco a poco, comenzó a perder los estribos de su compostura. 

---Paremos en aquél rancho ---le habría dicho en forma agónica a la comitiva integrada por 75 guaraníes y 9 andaluces.

---Necesito reponerme si es que habremos de alcanzar nuevas conquistas.

Su gente, así lo hizo y, al escuchar el frenado de los caballos, salieron del lugar un hombre y una mujer mayor.

Garay enfiló como un rayo para la parte trasera del rancho y a los cinco minutos regresó hasta la puerta del mismo, ante la mirada expectante de todos. Su semblante estaba, ahora, más relajado.

En ese momento, los dueños de la vivienda se apresuraron a darle la bienvenida. Fulgencio Vera y su progenitora, Clorinda Cruz, no tardaron en hacerlo pasar y allí fue donde la mujer ofreció a Garay un brebaje preparado por sus manos. Como una ráfaga, ésta colocó en un mortero hierbabuena, menta, orégano, jengibre, ajenjo y canchalagua y media copa de un líquido amarronado. 

---Tome Don, La cordial es una bebida infalible para estos males, es un mal asiento lo que tiene usted.

Don Juan aceptó la infusión, no sin dejar de mostrar gestos de desagravio, a medida que el líquido ingresaba por su esófago. Luego, continuó con una charla de agradecimiento y los tres se despidieron. 

Garay retomó la senda hacia el norte y con el paso de las horas, su cuerpo y su estómago volvieron a la normalidad, se sentía aliviado, purificado por dentro.

Entrada la tarde, la expedición se detuvo en el lugar indicado y fue lo que hoy se conoce como Cayastá, el sitio que Garay eligió para su hazaña.

---¿Qué nombre ponerle? ---se dijo e inmediatamente recordó los nombres de aquellos nativos solidarios que lo asistieron--- Sí, se llamará Santa Fe como síntesis de mi perseverancia y en honor a los mencionados, habrán de recordarla como Santa Fe de la Vera Cruz. Y si algún día, por alguna razón, ella cambiara su nombre, dejaré sentado en acta otro título a modo de homenaje a la homeopatía del lugar: Santa Fe, La cordial.

Pasados algunos siglos de aquél episodio hay quienes refutan, en verdad, esta denominación y proponen hablar de «Santa Fe de la Cruz de Vera».

No obstante, el machismo fue y sigue siendo un discurso imperante.

Hasta el día de hoy, los habitantes de ésta y, aún, los visitantes que llegan a la ciudad creen hallarse en un lugar de gente afectuosa, pero no es tan así. La cosa viene por otro lado como se habrá podido observar y «a río revuelto... extranjeros contentos»

El relato se propagó en el tiempo y en algunos libros de historia, se menciona por lo alto a Fulgencio Vera y se dice que «el mismísimo Juan de Garay tuvo oportunidad de conversar con él». Los detalles son mínimos y confusos, abunda la imprecisión. En cambio, de la pobre madre, nadie recuerda su aporte y muchos son los que ignoran que la cordial fue ese remedio indispensable para que la Santa fe se fundara. 

Entonces, ¡Nada de generosidad y afecto del habitante, ni otra cosa que se le parezca! Santa Fe es Santa Fe gracias a la receta casera de una madraza sabia, naturista y de buena cepa.

\section*{La rosa negra}
\addcontentsline{toc}{section}{La rosa negra (Por Dolores)}
                                                                                                   \begin{flushright}Por Dolores\end{flushright}

El auto era el único vehículo que transitaba esa mañana por el camino de tierra. Al volante iba Víctor, un periodista que buscaba historias o leyendas.

Éste vio un hombre que iba en la misma dirección, paró y le preguntó:

---¿Voy bien a la Estancia, la madrugada?...

---Sí Señor, contesta el paisano. Yo voy para allá.

---Bueno, ¡suba!

El paisano, curioso pregunta:

---¿Y qué anda haciendo por acá?

---Mire, yo recojo historias, leyendas, relatos de los pueblos, y me dijeron que acá había una leyenda de la rosa negra.

---Ah, sí usted se refiere a la rosa que apareció en la tumba de los Yender...

---¿Y usted la conoce?

---Sí ---dijo el paisano gustoso de poder contarla. Para esto, Víctor había parado la marcha del auto y prendido un cigarrillo.

---Entonces lo escucho.

---Había una familia de alemanes, con un solo hijo, según cuenta la  gente, padre e hijo se enamoraron de la misma mujer, Rosita, la hija del pulpero.

---¿Y era linda?, preguntó Víctor.

---Dicen que no había en tantas leguas a la redonda, una paisana más linda y querendona. Una noche, el padre lo vio al hijo hablando con la rosita, detrás de la pulpería, parece que discutieron fuerte y el padre que tenía un revolver le pegó un tiro al hijo y se suicidó. La madre hizo enterrar a los dos en la misma tumba y a los pocos días sobre ésta apareció una planta de rosas y la flor era negra.

---¿Y eso cuándo sucedió?

---Hace muchos años y cuando se habla de esa tumba, se dice la de la rosa negra. Pero de noche nadie pasa cerca del cementerio, porque se escucha un grito lastimero llamando a la Rosita.


\section*{El caballero de la noche}
\addcontentsline{toc}{section}{El caballero de la noche (Por Hernán)}
                                                                               \begin{flushright}Por Hernán\end{flushright}

Cuenta la historia que el caballero de la noche era una de las personas más iluminadas que habitaba esta tierra. Este ser se caracterizaba por vivir en planos multidimensionales y por poseer poderes impensados para las personas. También  lo llamaban El Guardián de la luz o Guardian of the Light, en ingles. Personaje místico si los hubo, uno de los mas influyentes: nieblas, bosques, árboles, murmuran su nombre. Se dice que solía dormir junto al fuego y lejanas estrellas guiaban su destino.

Hay quienes dicen que se desplazaba a gran velocidad durante el alba y en una moto de pista.  Su misión era vencer al mal y liberar a la humanidad de todo presagio maligno. 

Su poder era tal que equivalía a más de 10.000 ejércitos y era capaz de  traspasar  todas las fronteras, todos los planos místicos ancestrales, el Alfa, el Omega, no conocía de límites.

Un círculo de fuego era su protector y debía luchar contra la oscuridad, que lo seguía como un lobo, con el deseo de verlo caer.

El mensaje del caballero habla de la necesidad de los hombres por encontrar el reino olvidado, hallar la salvación de los hombres perdidos. Este ser conoce de la maldad de los humanos y ve cómo, poco a poco, los hombres van desolando la tierra y el mar, en un tiempo que es de odio y de dolor.

El caballero te hallará en la noche y te dirá: 

«Necesitas despertar tu corazón. El reino olvidado está sólo dentro de ti, puedes ser rico, mendigo o ladrón, duelen las llagas en tu corazón, volvamos a empezar y el reino lograra salvar la tierra y a toda la humanidad»

\chapter*{Dios}
\addcontentsline{toc}{chapter}{Dios}

%Marilina
\vspace{0.5cm}
\emph{El mes extingue algunas creencias, al tiempo que revive otras, más profundas, más existenciales, más sublimes, como para ser tocadas por la palabra. El desafío está en marcha y Dios, la creencia máxima, es la que finalmente golpea a la puerta de nuestro Taller... ¿habrá algún empleo para él?}

\section*{Siempre Dios}
\addcontentsline{toc}{section}{Siempre Dios (Por Ana)}
                                                                                                          \begin{flushright}Por Ana\end{flushright}

¿Qué Dios busca trabajo? ¡Qué disparate, por Dios! Si no hay en el mundo alguien más ocupado que él. Sea cual fuere el nombre que te guste ponerle o el rostro que quieras que tenga.

A  cada segundo le llueven sin tregua: «Dios mío, ayúdame a comprar mi casita», «Dios mío, por favor, hacé que rinda bien mi examen», «Te ruego mi Dios que se cure mi hijo». «Ay Dios, si tú quisieras hacé que regrese mi amada».

Ojos negros, azules pardos, grises, verdes, escudriñan el cielo en su búsqueda, mientras juntas con fervor las manos en el pecho.

Manos blancas, manos negras, suaves, rudas, jóvenes, sarmentosas, se alzan al cielo temblorosas en la súplica.

No importa en qué idioma le hables, él te escucha en su idioma, que es el tuyo, el mío, el de todos, desde que despunta el alba, cuando decimos «¡al que madruga, Dios lo ayuda!». Así hasta que cae la noche y buscamos el descanso, también se lo pedimos a él: ¡Dios mío, dame buenos sueños!»

¿Todavía crees que Dios necesita buscar trabajo?

¡Ya ni lo pienses! Ve tranquila, qué Dios te acompañe. ¡Que Dios te acompaña!

\section*{Sobre Dios}
\addcontentsline{toc}{section}{Sobre Dios (Por Norma)}
                                                                                                   \begin{flushright}Por Norma\end{flushright}


¡Tarea difícil ésta la de entrevistar a Dios! 

Durante el tiempo que llevamos en este trabajo es la primera vez que nos hace temblar el tener cara a cara a un ser con una investidura universalmente conocida y respetada.

La oficina cumple con variadas funciones a la vez, que en el horario laboral, resuelve severos problemas que presentan los habitantes que buscan ayuda en un lugar tan especial.

La llamada que tuvimos de Dios nos llegó telepáticamente, a todos los empleados, al mismo tiempo. No fue necesario utilizar ningún medio de comunicación en red para que se enteraran lo que estaba ocurriendo. Había perdido su empleo, pero conservaba su poder.

Dentro de nuestra gran bolsa de posibilidades, llena de ofertas guiadas por las demandas de nuestros clientes (siempre satisfechas y con las cuales jerarquizamos nuestro espacio laboral), ¿tendría ésta algo que pudiera interesar, justamente a Dios?

Se hizo la hora, el momento y el lugar del gran Encuentro, y la emoción sensibilizándonos tanto que pudimos sentirnos como un grupo de espíritus convocados.

La Babel que nos había acompañado siempre, quedó disuelta al escuchar a una voz decir despacio: «No hables Dios ahora, tomáte un descanso que haga sentir tu ausencia, es cuestión de tiempo. Estamos seguros que lloverán llamadas selectas para que retomes tus trabajos. Pero, esta vez desde otro lugar...»

\section*{Capítulo Uno}
\addcontentsline{toc}{section}{Capítulo Uno (Por Juanjo)}
                                                                                                       \begin{flushright}Por Juanjo\end{flushright}

Dedos de luz y polvo atravesaron el sudario escarlata que formaban las nubes sobre la ciudad. El día llegaba a su fin y un hombre cansado y arrugado estaba sentado en el suelo, con la espalda apoyada sobre los muros de la catedral Nuestra Señora del Carmen. Los pies desnudos y sucios, una manta harapienta como único vestido,  y cabellos blancos y desordenados pintaban al personaje. Tenía la cabeza caída, como si se quisiera ocultar entre sus piernas. Los pies de otro hombre se detuvieron a su lado. Este, a diferencia, lucía un traje de tres piezas de hilo color hueso y llevaba en su mano un bastón de marfil. De la solapa del saco colgaba un prendedor plateado con forma de ángel.

---Mi buen amigo. Hace siglos que no lo veía. ¿Cómo se encuentra usted?

En anciano seguía con la cabeza gacha y dió la impresión de que no reparó en la presencia del misterioso caballero.

---Ahora me dedico a los recursos humanos. Tome, le dejo una tarjeta por si se decide a trabajar.


Horas más tarde pasó por el mismo lugar una doncella. Vestía una falda larga y rosada y una blusa celeste. Al verlo tirado lo reconoció. Con delicadeza le tomó el rostro entre las manos pero este no la miraba. Tenía la vista perdida, como mirando para adentro.
La noche ya se había posicionado y la luna derramaba un resplandor plateado sobre toda la escena. Sacó un pañuelo y le limpió la roña de entre las arrugas. Mientras lo hacía, el viejo se quedó dormido.


El reclamo de unos gorriones que le pedían alimento a su madre lo despertó. Entre trapos sucios y diarios viejos abrió los ojos al nuevo día. No recordaba nada de la noche anterior, pero se sentía mejor por primera vez en semanas. Cuando, con dificultad, se puso de pie, encontró la tarjeta entre sus ropas; tenía un ángel estampado y una dirección.

Caminó por la avenida principal y esquivando carrozas mecánicas llegó a la fuente donde bebía cada mañana. Un niño de cutis mohoso sostenía una jarrón en sus hombres y del jarrón se vertía un chorro de agua cristalina que iba a parar a un piletón. Se apoyó en el borde y estiró el pescuezo. Para completar el desayuno real, sacó un mendrugo del bolsillo y lo engulló mientras siguió caminando a su destino.

El edificio se levantaba en la esquina en la que se encontraban la avenida principal y una calle con nombre de santo. Una proeza arquitectónica que, para unos ojos cansados como los suyos, parecía llegar hasta el mismísimo cielo. Un muchacho terminaba de limpiar los cristales de la puerta de entrada y secaba su herramienta en el pantalón cuando el viejo entró. Un guardia de las dimensiones de un ropero modesto lo esperaba con una sonrisa tan falsa como el arma que colgaba de su cintura. El anciano no atinó a más que extender la mano y entregarle la tarjeta. El centinela la miró y, sin hacer comentario alguno, caminó a su puesto para hacer una llamada.

---Suba hasta el piso 7. Lo espera el Sr. Corelli.

Se metió en la cápsula de metal y oprimió el botón que le habían indicado.

Corelli era un hombre de unos 50 años. Vestía un elegante traje de hilo a rayas negras y grises y estaba peinado con gomina, como los compadritos de antaño. Lucía unos delgados trazos negros a modo de bigotillo con los que, al parecer, intentaba conferirle una estética afrancesada a su tez oscura.

---Estimadísimo Señor. Estaba esperando su visita. El patrón me dijo que era muy probable que se de usted hoy una vuelta por las oficinas. El negocio crece rápido, pero contar con alguien con vuestra trayectoria en las filas de La Legión sería más que un golpe certero a nuestro competidores. Ni que hablar de lo que significaría para La Legión como marca. Y que sea yo el encargado de entrevistarlo... el bono... el ascenso... Pero no nos adelantemos. Tome asiento por favor y comencemos con la entrevista.

El viejo, que no alcanzaba a entender las palabras de su interlocutor, aceptó la silla que le ofrecía, más por cansancio que por otra cosa.

---Aquí mismo tengo su hoja de vida. Por supuesto que tratándose de usted no necesito revisarla, es tan solo un formalismo. Veamos... aquí dice que tiene basta experiencia en medicina, leyes, milicia y docencia. Todo un cuadro, ¿eh? Hay varios puestos vacantes donde alguien con sus... habilidades, podría ser muy bien aprovechado.


El viejo salió del edificio con un cheque en una mano y una carpeta en la otra. El muchacho que limpiaba los vidrios seguía entretenido en el mismo metro cuadrado de cristal con el que estaba cuando llegó. Se sentó en el banco de una plaza cercana y empezó a leer mientras daba voraces mordiscos a una manzana que un verdulero le había regalado cuando lo vio pasar. Instrucciones:

\begin{enumerate}
\item{Ir a la barbería San Marcos y preguntar por el dueño. Pedir un corte y lavado completo.}
\item{Visitar la sastrería San Lucas y adquirir un traje oscuro y un par de zapatos italianos. Poner todo en la cuenta de La Legión.}
\item{Llamar a la puerta del jardín maternal Los Querubines y preguntar por Magdalena.}
\end{enumerate}

Irreconocible luego de completar los items 1 y 2, el antes mendigo parecía un músico de jazz norteamericano. No estaba en las instrucciones, pero se tomó la libertad de agregar a la indumentaria un par de lentes negros tipo ranger. Buscó en la guía la dirección del establecimiento y, revestido de la confianza que le confería su nuevo atuendo, se fue caminando a completar el tercer punto.

La puerta de Los Querubines era una estructura maciza de roble blanco adornada con la figura de un niño alado. El timbre estaba en el ombligo de la escultura. Toco y esperó.

Una mujer de unos cincuenta años apareció tras el umbral. Lo miraba fijo con sus ojos celestes y tras una cortina de arrugas que surcaban su rostro todavía se podía reconocer la belleza que alguna vez supo portar. Sin mediar palabra le hizo un gesto para que la siguiera. Pasaron un salón lleno de bebés jugando, para luego bajar por una escalera de madera. Entraron en una habitación pequeña, en donde apenas cabían los dos y se sentaron frente a frente a la luz de una vela. Magdalena lo miraba como si lo conociera de toda la vida.

---No me recuerdas, ¿verdad?

El viejo abría los ojos y mostraba las palmas de las manos como disculpándose.

---No importa, es todo diferente ahora. Lo que fue ya no existe y lo que podía ser desapareció como una duna en el desierto. Que tu también trabajes ahora para el patrón es un signo de que no quedan esperanzas. Toma este sobre, en el interior está tu encargo.

Siguieron contemplándose por espacio de varios minutos más. Cuando la mujer, con un gesto, indicó que la visita llegaba a su fin, ya eran varios los racimos de cera negra que se acumulaban a los pies de la vela.

Volvió a la plaza donde había leído el contenido de la carpeta y, asegurándose de que no haya nadie cerca, abrió el sobre. Dentro no encontró más que una fotografía. En el retrato se veía a una muchacha de mirada triste. Lucía una falda rosada y una blusa celeste. Detrás figuraban una dirección y un horario. El horario era 30 minutos más tarde y la dirección 40 cuadras al norte.

Sin saber muy bien por qué, empezó a correr. Corrió, saltó, se trepó, se tropezó, se incorporó y siguió corriendo.

Cuando habían pasado 29 minutos y estaba a una cuadra del lugar, alcanzó a divisar a la muchacha que forcejeaba con 3 hombres vestidos totalmente de negro. Inmovilizado por la situación, vio el reflejo de luz del cuchillo que uno de ellos acababa de desenfundar. Todo estaba más claro ahora. La Legión lo había enviado a salvarla. No sabía por qué ni sabía como hacerlo, pero esa era su misión. Corrió hacia el grupo gritando y agitando los brazos en el aire y, cuando estuvo ahí, no tuvo mejor idea que tirar todo el peso de su cuerpo sobre el que sostenía el cuchillo. Antes de caer, el desconocido logró asestarle la hoja metálica en el vientre. Los otros dos escaparon y cuando el viejo se incorporó, encontró su costado lleno de sangre. Cayó de rodillas y la miró a la muchacha a los ojos. En los ojos había la misma o más tristeza que en la fotografía. El tercer hombre también huyó pero en el lugar dónde había caído el viejo alcanzó a ver un prendedor con forma de ángel.   

\section*{Dios busca trabajo}
\addcontentsline{toc}{section}{Dios busca trabajo (Por Hernán)}
                                                                                          \begin{flushright}Por Hernán\end{flushright}

Dios viene a la tierra para ejercer su labor como arquitecto, las primeras personas quedaron impresionadas y le preguntaban quién era él y él les decía que era Dios. A la gente le costo creer que era cierto. Cuando lo interrogaron acerca de por qué él se vino del cielo a la tierra, éste les respondió: porque había una gran necesidad de ayudar a sus hijos, veía a muchas personas perdidas, que se sentían solas, deprimidas, sin lograr sus objetivos, mucha indiferencia entre las personas, que es lo que más le dolía.

Además, agregó que las riquezas del mundo estaban muy mal distribuidas y muchas personas no tenían lo suficiente para vivir en condiciones dignas, que el poder y la acumulación enceguecía a los hombres, aferrados sólo a lo material.

Los hombres padecían un vacío interno, estaban incomunicados, desprotegidos.

De esta manera, su misión en la tierra iba a ser  edificar un mundo más justo, con lugares diferentes y comodidades para todos. Un mundo en el que los chicos que lo habitaran no tendrían  nunca más, frío, ni sed, ni hambre y sí, amor, comprensión, cuidado.

Dios edificaría así, un mundo nuevo, al que llamaría la Nueva Jerusalén.
  
\section*{Replanteos de Dios}
\addcontentsline{toc}{section}{Replanteos de Dios (Por Gustavo)}

\begin{flushright}Por Gustavo\end{flushright}

Dios se encuentra un poco decepcionado con sí mismo. Está notando que los días se le hacen cada vez más extensos. Al llegar cada noche se siente vacío, como que no logró cumplir sus objetivos. A causa de estas sensaciones, un poco desconcertantes para él, comienza a replantearse la posibilidad de que algo en su ser anda mal pero... ¿a quién recurrir?, se pregunta, no tiene alguien superior al cual dirigirse.

Por otro parte, se plantea internamente una serie de interrogantes:

¿Quién me colocó encima de todos? Acaso... ¿yo no soy parte de todos? Me hablan, imploran, agradecen, pero al fin de cuentas muchos me ven afuera de ellos, allá en un lugar remoto que ni siquiera sabrían describir con cierta verosimilitud. 

¿Quién les trasmitió ese mensaje? No era mi intención, yo quería ayudarlos a integrarse con el universo y conmigo, todos juntos. 

Porque yo soy parte de todos y todo; y todo es parte mía. ¿Tan difícil es que lo entiendan?

Luego de esta cadena de replanteos autorrealizados por Dios, se propone cambiar la estrategia, ir paso a paso, como quien escala el Aconcagua, el Himalaya o el mismísimo Everest. 

Como primer peldaño a subir decide buscar un trabajo entre los Homo Sapiens (del latín Homo=hombre, sapiens=sabio). Compra el diario del domingo y elige de entre los avisos clasificados uno de su interés, se trataba de un curso de teatro.

Es el día de la entrevista. Dios llega unos minutos antes de la hora señalada, se sienta, sus piernas tiemblan, la gota gorda cae de su frente, su mente se pone en blanco, está como paralizado. Gracias a una inhalación salvadora que airea su cuerpo, logra dilucidar un dispenser, se acerca y toma un vaso con agua, su ansiedad cesa. Es su turno, lo llaman:

Ya en la entrevista, luego de unos minutos iniciales en que se genera un intercambio de preguntas y respuestas suaves, de introducción, vienen las interpelaciones centrales del entrevistador:

---¿Qué perdió en su anterior trabajo, señor? 

---La risa, contesta el aspirante.

---Y... ¿qué quisiera hacer en mi academia?

---Desarrollar el oficio de humorista.

---Qué interesante lo que me plantea «jefe», pero... ¿con qué cuenta para hacerlo? 

---Con una vasta experiencia.

---Podría explayarse un poco más, dice el entrevistador.

---Bien, en esencia tengo muchos cuentos e historias que contar, pero necesito de sus técnicas teatrales para que transmitan alegría. Busco una manera alternativa a mi anterior trabajo de alegrar a las personas. Quiero poder causar la risa hasta el éxtasis de la gente, creo que de esta manera no perdería mi esencia, porque al fin de cuentas la risa es sabia consejera de la paz.

\section*{Una visita especial}
\addcontentsline{toc}{section}{Una visita especial (Por Nelly)}
                                                                                              \begin{flushright}Por Nelly\end{flushright}
 
Después de muchos días grises, opacos, hoy el cielo luce todo su esplendor, mi ser agradece a la naturaleza este regalo; voy a disfrutarlo. Mi casa tiene un lugarcito especial para mi deleite espiritual, momentos en que los recuerdos acuden y a cada uno de ellos les dedico un instante envolviéndolos con buenos sentimientos. Sumida en la galería de visitantes mentales me interrumpe un suave y firme llamado a la puerta.

El brillo del día me predispone gustosa a atender, es un hombre quien extiende su mano, mira tiernamente mis ojos y sonríe, no descifro su sonrisa ¿alegre, melancólica, cautivadora, dulcemente penetrante?

---¿Con quién tengo el gusto? atino a decir y  quedo suspendida en su mirada.

---¿No me conoces? 

Mi cabeza de un lado a otro sin palabras.

---Cómo, si llevas a mi hijo en tu pecho...

Mis manos tocan mi pecho  y allí está su hijo, es verdad, tengo a Cristo con una cadenita en mi pecho.

Es Dios quien ha golpeado mi puerta.

---Pasa, pasa Dios...

---Hija, ya estamos adentro, y ríe, ríe como yo, como mi esposo, como mi hijo, mi hermano, mis padres, mis amigos, a todos ellos veo reflejados en esa risa ¡cómo nos parecemos!

Mis manos en las suyas, mis ojos atrapados en los de él en sublime conexión.

---Bueno hija, tranquila, estamos juntos, quiero escuchar tus requerimientos, esos que hace largo tiempo no te animas a hacerlos en voz alta.

A borbotones felices, digo:

---Quédate.

---Siempre estoy, a veces estás tan ocupada que no reparas en mi presencia.

---Sálvame de la soledad.

---Para eso te envié el amor y tú le pusiste el nombre de esposo, nunca te dejo sola; te envío mis ángeles, escúchalos.

---Dios, mis hermanos los que sufren,  los que no te conocen....

---No te preocupes, ayúdalos.

---Dios, no sueltes mis manos, soy muy débil.

---Nunca te he soltado, te he fortalecido, te hice madre.

---Dios no te vayas tan pronto, sonríe, qué bella tu sonrisa.

---Hace mucho que estoy contigo, mira el reloj, se ha detenido para que disfrutemos de este momento; yo te necesito mucho más a ti.

---¿Qué hago señor?

Interiormente me pregunto ¿qué misión difícil me das? ¿Seré capaz?

---No te asustes, sólo piensa en mí y pondré en tu boca la palabra justa, tus hermanos empezarán a conocerme, cambiarán.

---No te esfuerces, no me dejes encerrado dentro tuyo, llévame donde vayas y déjame donde más me necesiten, en la cama del enfermo, en la mesa del hambriento, en la copa del sediento, junto al anciano olvidado. Si la emoción te invade y tus palabras se silencian, en mi nombre, pon  tus manos en su hombro, así seremos tú y yo, tu corazón y el mío, sembrando amor, amor es sólo lo que necesita la humanidad para alcanzar la felicidad., Amígate con la prostituta, ella y tú son una sola para mí, no reniegues de los que no piensan y actúan como tú, a ellos también les pesa su cruz, alíviaselas.

Y como meta recuerda; «perdona, perdona» a los  que te mienten, a los que te traicionan, a los que te hacen llorar, refúgiate en mi hijo, acurrúcate en los brazos de la madre de mi hijo que también es tu madre, no sufras, sé feliz para eso te quiero en esta tierra.

Ahora voy a visitar a un vecino que en este instante está llorando.

---Dios vuelve pronto.

¡Aleluya, hoy Dios ha visitado mi casa!

\part*{Noviembre\\«Un mes para reir-nos y pensar en lo que somos»}
\addcontentsline{toc}{part}{Noviembre}

\chapter*{Escritos con humor}
\addcontentsline{toc}{chapter}{Escritos con humor}

%Marilina
\vspace{0.5cm}
\emph{La palabra se va moldeando cada vez más. Las motivaciones para la escritura se hacen rogar cada vez menos. Y así, continuamos por el camino de la palabra, exigiéndole otros caprichos, como por ejemplo, la humorada:}

\section*{Fabricante de zapatos juaneteros}
\addcontentsline{toc}{section}{Fabricante de zapatos juaneteros (Por Belkis)}

                                                             \begin{flushright}Por Belkis\end{flushright}

Para el que no me conoce, les cuento que siempre ando medio distanciado de las cuestiones laborales. La última vez que empecé a vender paraguas, en la esquina de La Rioja y San Martín, era un día inestable, pero a los pocos minutos de llegar al lugar, comenzó a llover torrencialmente. Me dije: « ¡Por fin, ésta sí que es la mía! »

Pero no fue así, a la media hora salió el sol y empezó a brotar la humedad y el calor. Levanté mi puesto, sin vender ejemplar alguno, con los reproches de mi socio amigo. Con toda la bronca, me fui a la playa y fue ahí que empecé a mirar con muchísima atención los pies de las mujeres, ya sea las maduras como las más jovencitas.

De cada diez que pasaban descalzas, seis tenían unos terribles juanetes y ahí se me cruzó la idea genial. Llegué a mi casa e imaginé cómo iban a ser mis zapatos, zapatillos, chatitas, tacos agujas «juaneteros» La preocupación era cómo esconder los juanetes y que por fuera, no sobresaliera ese sobrehueso deforme. 

Como todo se arregla con siliconas, empecé a pensar y pensar en mi invento juanetero, recubierto de mucha silicona. Al poco tiempo, todo estaba en marcha. Levanté mi puesto ambulante en plena peatonal San Martín, y ¿qué pasó? En el transcurso de dos horas tuve que mandar a mi socio a buscar más mercadería. Hacían cola, todos querían llevarse uno o dos pares de los zapatos juaneteros. ¡Se vendían como pan caliente!

¡Qué bien iba todo! Con este negocio pasaba al frente, iba a tener por fin, mi local propio. Y la gente seguía y seguía comprando.

Todo cambió un día, cuando me golpean fuerte el hombro diciéndome: «Enrique, despertate, que llegó tu socio «bueno para nada», quiere saber cuál es tu plan, tu nuevo invento»

Sí, otra vez,  todo fue un dulce y hermoso sueño, y como soñar no cuesta nada...yo seguí soñando. Aunque no me voy a desanimar. Esta vez mis «zapatos juaneteros» van hacerse realidad. ¡Y  diez veces mejores que los de Ricky Sarkany!

\section*{Un Kaníbal Vegetariano}
\addcontentsline{toc}{section}{Un Kaníbal Vegetariano (Por Gustavo)}

\begin{flushright}Por Gustavo\end{flushright}

Cuando escuchamos hablar de un caníbal sabemos que se trata de una persona que se alimenta o come a otro ser humano. En caso de no saberlo asistimos a un diccionario los más rústicos o a algún medio electrónico los más tecnificados, para que nos entregue su significado. La RAE es la alternativa más utilizada.

Por otro lado, es cada vez más común en nuestros días que mediante la amplia proliferación de las denominadas corrientes New Age o a causa de crecer  en una burbuja (alejadazos de la naturaleza) y, de pronto, ver cómo se mata a los seres humanos, las personas toman la decisión de ser vegetarianas. Estos motivos no intentan ser una lista exhaustiva.

Hasta aquí, podemos estar de acuerdo que caníbal y vegetariano expresan preferencias de alimentación distintas. Los más ortodoxos dirían que son gustos no compatibles o en términos de lógica matemática se es vegetariano si y sólo sí no se es caníbal. Un mordisco minúsculo de carne humana significa la traición e inmediata expulsión del «ser» vegetariano de su círculo.

Sin embargo, en la Argentina al día de la fecha encontramos un caso excepcional, absurdo para algunos o la excepción a la regla que viene a confirmar la teoría. Se trata del descubrimiento de un ser se estatura media, cuerpo venido a menos por el paso del tiempo, ojeras pronunciadas, barba tupida y bigotes desaconsejablemente largos.

Por otra parte, un grupo interdisciplinario compuesto de psicólogos, sociólogos y neurólogos determinó en el espécimen un mal humor permanente, causado por una isla intracerebral que se manifiesta en un lenguaje exasperante, sobre todo caracterizado por gritos que defenestran al interlocutor y chiflan al oyente. 

Por último, se descubrió luego de numerosos estudios y seguimientos que la dieta del referenciado se compone en un 50\% por frutas y verduras y, en la mitad restante de la práctica de alimentarse de miembros de la propia especie.

Se lo denominó científicamente Kannibale Fernandea o vulgarmente llamado Kaníbal Fernández.
\clearpage
\section*{Guiso de excusas}
\addcontentsline{toc}{section}{Guiso de excusas (Por Willy)}
                                                                                                    \begin{flushright}Por Willy\end{flushright}

Los tiempos cambian... y las personas también, ¿alguien lo duda? Pero... ¿de qué forma? ¿En su forma de vestir, de comer, de hablar? Eso es algo externo, que se ve. Pero vayamos a lo otro, a lo que no se ve: al comportamiento humano. Uno de esos cambios que en realidad no ha cambiado en el fondo pero sí en las formas, y en los tiempos, son las excusas. Veamos algunos ejemplos: carta de amor de hace 200 años. El hombre escribía algo así: «mi bella amada, la extraño un poco pues hace dos meses y medio que no la veo; sé que sueno un poco apurado pero necesito verla si es posible el mes que viene para seguir con el amor que enternece a mi pobre corazón»

Hoy en día, con la tecnología de internet y los celulares no podés pasar más de un par de horas sin comunicarte con tu pareja porque sino ya te engañaron... ¡y mal!, pasás a ser un alce de las praderas o un venado con cuernos bien grandes de los bosques más helados de la tundra siberiana.

Pero las excusas siguen al pie de nuestra humanidad, nos sirven para evadir la responsabilidad, para no reconocer una falla que hemos cometido y queremos buscar alguna salida a nuestro error, y la excusa es y será eso... nuestra inseparable excusa.

En la época de San Martín, los soldados volvían luego de dos o tres años de pelear en las guerras y sus mujeres, muchas veces, tenían algún amante que las hiciera olvidar por lo menos un rato la soledad. Cuando volvía su marido su excusa era: «me sentía tan sola que no sabía qué hacer» y la soledad la hizo aferrarse a lo que se le cruzara, y no quedaba muñeco en pie ni hombre que se pudiera resistir a los encantos de una mujer tan necesitada de amar y de ganarle a la soledad.

Hoy en pleno Siglo XXI, en la era digital y de la comunicación total vivimos cada vez más incomunicados, o bien la excusa es estar incomunicados para no comunicarnos con los demás. Por ejemplo, a un hombre le gusta una chica y la quiere llamar por celular, del otro lado no le contestan o no quieren contestarle, excusa: «no tenía el celular conmigo», dice ella.

Otro ejemplo es que nunca una mujer te va a llamar ni mandarte un mensaje, salvo que esté metida con vos hasta las orejas y ya haya perdido el sentido femenino de orgullo y seducción...que una mujer llame o mande mensajes o llame insistentemente significa que ya está perdida y no le importa ya  nada más, una presa que ya caído una y mil veces en la misma trampa que vos hiciste.

Que no te llame o no te mande mensajes conlleva la excusa: «no tenía crédito en el celular». En otras palabras, le interesás poco y nada. Consejo: andá buscando otra... ¡mujeres sobran!

Por eso, al día de hoy, las excusas han cambiado en sus modos pero no en sus causas. 

Pero para estudiar las causas habría que escribir un tratado de Psicología total de la excusa...y eso será en una próxima entrega.

\section*{Caníbal vegetariano}
\addcontentsline{toc}{section}{Caníbal vegetariano (Por Hernán)}

                                                                                        \begin{flushright}Por Hernán\end{flushright}

Caníbal: término usado en los principios de la humanidad para referirse a una especie que se come a sí misma. Se dice que los hombre llegaron a ser caníbales en esos años, aparte de ser homníboros, frugíboros, homínidos y tantas cosas más en el proceso de la evolución del cerebro humano. Éste último comenzó pesando 600 gramos. Hoy pesa 1.400 y tiene características de homosapiens, el cual existe hace aproximadamente 70.000 años, después de cinco generaciones: Astralopithecus Afarensis,  Astralopithecus Africanus,  Astralopithecus Robustus,  Homo Habilis, Homo Erectus, y Homo Sapiens

Del homo sapiens adquirimos la  capacidad de resolver problemas,  la capacidad  de imaginación y de  creatividad 

Sin embargo, esta especie que tanto ha avanzado, se volcó al lado vegetariano y le empezó a flaquear la carne. Así, se encontró con la verdura y las dietas y más recetas que vida.

De esta forma,  el hombre terminó  privándose de todo  y sobre todo las mujeres, que van a una cena y comen verduras y mirando el vacío que está  comiendo el marido o alguno de los que están en la mesa, se babean por dentro,  pero dicen «no», eso engorda. Y se lo devoran con la vista treinta veces, aunque con la boca ni una. Y el marido le dice: «.come un poquito, está rico » y ella le responde: «ya se que está rico,  pero hace mal » 

---¡Hace mal las tarlipes, qué te  va a hacer!

---No, pero me engorda.

---Si estás divina. Y  ella entre medio del deslumbramiento por la falta de no querer creerle, no sabe si... «Me  lo dice en serio o me lo dice para hacerme sentir bien» - piensa ella- «¿Será blanco o negro, o tal ves gris?»

Y sumergida en la confusión,  en milésima de segundos más rápido que la luz eléctrica, le pregunta:

---Vos me lo decís porque me querés,  pero en realidad engorda»

Y el marido se lo dice porque la quiere y le dice la verdad:

---Yo te quiero,  gorda o flaca te quiero y te elegiría  una y otra y otra vez. Y ahí ella le dice:

---Bueno dame un poquito.

Y ahí le producen endorfinas al cerebro y se siente bien y el marido la mira y le dice:

---Agarráte un pedazo.

Y ahí empiezan a comer los dos, después bailan a lo loco, y se van los dos contentos.

La lección: si te gusta algo, sacáte las  ganas,  es lo más lindo que hay,  en su medida. Pero te lo pido de corazón, sacáte las ganas porque es lo más lindo que hay.

Si algo te llena de placer hacélo, date ese gusto que tanto querés, en cualquier aspecto de la vida y sé feliz y disfrutá de la vida que es... ¡hermosa!


\section*{Fiestonga de 15 }
\addcontentsline{toc}{section}{Fiestonga de 15 (Por Juanjo)}

                                                                                        \begin{flushright}Por Juanjo\end{flushright}


Dos días atrás me llamó Carlos. No podría decir que Carlos es mi amigo. Es, digamos, un conocido. Conocido y gracias, como dicen algunos. Carlos me llamó a las once de la noche. ¿Querés venir a una fiesta de 15? Aburrido como estaba, no lo pensé dos veces. Me enfundé en el traje que había usado para el velorio de la abuela y salí hacía la dirección que me dió. Cuando llegué estaba oscuro. Toqué y me hicieron pasar. Entre catorce me dieron una golpiza de la que no me voy a poder olvidar aunque quiera. Voy a pedirle a Roque el teléfono del brujo.

\chapter*{Cuerpo y alma}
\addcontentsline{toc}{chapter}{Cuerpo y alma}

%Marilina
\emph{El humor, que tanta falta nos hace, vino a soltar la ocurrencia y nos mostró cómo la palabra vehiculiza  otros sentidos cuando somos capaces de desestructurar esa manera habitual de dar forma a los escritos.}

\emph{Sin embargo, el mes se desdobló y para la segunda parte nos esperaba otro tema, más atrapante y desafiante al mismo tiempo: el cuerpo y el alma. ¿Podrán reconciliarse ambos, desde nuestra particular percepción? Veremos...}

\section*{Presagios de cambio }
\addcontentsline{toc}{section}{Presagios de cambio (Por Gustavo)}

                                                                                        \begin{flushright}Por Gustavo\end{flushright}
                                                                                     
A continuación serán testigos de un diálogo entre el cuerpo del escritor y su alma. Acto seguido aparecen en escena el cuerpo y alma de un amigo del precedentemente señalado.

---Hola, hola aquí estoy, susurró el alma del escritor.

---Me molesta el oído, escucho un zumbido voy a tener que ir al otorrino, expresó el cuerpo del escritor.

---No que otorrino, soy yo tu compañera ¿acaso no me vez?

---Voy a tener que asistir al oftalmólogo, veo algo borroso.

---¿Pero no sentís mi presencia a tu alrededor?

---Tengo visión borrosa, escucho susurros, siento escalofríos, ahora estoy somatizando. Me voy a pedir un turno con la psicóloga que recomendaron, no me queda otra. ¡Esto me está asustando!

Luego de esta sucesión de hechos, el cuerpo y el alma del escritor se encuentran con el cuerpo y el alma de su amigo.

---Hola papá ¿cómo estás? Tanto tiempo sin vernos ¿qué es de tu vida?, expresó el cuerpo del escritor al de su amigo.

---Bien genial, te presento a mi esencia, mi cable a tierra, mi fiel compañera. Es más, te diría que somos como una sola cosa. Ella es mi alma, pronuncia el cuerpo del amigo.

---Encantado es un placer conocerla.

---El placer es mío pero algo me llama la atención, ¿Por qué no presentó a su acompañante, es muy bella, expresa el alma del amigo.

---¿Mi acompañante?

---Sí, esa que probablemente es tu vía de escape y relax diario.

---Ah, te referís a mi psicóloga, tendría que llamarla, pero me dio su número sólo para urgencias.

Interrumpe la conversación el cuerpo del amigo para tratar de orientar al cuerpo del escritor: 

---Campeón ¿cómo creés que estás en lo espiritual?

---Mirá, voy a misa todos los domingos, rezo tres padres nuestros y el Rosario completo todos los días.

---Pero ¿eso te conecta con tu interior, realmente?

---¿Con mi interior?  ¿Qué por las sesiones con la psicóloga me decís?  Un poco sí.

---No, me refiero a la misa, los rosarios y los padres nuestros.

---Ah, ¿qué decís? Eso te conecta con el barbudo, si justo de casualidad salís sorteado en la quiniela sagrada o tenés la suerte de Jym Carrey en el film «Todo Poderoso».

---Amigo junto a mi compañera nos tenemos que ir, expresó el cuerpo del amigo al del escritor. Te doy un consejo: cambiá la psicóloga por esa belleza.

---¿Qué belleza?

---Esa que seguramente te susurra, te acaricia, te envuelve, te hace creer que ves borroso. Que con sus reclamos de atención te asusta, pero sobre todo esa que te hace reír como un niño y mantiene un puñado de inocencia dentro de ti. Buena suerte amigo.

---Al fin alguien me registra, aunque sea por unos segundos, piensa por dentro el alma del escritor.


\section*{Mi cuerpo y mi alma}
\addcontentsline{toc}{section}{Mi cuerpo y mi alma (Por Ana)}

                                                                                        \begin{flushright}Por Ana\end{flushright}


Estoy en el silencio poblado de ecos de mi cuarto, dispuesta a descansar. Aún el sueño no acude; acuden los recuerdos que irrumpen en tropel, profanan mi descanso.

Me incorporo protestando para ahuyentarlos, cuando escucho voces dentro de mí. Me ganan el miedo y la incertidumbre. Pero agudizo todos mis sentidos para escuchar:

Cuerpo: ¡Siempre a flor de piel! Así me tienes todo el tiempo. Has hecho de la mujer que me habita, una llorona que sufre por todo y por todos.

Alma: ¡no, no  es así! Yo puse en ti, sentimientos. Tú les diste los colores. Yo puse en ti, sensaciones. Tú, las viviste y experimentaste.

Yo colmé tu pecho de amor y tu razón lo aceptó, y en tu corazón encontró nido. Aunque, hoy, se debate dolorido por no ser correspondido. Soy tu alma, te acuné con ilusiones desde que naciste; sembré sueños en tu campo fértil. No quise que sufrieras, pero… ¡es la vida!

Cuerpo: es cierto lo que me decís, me siento un desagradecido. Ahora comprendo lo que antes no supe ver ni oír. No sé, si no supe o terca no quise ver tus señales, cuando dudas y cavilaciones oscurecían mi cielo, tú me enviabas presentimientos certeros que, yo, estúpido ignoraba. Tú echaste mano de las premoniciones para mostrarme que me equivocaba. Tú pugnabas por rescatarme de la niebla. Pero, yo siempre regresaba. Hoy te agradezco que sigas en mí. Con un profundo suspiro me llego hasta ti. 

¡Gracias por tenerme siempre a flor de piel!

\section*{Preocupaciones habladas}
\addcontentsline{toc}{section}{Preocupaciones habladas (Por Marilina)}

                                                                                        \begin{flushright}Por Marilina\end{flushright}


La discusión nació una noche de verano, cuando las miserias salieron a la luz:

---Malditos mosquitos, son la plaga número uno en Santa Fe y yo no tengo por qué soportarlo, dijo Cuerpo, ancho de pata, en el patio de la casa, sin pensar que alguien podría retrucarle.

---No culpes a la naturaleza por la debilidad de tu carne, resonó una voz en eco, que provenía detrás del asador y con sonido acústico envolvente.

¡Flor de julepe, se llevo el envase aquél, después de oír semejante sermoneo, con entrada libre y gratuita!

---¡¿Quién anda ahí!? Si tanto te interesa mi problema, hablemos a calzón quitao!, gritó el dueño de casa, sin saber muy bien a quién se enfrentaba.

---No soy de hablar, me gusta manejarme con acciones. El lenguaje de los hombres me es, a veces, inentendible. Soy un alma, o si te interesan los parentescos, soy tu propia alma. Llámame Alma, a ti que te gusta medir todo por nombres, dimensiones, materia. Vendríamos a ser algo así como las dos caras de una misma moneda, y conste que te hablo en tu sintonía, de eso he aprendido mucho a tu lado.

---¡Vah, no me vengas con chamullos! Mi propio alma, ¿y quién te dio ese cargo acaso?

---Fui asignada el mismo día que vos, pero con funciones totalmente distintas, eso está a la vista.

---La verdad que no te conozco che ---dijo Cuerpo---. Ha corrido mucha agua bajo el puente, el mes que viene cumplo 79 pirulines y a esta altura, no sé si vale la pena enterarse que una mujer más fue asignada a mi lado. ¡ y yo ni enterao!

---En primer lugar, no soy mujer, no tengo sexo, soy incorpórea y no te ilusiones, que no estoy acá para cumplirte ningún deseo.

Cuerpo la miró extrañado. Por primera vez, una mujer o lo que fuera, lo había dejado sin palabras. Entonces, por lo bajo continuó preguntándole:

---No sos mujer ni me pretendés. Pero te llegás al patio de casa a retarme como mi mujer, que en paz descance la pobre, ¡¿cómo se explica eso?!

---Otra vez equivocás tus interpretaciones. Ni vine a retarte, ni gané tu patio. Aunque, te parezca irreal, yo vivo en vos, soy parte de vos.

---¡Ah bueno, qué vino de porquería me vendió el manolo éste en la carnicería hoy, me está trastornando el marote!

---A ver... tal vez fui apresurada en mi comentario, en esta aparición que te hice. Sé que he sido para vos una indiferencia a lo largo de estos años. En realidad, ya poco interesa que reconciliemos diferencias entre partes, cuando toda una vida no fue suficiente para ello. Lo que vos juzgarías como  reproches, yo lo mido en términos de aprendizaje. Eso que vos mencionás como reto, yo lo diseño como pequeñas señales de luz, cuyo objetivo es iluminar cada célula que te forma, para dar  más y mejor sentido a tu existencia.

---¡Chupáte esa mandarina!, dijo Cuerpo, al borde de la carcajada. ¡A vos también te vendieron bebida vencida!

---Para que te quedes más tranquilo, tenés que saber que hay cosas que no se explican por la razón. Yo soy una de ellas, por ejemplo. La clave consiste en una comprensión más emocional, que de otra índole. Sentir viejo, de eso se trata, o ¿para qué crees que viniste a este mundo? 

---¿Yo?... pa’ laburar como un perro, lidiar con las desgracias propias y ajenas y  cuando el bolsillo me lo permitía, panza llena y corazón contento. De amores mejor ni hablemos, siempre fui de hacer rancho aparte.

---Viviste a los tumbos entonces, querido Cuerpo. Te preocupaste tanto por apuntalar esa estructura y…mirá…la carrocería se te está cayendo a pedazos y qué va a quedar de ti.

---¿Quedar?, no pienso que quede mucho. ¡Ya tengo pagado el nicho y no pienso salir de ahí!

Alma tomo impulso, bostezó largamente y detuvo sus ojos en una azucena que permanecía abierta con la luz de la luna. Miro al cielo, sintió que su recorrido en ese lugar estaba llegando a su fin, pronto volvería a tomar vuelo. Alma sabía que una nueva misión en la tierra aguardaba para ella. Lo había comprobado esta noche, luego de conversar con un cuerpo ruin, mezquino, cegado por una vida mundana, que sabe atrapar a muchos y los hace prisionero de una esclavitud construida por comodidad.

---Me voy en vos, le dijo alma al despedirse, cuando su voz  volvía a adquirir esa sonoridad del principio. Olvídate que estuve, tal vez me recuerdes como un mal sueño o una copa de más con la que te excediste hoy. Sé que volveremos a tener otra charla y seguramente, seas vos el que entonces,  vengas a mí. Te estaré esperando.

Y de esa manera, cuerpo encontró de nuevo su anhelada calma. Cuerpo y su dueño, porque él también era un cuerpo alquilado, habitando un ser que le dio derechos y sobre el cuál él se tomó sus propias atribuciones.

Para modelo de conversación vaya este sentido homenaje. Aunque, pensándolo bien, la muestra sirva más como botón al ojal de un diálogo de sordos.

\chapter*{Despedida}
\addcontentsline{toc}{chapter}{Despedida}

\emph{Poquito a poco el 2012 quiere bajar su telón. Ya hablamos del tiempo y de la imposibilidad de ir en contra de él. Lo mejor será respetar «su tiempo» y aprovechar el lapso que nos ofrece, para tomar impulso y comenzar el nuevo año con ideas y palabras necesitadas de expresión.}

\emph{El brillo de la palabra se despide hasta el 2013, con la convicción de que en su recinto se han generado nuevas asociaciones, nuevas catarsis, nuevos vínculos afectivos, que la palabra coronó desde el principio.}

\emph{Para despedirnos del lector, le obsequiamos un bonus track de algunos fragmentos de textos con tonos humorísticos y otros más sagaces, que quedaron guardados en el baúl del Taller:}

\vspace{0.5cm}

\textbf{Chuchería:} familia del chucho, chocho, checha y cheche. Chochean todo el día, es decir, cha no trabajan, cha no estudian, cha no hacen nada productivo. (Willy)

\vspace{0.5cm}

\textbf{Calambre:} cala hambre. Vacío estomacal que nos indica que debemos ingerir alimentos, para evitar un colapso psico-emocional. Frase: «cala hambre che». (Gustavo)

\vspace{0.5cm}

«... ¡No me escuches, No me escuches, si te revienta lo que te digo!
Estaba tan cansado el hombre con todos los problemas que tenía y todavía tener que bancarse a la mujer, que le taladraba el cerebro en semejante medida. Algo innecesario para una relación, porque eso pasa cuando no se confía en el otro. El hombre abrió las ventanas y entró el sol y en ese momento terminó de descubrir lo que no había llegado a saber en el momento que se casó y ahora sabe... ¡Él no sabia que casaba con una bruja!» (Hernán)

\vspace{0.5cm}

\textbf{Comparsa de enanos}

Estábamos todos acongojados a moco tendido despidiendo a la abuela cuando se escuchó una sucesión de timbrazos muy cortitos. Dejé mi lugar junto al cajón y me llegué hasta la puerta. Seis enanos vestidos de colores me miraban desde medio metro más abajo. Estaban buscando un casamiento en el cual iban a presentarse como número sorpresa entre la mesa dulce y el carnaval carioca. Les dije que a la vuelta había un salón de fiestas. Estaba cerrando la puerta cuando uno puso su zapato de duende antes del marco. ¿Podría pasar al baño? Me encogí de hombros y volví a abrir. A todo esto, tías y vecinas no dejaban de ejercitar el cuello desde atrás del cajón para saber qué estaba sucediendo en la calle. (Juanjo)

\vspace{0.5cm}

\textbf{Ortiva:} mezcla de ortiga y vacío, es decir, se le dice a alguien ortiva cuando su vacío como persona no suma nada o bien resta y su personalidad es urticante como la ortiga y provoca escozor y rechazo. (Willy)

\vspace{0.5cm}

¿Para qué sirven... los agujeros del colador?: para quemarse la mano menos hábil. (Gustavo)

\end{document}
