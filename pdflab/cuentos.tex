\documentclass[11pt,twoside,openright,a6paper]{book}

\usepackage[a6paper]{geometry}

\usepackage{times}
\usepackage[Lenny]{fncychap}
%\usepackage[Conny]{fncychap}

\usepackage[spanish]{babel}
\usepackage[utf8]{inputenc}
\usepackage[T1]{fontenc}

\newcommand{\personaje}{Nombre del personaje}

\title{Leer con moderación}
\author{Juanjo Conti}
\date{}

\hyphenation{per-te-ne-cien-tes}
\hyphenation{ín-di-ce}
\hyphenation{co-ra-zón}
\hyphenation{via-je-ro}
\hyphenation{vi-kin-go}
\hyphenation{a-tra-ve-só}
\hyphenation{dio-ses}
\hyphenation{fe-rry}
\hyphenation{ca-mi-nan-do}
\hyphenation{e-le-va-do}
\hyphenation{ba-ta-lla}
\hyphenation{ber-me-llón}
\hyphenation{bo-te-llas}
\hyphenation{pro-ta-go-nis-ta}
\hyphenation{ar-chi-pié-la-go}
\hyphenation{li-bre-rí-a}

\begin{document}

\pagestyle{plain}

\maketitle

\tableofcontents

\chapter*{Prólogo}
\addcontentsline{toc}{chapter}{Prólogo}

Este libro es una escusa. Es una prueba de concepto sobre cómo imprimir libros tamaño A6. Los 3 cuentos incluídos fueron trabajados en el
taller El brillo de la palabra.

\chapter*{La convención}
\addcontentsline{toc}{chapter}{La convención}

El 22 de agosto de 2012 se llevó a cabo en la ciudad de Santa Fe una reunión muy particular. Esta tuvo lugar en el salón de convenciones del hotel Los Silos, ubicado en el puerto de la ciudad, junto al casino.

Si se lo miraba desde lo pisos superiores, el lugar era un campo de frutillas. Las caperuzas cubriendo las cabezas de todas las invitadas casi no dejaba ver el gris del mármol que pisaban.

Había caperucitas de alma y caperucitas de profesión. Había caperucitas quinceañeras y caperucitas que ya eran abuelas.
Estaba, por ejemplo, la reconocida actriz Lucía Vertucci, protagonista del éxito Caperucita y el motochorro. E incluso estaba Caperucita Rosa, cuyo nombre de bautismo era Carlos.

En total eran más de 200 personificaciones del legendario personaje de rizos dorados y sonrisa inocente.

El objetivo de la convención era dividirse las rutas para ir a visitar a las correspondientes abuelitas, ya que en una ciudad tan chica, muchas se encontraban cuando tenían que perderse y eso no era ninguna gracia en los cuentos. No sabían quien era la organizadora de semejante reunión, pero todas coincidían en que era una estupenda idea.

Cuando el maestro de ceremonia se disponía a hablar, la música bajó, las luces se atenuaron y todos hicieron silencio. El orador se aclaró la garganta tosiendo y su tos retumbó en toda la sala. Se acomodó las gafas sobre las puntiagudas orejas y tomó entre sus manos cuatro o cinco hojas de papel. No había empezado la primer oración de su discurso cuando caperucita, una, notó que las puertas se habían cerrado y que de la espalda del maestro de ceremonia asomaba una horrible, enorme y peluda cola de lobo.

\chapter*{Bermellón}
\addcontentsline{toc}{chapter}{Bermellón}

Entorné los ojos para enfocar y entender lo que estaba viendo. Dos puntos luminosos, uno arriba del otro. Luego el campo de visión se amplió y aparecieron unos números en el panorama. El reloj digital indicaba las dos y diez. Al costado, sobre la misma repisa, mis herramientas. Pinceles, lápices y la cuchilla con la que saco punta a esos lápices. Me desvestí de las sábanas usando las piernas y con un movimiento que a mi edad podría calificarse de ágil, dos segundos después, tenía los pies enfundados en las pantuflas de paño. Arrastré las suelas de goma por el atelier, tomé un pincel con la mano derecha y continué donde había dejado al caer rendido ante el ataque sorpresivo del sueño.

Mientras trabajo no puedo dejar de pensar en María. Ella duerme en la habitación, silenciosa. El camastro en el taller me permite trabajar durante la noche, tomando pequeñas siestas de media hora sin molestar a mi esposa. Cuando los primeros rayos de luz entran por la ventana, dejo todo y voy a dormir a su lado. Cuando me despierto al mediodía, ella ya está terminando alguna clase. Almorzamos juntos y vuelvo a trabajar.

Estoy convirtiendo una de las paredes del taller en un nuevo mural. Me gustan los murales. Huelen a inmensidad, a sin frontera. Para completar un mural uno tiene que dedicarle semanas, en oposición a un cuadro chico, tal vez una naturaleza muerta, que se puede completar en a lo sumo dos días. Gracias a esta cantidad de tiempo requerida por la obra es que se logra desarrollar una onda sensación de pertenencia. En ambos sentidos. En el más clásico, la obra te pertenece, puesto que la creaste. Pero en uno más metafísico, es la obra la que te empieza a poseer. Te pide más, dicta su desarrollo, expande sus límites.

El mural en el que estoy trabajando ahora se llama Revuelta o tal vez termine llamándose distinto. Muchas personas se han juntado en una plaza a manifestarse. Llevan carteles y pancartas. Insignias y lemas. Rostros y banderas. Yo mismo me veo en la revuelta. Soy uno más y a la vez soy todos. Pinto horas enteras sin descansar. El olor a pintura fresca me llena y me vacía. Inflo mis pulmones y soy irrigado. A mi alrededor, el taller. Trapos sucios, latas, botellas. Olor a aguarrás y resinas. Pinceles y paños. Luces y sombras. Colores y engaños. La fuerza creadora me eleva. Y para materializar la metáfora me subo a un andamio y pinto la parte superior del mural. Puntas de lanzas que rasguñan el cielo. Gritos y plegarias que ascienden. Y ahí, desde arriba, escucho al gato de la vecina en la ventana. Maulla y araña el vidrio como queriendo entrar. No, ahora no puedo. No molestes, estoy trabajando. Pinto, delineo, coloreo. Ropa sucia, pinturas y otras obras. Todo, escenario de la actual concreción. El gato sigue maullando y me interrumpe. No ahora, no. No te puedo dar de comer. Sigo pintando. Amarillos, bermellón. Negros, grises y marrones. Hay fuego. Multitudes. El pueblo grita, se exalta, canta. Y yo soy su voz. Tengo que pintar para que puedan gritar, exaltarse, cantar. Si no pinto no existen. El gato sigue molestando, ahora con más insistencia. Mezclo lo que queda de naranja con bermellón sobre la tapa de una lata. Cargo el pincel y sigo. No puedo detenerme. Ahora son estallidos. Columnas de fuego y humo circundan la escena. La parte derecha del mural explota en una batalla campal entre el orden y los que se manifiestan. Yo soy su arsenal, el que le carga las armas, el que fabrica sus balas. Sin mí no tienen con qué disparar y la batalla está perdida. Siguen los estallidos y las explosiones. Amarillo, naranja, bermellón. Sigo pintando. Y el gato de la vecina golpea el cristal con sus uñas. Y aprieto el pomo de bermellón y ya no queda. Lo exprimo, lo estrujo, lo estrangulo. No salen más que las últimas gotas. Pero el mural no está terminado. Me pide más, me interpela, me exige. El pueblo me grita, me necesita. Están perdiendo la batalla. El fuego también me reclama. Y el gato vuelve a maullar. Y por primera vez lo miro. Lo miro a los ojos. Desde el andamio. Dos, tres metros elevado sobre el atelier. María duerme. Estiro el brazo y muevo el barral que abre la ventana. Y el gato entra. Corre. Entra corriendo y se para junto al platito que le hace a veces de comedor. Me bajo fatigado. Malhumorado. Quería seguir pintando, no ser interrumpido. El gato me mira, confiando que como siempre voy a abrir la bolsa de alimento balanceado. Sirvo una porción en el platito y lo dejo comer un rato. La cuchilla está al alcance de la mano y con un movimiento que a mi edad podría calificarse de ágil, dos segundos después, le separo la cabeza del cuerpo. Dejo la cabeza comiendo del platito y me llevo el resto arrastrado por la cola, chorreando gotas bermellón.

\chapter*{La caja}
\addcontentsline{toc}{chapter}{La caja}

La primera vez que vi la caja fue en Suecia, en la ciudad de Gotemburgo. Era domingo y agujas de agua caían infinitas sobre la metrópolis. Empezaba a descubrir que la lluvia era una constante en aquella ciudad. Estaba encerrado en lo que por esos días me servía de morada. Un caserón de madera pintada de un color entre gris y celeste ubicada en un barrio impronunciable sobre una calle de similar dicción. Yo ojeaba a desgano una edición completa y ampliada de \emph{La guía del autoestopista espacial} que había conseguido por unas pocas coronas en una librería de libros usados en el centro. Si, en Suecia también hay librerías de libros usados. ¿Quién lo diría?

El sistema de transporte público en esa ciudad es genial.  A diez metros de la puerta de calle me tomé un \emph{bus} al centro. Con el mismo boleto, en el centro, me tomé un tranvía hasta el puerto. Y en el puerto, todavía sin volver a pagar, me tomé un \emph{ferry} que recorría las islas del archipiélago.

Me bajé en la última de las islas. Algunas personas bajaban sus bicicletas. Caminé por un sendero que se internaba en un bosque en el corazón del islote. La calzada estaba cada vez más desdibujada y los árboles a sus costados cada vez más frondosos. Las sombras entraban en la luz y, en cierto momento, me encontré en la oscuridad total. Apreté con las manos las tiras de mi mochila para darme seguridad y seguí caminando. No podía sacarme la sensación de que alguien me observaba. De repente, el camino se abrió en un claro de luz. Cuando estuve a campo abierto, no sabía muy bien dónde me encontraba, no veía el puerto y unas nubes de plomo se apoderaban del cielo.

Un lugareño apareció detrás mío por el mismo sendero. Me dijo unas palabras en sueco y, ante mi perplejidad, intentó en inglés. El hombre vivía en la lista desde hacía 40 años aunque, una vez a la semana, iba a trabajar a la ciudad. Consultor de algún tipo, creo recordar que le entendí. Me invitó a tomar café en su cabaña. Miré a las nubes en el cielo, cada vez más amenazantes. Miré la hora en mi reloj; el último \emph{ferry} de vuelta pasaba en dos horas. Cierto mantra de la niñez que versaba sobre desconocidos atravesó el tiempo como una flecha y me resonó en los oídos. Me encogí de hombros y acepté la invitación.

La cabaña estaba hecha de troncos de pino y, aunque acogedora a la vista, dudaba qué tan efectiva sería para resistir a los crudos inviernos que azotaban a esa zona, tan cercana al círculo polar.

George, que así era como se llamaba el hombre, encendió la hornalla de una cocina con un fósforo y lo sopló con pesadez antes de tirarlo a la basura. Unos minutos más tarde estaba tomando el mejor café que tomé en mi vida. Llevábamos charlando varios minutos sobre el acontecimiento del mes preferido de todos los habitantes de aquel país, la boda real, cuando a mi interlocutor, a quién se lo notaba contentísimo de poder practicar su inglés, se le ocurrió mostrarme algo. Lo entendí más por el dedo índice que levantó a la vez que abría exageradamente la boca que por sus esfuerzos en la lengua de Shakespeare. Menos de una centena de segundos más tarde, estaba de vuelta con una cajita metálica.

Depositó el artilugio frente a mi, sobre la mesita ratona que nos acompañaba. George me observaba divertido. Pude observar distintos glifos que la atravesaban. Los reconocí como pertenecientes al alfabeto rúnico, sin poder descifrar una sola palabra.“¿Qué es?”, pregunté.

El hombre se puso de pie y recitó una poesía en el idioma de sus antepasados. Le pedí que me lo traduzca y, haciendo un gran esfuerzo, me contó una historia sobre dioses y vikingos, objetos mágicos y barcos que volaban, una doncella y una competencia por su amor.

Loki, me contó, el más astuto de los dioses, había puesto sus ojos sobre una muchacha que estaba a punto de casarse con uno de los más bravos vikingos de cierto poblado. La deidad se le aparecía tomando distintas formas y le hacía maravillosos regalos a fin de caerle en gracia. Cuando su prometido se enteró de esto entró en cólera y exigió la presencia de Loki, el dios de las travesuras. Este, le ofreció una forma de dirimir quién se quedaría con el corazón de la señorita; una carrera en barco hasta cierta isla: el primero en alcanzar la costa sería el vencedor. El vikingo aceptó sin vacilar y al día siguiente se llevó a cabo la competencia.

Loki, que también era conocido como el viajero del cielo, se presentó con un barco que en lugar de navegar por el mar, surcaba el aire. La disputa dio comienzo tras el sonido de un gran cuerno soplado por la propia muchacha. El vikingo empezó a remar con todas sus fuerzas, transpirando y jadeando, mientras que Loki iba muy tranquilo en su embarcación mágica, esperándolo cada vez que le sacaba un buen trecho para poder propinarle una burlona carcajada. La pugna estaba por llegar al final y se veía que el vikingo no podría llegar antes que Loki. Fue entonces que tomó su hacha, se cortó una mano y, con todas las fuerzas que le quedaban, la arrojó hasta la orilla.

Loki, que nunca acepta una derrota, enfurecido, mató a la doncella y puso la mano en un cofre de metal. Se lo entregó al vikingo y lo condenó a vivir por siempre.

Miré mi reloj. El último \emph{ferry} salía en cinco minutos. Me incorporé abruptamente. George me acompañó a la puerta y le extendí la mano para saludarlo y agradecerle la historia. Recién en ese momento lo noté. El frío de esa prótesis ortopédica me heló la sangre.

\end{document}
