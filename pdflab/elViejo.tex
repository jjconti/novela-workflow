Dedos de luz y polvo atravesaron el sudario escarlata que formaban las nubes sobre la ciudad. El día llegaba a su fin y un hombre cansado y arrugado estaba sentado en el suelo, con la espalda apoyada sobre los muros de la catedral Nuestra Señora del Carmen. Los pies desnudos y sucios, una manta harapienta como único vestido,  y cabellos blancos y desordenados pintaban al personaje. Tenía la cabeza caída, como si se quisiera ocultar entre sus piernas. Los pies de otro hombre se detuvieron a su lado. Este, a diferencia, lucía un traje de tres piezas de hilo color hueso y llevaba en su mano un bastón de marfil. De la solapa del saco colgaba un prendedor plateado con forma de ángel.

---Mi buen amigo. Hace siglos que no lo veía. ¿Cómo se encuentra usted?

En anciano seguía con la cabeza gacha y dió la impresión de que no reparó en la presencia del misterioso caballero.

---Ahora me dedico a los recursos humanos. Tome, le dejo una tarjeta por si se decide a trabajar.


Horas más tarde pasó por el mismo lugar una doncella. Vestía una falda larga y rosada y una blusa celeste. Al verlo tirado lo reconoció. Con delicadeza le tomó el rostro entre las manos pero este no la miraba. Tenía la vista perdida, como mirando para adentro.
La noche ya se había posicionado y la luna derramaba un resplandor plateado sobre toda la escena. Sacó un pañuelo y le limpió la roña de entre las arrugas. Mientras lo hacía, el viejo se quedó dormido.


El reclamo de unos gorriones que le pedían alimento a su madre lo despertó. Entre trapos sucios y diarios viejos abrió los ojos al nuevo día. No recordaba nada de la noche anterior, pero se sentía mejor por primera vez en semanas. Cuando, con dificultad, se puso de pie, encontró la tarjeta entre sus ropas; tenía un ángel estampado y una dirección.

Caminó por la avenida principal y esquivando carrozas mecánicas llegó a la fuente donde bebía cada mañana. Un niño de cutis mohoso sostenía una jarrón en sus hombres y del jarrón se vertía un chorro de agua cristalina que iba a parar a un piletón. Se apoyó en el borde y estiró el pescuezo. Para completar el desayuno real, sacó un mendrugo del bolsillo y lo engulló mientras siguió caminando a su destino.

El edificio se levantaba en la esquina en la que se encontraban la avenida principal y una calle con nombre de santo. Una proeza arquitectónica que, para unos ojos cansados como los suyos, parecía llegar hasta el mismísimo cielo. Un muchacho terminaba de limpiar los cristales de la puerta de entrada y secaba su herramienta en el pantalón cuando el viejo entró. Un guardia de las dimensiones de un ropero modesto lo esperaba con una sonrisa tan falsa como el arma que colgaba de su cintura. El anciano no atinó a más que extender la mano y entregarle la tarjeta. El centinela la miró y, sin hacer comentario alguno, caminó a su puesto para hacer una llamada.

---Suba hasta el piso 7. Lo espera el Sr. Corelli.

Se metió en la cápsula de metal y oprimió el botón que le habían indicado.

Corelli era un hombre de unos 50 años. Vestía un elegante traje de hilo a rayas negras y grises y estaba peinado con gomina, como los compadritos de antaño. Lucía unos delgados trazos negros a modo de bigotillo con los que, al parecer, intentaba conferirle una estética afrancesada a su tez oscura.

---Estimadísimo Señor. Estaba esperando su visita. El patrón me dijo que era muy probable que se de usted hoy una vuelta por las oficinas. El negocio crece rápido, pero contar con alguien con vuestra trayectoria en las filas de La Legión sería más que un golpe certero a nuestros competidores. Ni que hablar de lo que significaría para La Legión como marca. Y que sea yo el encargado de entrevistarlo... el bono... el ascenso... Pero no nos adelantemos. Tome asiento por favor y comencemos con la entrevista.

El viejo, que no alcanzaba a entender las palabras de su interlocutor, aceptó la silla que le ofrecía, más por cansancio que por otra cosa.

---Aquí mismo tengo su hoja de vida. Por supuesto que tratándose de usted no necesito revisarla, es tan solo un formalismo. Veamos... aquí dice que tiene basta experiencia en medicina, leyes, milicia y docencia. Todo un cuadro, ¿eh? Hay varios puestos vacantes donde alguien con sus... habilidades, podría ser muy bien aprovechado.


El viejo salió del edificio con un cheque en una mano y una carpeta en la otra. El muchacho que limpiaba los vidrios seguía entretenido en el mismo metro cuadrado de cristal con el que estaba cuando llegó. Se sentó en el banco de una plaza cercana y empezó a leer mientras daba voraces mordiscos a una manzana que un verdulero le había regalado cuando lo vio pasar. Instrucciones:

\begin{enumerate}
\item{Ir a la barbería San Marcos y preguntar por el dueño. Pedir un corte y lavado completo.}
\item{Visitar la sastrería San Lucas y adquirir un traje oscuro y un par de zapatos italianos. Poner todo en la cuenta de La Legión.}
\item{Llamar a la puerta del jardín maternal Los Querubines y preguntar por Magdalena.}
\end{enumerate}

Irreconocible luego de completar los items 1 y 2, el antes mendigo parecía un músico de jazz norteamericano. No estaba en las instrucciones, pero se tomó la libertad de agregar a la indumentaria un par de lentes negros tipo ranger. Buscó en la guía la dirección del establecimiento y, revestido de la confianza que le confería su nuevo atuendo, se fue caminando a completar el tercer punto.

La puerta de Los Querubines era una estructura maciza de roble blanco adornada con la figura de un niño alado. El timbre estaba en el ombligo de la escultura. Toco y esperó.

Una mujer de unos cincuenta años apareció tras el umbral. Lo miraba fijo con sus ojos celestes y tras una cortina de arrugas que surcaban su rostro todavía se podía reconocer la belleza que alguna vez supo portar. Sin mediar palabra le hizo un gesto para que la siguiera. Pasaron un salón lleno de bebés jugando, para luego bajar por una escalera de madera. Entraron en una habitación pequeña, en donde apenas cabían los dos y se sentaron frente a frente a la luz de una vela. Magdalena lo miraba como si lo conociera de toda la vida.

---No me recuerdas, ¿verdad?

El viejo abría los ojos y mostraba las palmas de las manos como disculpándose.

---No importa, es todo diferente ahora. Lo que fue ya no existe y lo que podía ser desapareció como una duna en el desierto. Que tu también trabajes ahora para el patrón es un signo de que no quedan esperanzas. Toma este sobre, en el interior está tu encargo.

Siguieron contemplándose por espacio de varios minutos más. Cuando la mujer, con un gesto, indicó que la visita llegaba a su fin, ya eran varios los racimos de cera negra que se acumulaban a los pies de la vela.

Volvió a la plaza donde había leído el contenido de la carpeta y, asegurándose de que no haya nadie cerca, abrió el sobre. Dentro no encontró más que una fotografía. En el retrato se veía a una muchacha de mirada triste. Lucía una falda rosada y una blusa celeste. Detrás figuraban una dirección y un horario. El horario era 30 minutos más tarde y la dirección 40 cuadras al norte.

Sin saber muy bien por qué, empezó a correr. Corrió, saltó, se trepó, se tropezó, se incorporó y siguió corriendo.

Cuando habían pasado 29 minutos y estaba a una cuadra del lugar, alcanzó a divisar a la muchacha que forcejeaba con 3 hombres vestidos totalmente de negro. Inmovilizado por la situación, vio el reflejo de luz del cuchillo que uno de ellos acababa de desenfundar. Todo estaba más claro ahora. La Legión lo había enviado a salvarla. No sabía por qué ni sabía como hacerlo, pero esa era su misión. Corrió hacia el grupo gritando y agitando los brazos en el aire y, cuando estuvo ahí, no tuvo mejor idea que tirar todo el peso de su cuerpo sobre el que sostenía el cuchillo. Antes de caer, el desconocido logró asestarle la hoja metálica en el vientre. Los otros dos escaparon y cuando el viejo se incorporó, encontró su costado lleno de sangre. Cayó de rodillas y miró a la muchacha a los ojos. En los ojos había la misma o más tristeza que en la fotografía. El tercer hombre también huyó pero en el lugar dónde había caído el viejo alcanzó a ver un prendedor con forma de ángel.
