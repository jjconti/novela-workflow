\chapter*{Prólogo}
\addcontentsline{toc}{chapter}{Prólogo}

Esta historia me la contó el abuelo de un amigo...

\chapter*{Mescola}
\addcontentsline{toc}{chapter}{Mescola}

El tugurio es un bar sucio de mesas viejas en un primer piso sobre una calle poco transitada. Al bar no asiste ningún cliente ocasional porque
la puerta está cerrada para los desconocidos. El cantinero, que es también mozo y regente, mira a través del cristal de un vaso
liso que seca con su repasador. En una mesa el patrón escucha las noticias que el desgarbandazo esbirro le trae.

---El trabajo está hecho.

Don Mescola asiente casi imperceptiblemente con la cabeza, empuja la mesa de madera con las manos, haciendo rechinar sus patas contra el piso de cerámicas arábigas, y sale del lugar. No sabe si la noticia de su lacayo llega a tiempo o si, por el contrario, el perro faldero volvía a ladrar tarde. Con su chalina y sombrero baja por las escaleras y sale a la calle. Tiene el cuerpo de un oso, pero ha aprendido a moverse con agilidad en la vida. En la puerta del edificio lo espera su chofer frente al coche. No necesita darle indicaciones. Toman la Avenida del Sur y se dirigen a la clínica San José, en el centro de la ciudad.

Durante el trayecto mira a través de los cristales del Audi que conduce Roque. La ciudad ha cambiado desde que llegó en un colectivo desvencijado, con un bolso más vacío que lleno, a la estación General Manuel Belgrano. El calor del verano le dio la bienvenida ni bien bajó del armatoste. Caminó una cuadra y se encontró con una plaza que ocupaba toda una manzana. En el medio de la plaza había una fuente de la que emanaba agua fresca a borbotones. El jovencito no tuvo que hacer muchas cuentas para saltar dentro y apagar los incendios que le estaban agrietando la piel. La diversión le duró poco. Cuando escuchó el ruido del silvato policial y vio al uniformado acercarse desde la esquina blandiendo un bastón, tomó sus cosas y se mandó a mudar. 

%Mescola estaba undido en sus recuerdos cuando el Roque detuvo el coche.
Roque detiene el coche y Mescola, que está undido en sus recuerdos y en el asiento trasero, se despega del cuero dejando un halo de transpiración. Si, la ciudad ha cambiado, pero el calor es el mismo.

Camina hasta la habitación 101. No necesita que ninguna enfermera lo guíe porque se ha pasado yendo y viniendo por esos pasillos toda la semana. Dentro está su esposa casi dormida en una silla. Cuando lo escucha entrar, reacciona asustada y mira enseguida hacia la cama. Allí, en el centro de la habitación, entubada y conectada a una máquina, esta Marina, su hija. Parece una versión moderna de un Cristo en la cruz. Un Cristo mujer. Los pitidos constantes de la máquina indican que aún vivía. El perro ladró a tiempo. 

---¿Lo encontraron? ---preguntó la mujer, ansiosa.

\chapter*{Marina}
\addcontentsline{toc}{chapter}{Marina}

Marina trabaja, o trabajaba, o colaboraba (diría su padre, ya que no cobra, o cobraba, ningún dinero por ese trabajo) en la pastoral carcelaria visitando reclusos. Todas las tardes viaja, o viajaba, en colectivo los kilómetros que separan su ciudad del pueblo donde 
%la penitenciaría está emplazada
se emplaza la penitenciaría y se queda, o se quedaba, hasta entrada la noche. Vuelve, o volvía, en el colectivo de las 21:15, el último.

Siete días atrás se encontraba ayudando a uno de los reos a escribir una carta a su madre cuando este sin más le pidió perdón. 

---¿Perdón por qué? ---dijo Marina, casi con una sonrisa en los labios, a la vez que veía como el hombre, transformado, sacaba rápidamente una faca de su pantalon y le ponía el objeto filoso en el cuello. Las lágrimas de Marina caían sobre la improvisada cuchilla, resbalando hasta las manos de su captor. A los gritos, el desesperado hombre empezó a llamar al carcelero del pasillo. 

---Me dejan salir o la mato.

Segundos después había diez policías del lado de afuera de la celda. 

---Bajá el cuchillo, no seas pelotudo.

---La mato, te juro que la mato.

Los gritos en ambos sentidos llenaban el pabellón. Los otros reclusos empezaron a golpear sus rejas, conformando la música de fondo para los coros que lanzaban los oficiales y para el solo de garganta de los reclamos del hombre que blandía el arma blanca.

Cuando el cuchillero apartó la vista de su rehén, intentando distingir las palabras que le llegaban de los aullidos de los demás presidiarios, un policía disparó repetidas veces. El preso cayó muerto pero a Marina la habían alcanzado dos balas.

\chapter*{Lechuga}
\addcontentsline{toc}{chapter}{Lechuga}

A 46 kilómetros de la ciudad, donde se levanta el establecimiento penitenciario Santa María, la banda de Lechuga espera la puesta en libertad de su líder. El calor se hace sentir y los cuerpos transpirados de los hombres brillan contra el opaco paisaje. Lechuga estuvo guardado cuatro años por un error de cálculo en cierto trabajo. El error de cálculo consistió en que no terminó de hacer lo que tenía que hacer antes de que la policía llegue. A las 12 en punto del mediodía el reo volverá a caminar entre los hombres libres. Es lo que todos esperan, pero no sucede así. Son las 11:59 y las manecillas de todos los relojes están congeladas. El sol se queda donde está y las nubes detienen su viaje. Los guardiacárceles se niegan a soltar a Lechuga pero antes de devolverlo a su celda, le dejan hacer una llamada telefónica.

---¿Me podés decir qué carajo está pasando, Cartucho?

---No tengo idea, jefe. Está todo muy extraño acá afuera. Es como si hubiesen detenido al Tiempo.

---Eso no puede ser. El Tiempo es el tipo más astuto que hay. El Tiempo es escurridizo. Cuando lo estás por atrapar, el Tiempo se te escapa entre las manos.

---Vos quedáte tranquilo, Lechuga. Nosotros vamos a averiguar qué está pasando.

---Escucháme bien, Cartucho. Necesito salir hoy sí o sí.

\chapter*{Sr. Tiempo}
\addcontentsline{toc}{chapter}{Sr. Tiempo}

El Sr. Tiempo tiene aspecto de hombre, aparenta unos cuarenta años, tiene músculos desarrollados y luce un bigote negro y ancho bajo la nariz.

Siempre viste igual. A su atuendo lo componen un pantalon de jean color marrón, zapatos negros bien lustrados, una camiseta blanca sin mangas, casi transparente, una bufanda y un sombrero al estilo de los compradritos de antaño.

La noticia no tarda en llegar a oídos de Cartucho. En la ciudad el clima está enrarecido por la falta del Tiempo. Aparentemente los hombres de Don Mescola le hicieron una cama y el Tiempo cayó como el mejor. No se sabe cómo lo han logrado, pero se asegura que han recibido ayuda externa. La cuestión es que el Tiempo, o Sr. Tiempo, como lo llamaban quienes tienen el gusto, está detenido en un sector especial de la comisaría primera de la ciudad. Según comentan, en una fosa de tres metros de profundidad y con oficiales haciendo guardia a su alrededor. La orden es tenerlo allí, si vale la expresión, por tiempo indeterminado.

\chapter*{El plan}
\addcontentsline{toc}{chapter}{El plan}

La banda de Lechuga está reunida en un bar planeando el atraco y posterior liberación del Tiempo. La moza que los atiende está bastante malhumorada. Hace tres horas que se tendría que haberse ido a su casa, pero como el Tiempo está detenido, su jefe no la deja marcharse hasta completar el turno.

---¿Qué pudiste averiguar, Sota? ---habla Cartucho, que parece haber tomado el rol de jefe, en ausencia de Lechuga.

---El rumor parece cierto. Aparentemente la orden vino de arriba. Alguien importante quiere ver al Tiempo detenido por un buen rato y se encargó de pagar muy bien su capricho. No va a haber juicio, ni nada. Derechito al calabozo.

---¿Algún dato de interés sobre los oficiales que lo custodian?

---El cabo Rodríguez se casa mañana. ¿Sirve?

---Sirve. Sabueso, vamos a necesitar algunas cosas para el operativo. Anotá. Una camioneta, cotillón, dos bailarinas hawaianas, una comparsa...

---¿Con con bombo y platillos?

---Si, con bombo y plantillo... un barman con tragos para veinte personas,  sanguchitos de miga (de jamón y queso) y un mono. ¿En cuánto podés tenerlo listo?

---Dos horas.

---Ok. Como los relojes no funcionan vamos a hacer lo siguiente. El abuelo tarda diez  minutos en dar una vuelta a la manzana. Se va a poner a caminar y a contar las vueltas diciéndonos el número por radio. Cuando llegue al número doce, nos encontramos acá.

\chapter*{El atraco}
\addcontentsline{toc}{chapter}{El atraco}

El abuelo completa las vueltas y canta el doce. Todos los miembros de la banda de Lechuga se llegan hasta el bar, se suben a la camioneta y enfilan para la comisaría.

Llegan. Cartucho se encarga de anunciar la visita. Toca el timbre y espera. Una mirilla en la puerta de metal se corre para dejar ver el par de ojos de uno de los oficiales. Cartucho ya se había puesto una guirnalda como collar. Tiene un trago largo en una mano que equilibra con el sanguchito de la otra.

---Hola, venimos para hacerle la despedidaa de soltero del cabo Rodríguez. Tenemos todo.

El aluvión de personas toma por sorpresa al ocasional portero y todos entran velozmente a la comisaría: Cartucho, Sota, el abuelo, Sabueso, el del bombo, el de los platillos, las bailarinas, el mono y algunas personas que pasan por ahí y les apetece el jolgorio.

\chapter*{El final}
\addcontentsline{toc}{chapter}{El final}

Entonces acontece lo más fantástico de esta historia. Mientras todos se están divirtiendo, comiendo, tomando y bailando, en la profundidad de la fosa algo mágico sucede. El Sr. Tiempo siente un fuerte dolor en la espalda y se tira el suelo, estremeciéndose de dolor. Cuando el martirio cesa, se pone de pie. Le crecieron un par de alas de blancas en la espalda. Solo con aletear un par de veces logra elevarse por sobre la boca de lo que era su prisión y ver desde arriba a sus captores. Sucede que el tiempo vuela cuando uno se divierte.

Al verlo aletear e irse, los miembros de la banda de Lechuga brindan por el éxito de su operativo y se van raudos en la camioneta a encontrarse con su jefe. 

El cabo Rodríguez tiene que aceptar la realidad de que en menos de veinticuatro horas se casará pero de todas formas sigue abrazado a una bailarina hawaiana. 

En una clínica privada, un jefe de la mafia llora amargamente mientras pide la cabeza del jefe de policías. Comprende que lo que el Tiempo quita, el dinero no lo puede comprar.

Lechuga está afuera, pide las llaves de la camioneta y sale disparado. No ve la hora de reencontrarse con Marina Mescola, su enamorada, pero esta vez, afuera de la carcel.

\chapter*{Comentarios?}

\cleardoublepage
