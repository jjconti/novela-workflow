\documentclass[11pt,twoside,openright,a6paper]{book}

\usepackage[a6paper]{geometry}

\usepackage{times}
\usepackage[Lenny]{fncychap}
%\usepackage[Conny]{fncychap}

\usepackage[spanish]{babel}
\usepackage[utf8]{inputenc}
\usepackage[T1]{fontenc}

\usepackage[titles]{tocloft}


\newcommand{\personaje}{Nombre del personaje}

\usepackage{mathptmx}
\usepackage{etoolbox}


\renewcommand{\cftchapleader}{\cftdotfill{\cftdotsep}}

% change the space before the titles
\makeatletter
\patchcmd{\@makechapterhead}{\vspace*{50\p@}}{\vspace*{0pt}}{}{}
\patchcmd{\@makeschapterhead}{\vspace*{50\p@}}{\vspace*{0pt}}{}{}
\makeatother

% change the space after the titles
\renewcommand{\DOTI}[1]{%
    \raggedright
    \CTV\FmTi{#1}\par\nobreak
    \vskip 10pt}% original: 40pt
\renewcommand{\DOTIS}[1]{%
    \raggedright
    \CTV\FmTi{#1}\par\nobreak
    \vskip 10pt}% original: 40pt

\title{Microcuentos 1}
\author{Juanjo Conti}
\date{}

\hyphenation{co-me-mos}
\hyphenation{co-lo-ra-dos}

\begin{document}

\pagestyle{plain}

\maketitle

\tableofcontents

\chapter*{Justificación de los cuentos cortos}
\addcontentsline{toc}{chapter}{Justificación de los cuentos cortos}

Muchos de los cuentos que escribo son cortos. Esta cuestión anatómica de la literatura tiene una gran ventaja: si el cuento es bueno, en pocas líneas puede sacarte una sonrisa o provocarte un pensamiento profundo. Si el cuento es malo, no importa, en pocos segundos más termina.

Esta teoría, al igual que los cuentos cortos, goza de esta ventaja.


\part*{Sueños}
\addcontentsline{toc}{part}{Sueños}

\chapter*{Los eternos}
\addcontentsline{toc}{chapter}{Los eternos}

La primera vez que nos tocamos fue hace más de quinientos años en una playa en San Salvador, aunque ella llamaba al lugar Guanahaní. Se ocultó detrás de unas rocas sobre la arena cuando vio que una multitud de seres extraños salíamos del agua gritan y empujándonos. Los temores del viaje, las enfermedades, el hambre y el recuerdo de los que habíamos dejado en el mar fueron un catalizador que entró en acción cuando uno de los nosotros gritó “Tierra”.

\chapter*{El hombre que soñó con su gato\footnote{Publicado originalmente en mi blog}}
\addcontentsline{toc}{chapter}{El hombre que soñó con su gato}

Un hombre terminó de cenar, lavó los platos y sacó su gato al patio, para luego irse a dormir.

Mientras dormía, soñó que su gato lloraba en su puerta, se levantó y lo dejó entrar, para luego volver a dormir.

Mientras soñaba que dormía, soñó que su gato lloraba en su puerta, se levantó y lo dejó entrar, para luego volver a dormir.

Mientras soñaba que soñaba que dormía, soñó que su gato lloraba en su puerta, se levantó y lo dejó entrar, para luego volver a dormir.

Mientras soñaba que soñaba que soñaba que dormía, soñó que su gato lloraba en su puerta, se levantó y lo dejó entrar, para luego volver a dormir.

La secuencia se repitió 100 veces durante la noche.

Cuando se despertó a la mañana siguiente, casi se desmaya cuando vio lo que había en la cocina.

\chapter*{Se despertó}
\addcontentsline{toc}{chapter}{Se despertó}

Se despertó en el medio de la noche abrazado a su esposa. Se puso las pantuflas y camino hasta el baño. Cuando regresó, se vio durmiendo abrazado a su esposa. Por no despertarse, se fue a dormir al sofá.

\chapter*{Fantasmas}
\addcontentsline{toc}{chapter}{Fantasmas}

Unos niños fantasmas me señalaron y gritaron "¡un fantasma!".

\chapter*{La invención}
\addcontentsline{toc}{chapter}{La invención}


Tenemos el agrado de presentar esta invención la cual permite, ante un atentado o tragedia a gran escala, viajar en el tiempo para impedirla. Lamentablemente, desde su creación, no tuvimos ocasión de utilizarla.

\part*{Calle}
\addcontentsline{toc}{part}{Calle}

\chapter*{La nenita}
\addcontentsline{toc}{chapter}{La nenita}

Nueve de la noche, verano. En algún barrio de la ciudad.
Probablemente no en el centro ni en la costanera, sino en un barrio barrio. 
Una nenita de unos diez años avanza con una botella entre los brazos. 
Unos metros atrás, el kiosco de turno. 
La nenita camina, haciendo fuerza para que no se le caiga el encargue, más pesado que las muñecas con las que a veces juega. 
La nenita atraviesa la cuadra, con el porrón de cerveza entre las manos, recado de su papá, que la espera para apagar en su garganta el calor santafesino.

\chapter*{El pibe del semáforo}
\addcontentsline{toc}{chapter}{El pibe del semáforo}

Una cola de casi una cuadra ante el semáforo. Se acerca un pibe y me limpia el vidrio sin que se lo pida, queda más sucio que antes, se lo digo y se ofende. Ya tenía la moneda en la mano así que se la doy igual.

El gigante luminoso se pone verde, avanzamos despacio, casi a los empujones. No hay lugar. Cuando estoy a punto de cruzar se vuelve a poner en rojo.

El limpiavidrios vuelve a aparecer escondido atrás de una sonrisa de sin dientes y esta vez lo deja limpio.
Me vuelve a cobrar.

\chapter*{Libros marcados}
\addcontentsline{toc}{chapter}{Libros marcados}

La casa de la señora estaba llena de libros de su difunto marido con marcas de las tazas que les apoyaba encima. Para ella la literatura era menos importante que los muebles.

\chapter*{Cocción de un huevo}
\addcontentsline{toc}{chapter}{Cocción de un huevo}

Escapé una noche bajo el manto sepulcral que forman las nubes antes de que una tormenta se desate. Descalzo, corrí kilómetros hasta que llegué a una casa de campo. A unos metros, pude encontrar un gallinero. Abrí la puerta y me metí. Las gallinas gritaron, pero nadie las escuchó. Pude hacerme con un puñado de huevos colorados que aún no habían sido recogidos. Me senté detrás del gallinero apoyando la espalda contra una pared vieja y sucia, llena de telas de arañas. Tenía los pies cortados y ardidos. Escarbé en el bolsillo en busca del encendedor; al fin de cuentas, ya no tenía cigarrillos. Tomé un huevo por la parte superior, con las llemas de los dedos de mi mano izquierda y con la mano derecha accioné el mecanismo que generó el fuego. Lo sostuve así varios minutos, hasta que se puso negro y el calor cortante me alcanzó. Lo solté para volver a sostenerlo del otro lado. Uno se acostumbra al dolor que produce el calor, si se de a poco. Para cuando me pareció que el huevo estaba cocido, tenía todos los dedos llenos de ampollas y casi no podía sostenerlo. Tuve que romperlo con un pié y comermelo solo usando la boca, como los animales.

El siguiente huevo me lo comí crudo.

\end{document}
