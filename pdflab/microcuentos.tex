\documentclass[11pt,twoside,openright,a6paper]{book}

\usepackage[a6paper]{geometry}

\usepackage{times}
\usepackage[Lenny]{fncychap}
%\usepackage[Conny]{fncychap}

\usepackage[spanish]{babel}
\usepackage[utf8]{inputenc}
\usepackage[T1]{fontenc}

\newcommand{\personaje}{Nombre del personaje}

\title{Microcuentos 1}
\author{Juanjo Conti}
\date{}

\hyphenation{per-te-ne-cien-tes}
\hyphenation{ín-di-ce}
\hyphenation{co-ra-zón}
\hyphenation{via-je-ro}
\hyphenation{vi-kin-go}
\hyphenation{a-tra-ve-só}
\hyphenation{dio-ses}
\hyphenation{fe-rry}
\hyphenation{ca-mi-nan-do}
\hyphenation{e-le-va-do}
\hyphenation{ba-ta-lla}
\hyphenation{ber-me-llón}
\hyphenation{bo-te-llas}
\hyphenation{pro-ta-go-nis-ta}
\hyphenation{ar-chi-pié-la-go}
\hyphenation{li-bre-rí-a}

\begin{document}

\pagestyle{plain}

\maketitle

\tableofcontents

\chapter*{Justificación de los cuentos cortos}
\addcontentsline{toc}{chapter}{Justificación de los cuentos cortos}

Muchos de los cuentos que escribo son cortos. Esta cuestión anatómica de la literatura tiene una gran ventaja: si el cuento es bueno, en pocas líneas puede sacarte una sonrisa o provocarte un pensamiento profundo. Si el cuento es malo, no importa, en pocos segundos más termina.

Esta teoría, al igual que los cuentos cortos, goza de esta ventaja.


\part*{Sueños}
\addcontentsline{toc}{part}{Sueños}

\chapter*{Los eternos}
\addcontentsline{toc}{chapter}{Los eternos}

La primera vez que nos tocamos fue hace más de quinientos años en una playa en San Salvador, aunque ella llamaba al lugar Guanahaní. Se ocultó detrás de unas rocas sobre la arena cuando vio que una multitud de seres extraños salíamos del agua gritan y empujándonos. Los temores del viaje, las enfermedades, el hambre y el recuerdo de los que habíamos dejado en el mar fueron un catalizador que entró en acción cuando uno de los nosotros gritó “Tierra”.

\chapter*{El hombre que soñó con su gato\footnote{Publicado originalmente en mi blog}}
\addcontentsline{toc}{chapter}{El hombre que soñó con su gato}

Un hombre terminó de cenar, lavó los platos y sacó su gato al patio, para luego irse a dormir.
Mientras dormía, soñó que su gato lloraba en su puerta, se levantó y lo dejó entrar, para luego volver a dormir.
Mientras soñaba que dormía, soñó que su gato lloraba en su puerta, se levantó y lo dejó entrar, para luego volver a dormir.
Mientras soñaba que soñaba que dormía, soñó que su gato lloraba en su puerta, se levantó y lo dejó entrar, para luego volver a dormir.
Mientras soñaba que soñaba que soñaba que dormía, soñó que su gato lloraba en su puerta, se levantó y lo dejó entrar, para luego volver a dormir.
La secuencia se repitió 100 veces durante la noche.
Cuando se despertó a la mañana siguiente, casi se desmaya cuando vio lo que había en la cocina.

\chapter*{Se despertó}
\addcontentsline{toc}{chapter}{Se despertó}

Se despertó en el medio de la noche abrazado a su esposa. Se puso las pantuflas y camino hasta el baño. Cuando regresó, se vio durmiendo abrazado a su esposa. Por no despertarse, se fue a dormir al sofá.

\chapter*{Fantasmas}
\addcontentsline{toc}{chapter}{Fantasmas}

Unos niños fantasmas me señalaron y gritaron "¡un fantasma!".

\part*{Calle}
\addcontentsline{toc}{part}{Calle}

\chapter*{¿Qué comemos hoy?}
\addcontentsline{toc}{chapter}{¿Qué comemos hoy?}

---Papá, ¿qué comemos hoy?

La voz de su hija le llegó desde lejos mientras abría los ojos al despertar ese mediodía. Como tratando de entender dónde estaba, movió la cabeza para un lado y para el otro. Cuando se levantó, pateó unas botellas vacías de la noche anterior. Sí, había vuelto a emborracharse. Había jurado no volver a hacerlo, pero era la única forma que tenía de olvidar, al menos por un rato, el infierno en el que vivía. La casa destruida, los hijos sin madre, su inmensa soledad, el trabajo que nunca iba a conseguir. Respiró profundo y se puso de pie. Y cuando inhaló se percató de su presencia. Maldito perro y maldito sea su rancio olor. Ese olor que inundaba toda la casa. Y encima el perro se pone a aullar. Y ese ruido lo molesta, casi tanto como lo molesta el olor. Necesitaba explotar, de alguna forma. Va a intentar que las esquirlas sean pequeñas.
Y entonces su hija volvió a preguntar:

---Papá, ¿qué comemos hoy?

Maldición. Se volvió nuevamente al perro, que desde un rincón lo miraba con ojos tristes, ahora callado, como si en ese instante hubiese entendido que su destino estaba sellado.

---Hoy comemos carne, nena. Hoy comemos carne.

\chapter*{La nenita}
\addcontentsline{toc}{chapter}{La nenita}

Nueve de la noche, verano. En algún barrio de la ciudad.
Probablemente no en el centro ni en la costanera, sino en un barrio barrio. 
Una nenita de unos diez años avanza con una botella entre los brazos. 
Unos metros atrás, el kiosco de turno. 
La nenita camina, haciendo fuerza para que no se le caiga el encargue, más pesado que las muñecas con las que a veces juega. 
La nenita atraviesa la cuadra, con el porrón de cerveza entre las manos, recado de su papá, que la espera para apagar en su garganta el calor santafesino.

\chapter*{El pibe del semáforo}
\addcontentsline{toc}{chapter}{El pibe del semáforo}

Una cola de casi una cuadra ante el semáforo. Se acerca un pibe y me limpia el vidrio sin que se lo pida, queda más sucio que antes, se lo digo y se ofende. Ya tenía la moneda en la mano así que se la doy igual.

El gigante luminoso se pone verde, avanzamos despacio, casi a los empujones. No hay lugar. Cuando estoy a punto de cruzar se vuelve a poner en rojo.

El limpiavidrios vuelve a aparecer escondido atrás de una sonrisa de sin dientes y esta vez lo deja limpio.

Me vuelve a cobrar.

\end{document}
