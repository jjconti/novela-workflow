\documentclass[11pt,twoside,openright,a6paper]{book}

\usepackage[a6paper]{geometry}

\usepackage{times}
\usepackage[Lenny]{fncychap}
%\usepackage[Conny]{fncychap}

\usepackage[spanish]{babel}
\usepackage[utf8]{inputenc}
\usepackage[T1]{fontenc}

\newcommand{\personaje}{Nombre del personaje}

\usepackage{mathptmx}
\usepackage{etoolbox}

\usepackage[titles]{tocloft}

\renewcommand{\cftchapleader}{\cftdotfill{\cftdotsep}}

% change the space before the titles
\makeatletter
\patchcmd{\@makechapterhead}{\vspace*{50\p@}}{\vspace*{0pt}}{}{}
\patchcmd{\@makeschapterhead}{\vspace*{50\p@}}{\vspace*{0pt}}{}{}
\makeatother

% change the space after the titles
\renewcommand{\DOTI}[1]{%
    \raggedright
    \CTV\FmTi{#1}\par\nobreak
    \vskip 10pt}% original: 40pt
\renewcommand{\DOTIS}[1]{%
    \raggedright
    \CTV\FmTi{#1}\par\nobreak
    \vskip 10pt}% original: 40pt


\title{Cuentos de humor de cordura y de suerte}
\author{Juanjo Conti}
\date{}

\begin{document}

\pagestyle{plain}

\maketitle

\cleardoublepage

\thispagestyle{empty}
\noindent
Edición automágica. 2013.\\
Armado Mucho Ruido Muchas Nueces.

\vspace{0.5cm}

\noindent
\emph{Cuentos de humor de cordura y de suerte} lleva la licencia 
Creative Commons Attribution - NonCommercial - ShareAlike 3.0 Unported License.
Esto significa que podés compartir esta obra y crear obras derivadas de la misma
mencionando al autor, pero no ha\-cer un uso comercial.

\vfill

\noindent
Más información sobre este libro:\\
http://www.juanjoconti.com.ar/cuentos5\\

\noindent
Más libros del autor:\\
http://www.juanjoconti.com.ar/libros

\cleardoublepage

\noindent
\begin{flushright}
\emph{
\emph{Cuentos de humor de cordura y de suerte}\\
está dedicado a...
}
\end{flushright}

\cleardoublepage

\renewcommand*\contentsname{Índice}

\tableofcontents

\chapter*{Prólogo}
\addcontentsline{toc}{chapter}{Prólogo}

palabras... palabras... palabras... palabras... palabras... palabras... palabras... palabras... palabras... palabras... palabras... palabras... palabras... palabras... palabras... palabras... palabras... palabras... palabras... palabras... palabras... palabras... palabras... palabras... palabras... palabras... palabras... palabras... palabras... palabras... palabras... 

\chapter*{El hijo del escritor}
\addcontentsline{toc}{chapter}{El hijo del escritor}

\emph{El Dr. Martín Hogara observó a su hijo terminar de leer la última página de un grueso volumen. Complacido respiró profundamente.}

Martín Hogara Jr. tenía dieciseis años y era digno hijo de sus padres. El Dr. Martín Hogara era un escritor reconocido y su madre, profesora en la carrera Licenciatura en Letras de una prestigiosa universidad.
La verdad es que su hijo no había tenido siempre la actitud pulcra  y erudita hacia la literatura que hoy lo revestía. Menos de un año atrás podía contar con los dedos de su mano la cantidad de libros que había leído y no pensaba requerir de la otra en mucho tiempo.

Martín era el capitán del equipo de fútbol de su escuela, jugaba al básquetbol en un club de su barrio y practicaba judo. Los fines de semana salía a correr por la costa y una vez al mes se iba de pesca con su tío a un río cercano. Por su puesto, los veranos practicaba natación. En todos sus estilos.

No..., nada parecía demostrar que fuese a seguir los pasos de sus progenitores, y esto verdaderamente tenía preocupado a sus padres. La historia hubiera seguido su curso si no fuera por lo que aconteció en cierta ocación. En el verano de su décimo quinto cumpleaños, Martín se cayó del techo de su casa.

Era diciembre y su papá le había pedido ayuda para cambiar unas tejas del techo. El sol agobiante de verano al mediodía lo iluminaba desde arriba y en un momento empezó a sentirse mareado. Se le nubló la vista y de repente sintió un fuerte golpe, o al menos eso es lo que recordaba.

Cuando se despertó estaba en la cama de un hospital con sus padres a su alrededor. Tenía una pierna quebrada y vendas por todo el cuerpo. El yeso no le dejaba mover la pierna y, asustado, preguntó que le había pasado. Su padre con mucha calma lo tranquilizó y le explicó que se había mareado y caído del techo de la casa. Le dijo que no se preocupara, que en pocos días podría estar de regreso.

¡¿Qué no se preocupara?! Del sobresalto Martín casi calló de la cama. ¿Qué pasaría con su equipo de fútbol, con la natación y el resto de los deportes que practicaba? Este, sin duda, no sería un buen verano.

Cuando regresó a su casa encontró su habitación limpia como no había estado en años. Su mamá la había acomodado especialmente para él. No había ropa ni pelotas tiradas, la cama estaba tendida y por la ventana entraba una agradable luz natural.

Ese primer día en casa fue terrible. El médico le había mandado a quedarse en reposo por varias semanas y esto no le había hecho ninguna gracia. Todo lo contrario, lo tenía de muy mal humor.

Durante la tarde, su padre fue a verlo. Le explicó que no debía sentirse mal, que en cambio debía aprovechar esa situación para hacer algo diferente. En ese momento Martín notó que su papá cargaba bajo el brazo algunos libros.

—Libros no, papá...

—Se que no te gustan mucho Martín, pero de verdad pienso que deberías darle una oportunidad a éste. Se llama Un capitán de quince años y fue uno de los primeros libros que leí.

Martín miraba con desconfianza la cubierta del libro. Desde ella, un chico que debía tener más o menos su edad lo miraba desde un barco ballenero. Su padre, sin decir otra palabra, salió de la habitación y cerró la puerta.

Cuando su papá fue a visitarlo al otro día, Martín lo esperaba ansioso. El libro de Julio Verne le había fascinado y, con mostrada ansiedad, le pidió otros. Su padre le llevó clásicos del mismo autor como De la Tierra a la Luna, Viaje al centro de la Tierra y Veinte mil leguas de viaje submarino.

No pasó mucho tiempo hasta que Martín le pidió nuevos libros y conoció Los viajes de Gulliver, la planta de naranja lima de José Mauro de Vasconcelos y los vericuetos de Daniel Sempere en la Barcelona de los años treinta.

Las mañanas y las tardes se inundaban de palabras y, por las noches, comentaba con su papá las obras que había leído. Incluso su madre una tarde se animó y le acercó un libro de poemas. Más allá de la indiferencia con la que se había relacionado en su vida con la poesía, se encontró retrasando su cena para terminar de saborear los versos de un tal Pablo Neruda.

Pasaron las semanas y a medida que su cuerpo se fue sintiendo más fuerte, también se fortaleció su gusto por las letras. Pasó por autores clásicos y contemporáneos. Jorge Luis Borges y Julio Cortázar. Ciencia Ficción y Fantasía. Día a día fue descubriendo joyas en todos los géneros y tiempos, que su padre con buen ojo le sabía enseñar.

Así pasó Martín Hogara Jr. el verano en que cumplió quince años, descubriendo un mundo que hasta ese entonces desconocía. Cuando en marzo ya estaba recuperado salió corriendo de su casa y jugó un gran partido de fútbol; como hacía mucho tiempo no jugaba. Cuando volvió a su casa se bañó y luego cenó con sus padres. Antes de dormir, prendió su velador y empezó a leer una nueva recomendación de su padre, Rayuela. Seguía siendo un deportista, pero su vida había cambiado ese verano.

\emph{El Dr. Martín Hogara observó complacido a su hijo. Sí. Si retrocediera el tiempo, volvería a empujarlo desde el techo de su casa.}

\chapter*{El pelo en el jabón}
\addcontentsline{toc}{chapter}{El pelo en el jabón}

Probablemente un pelo en el jabón sea uno de los objetos más limpios del mundo. Sin embargo, cuando uno —con su cuerpo transpirado y los cabellos grasos— se dirige a la ducha para descargar ahí toda la mugre del día —del cuerpo y del alma— y se encuentra un pelo en el jabón... ¡Ah! que desazón y que sentimiento de violación a la intimidad de las gotas de agua que caen sobre el propio ser. Es que es tal la relación que uno tiene con el jabón, ese pan blanco protector y confidente, que el solo hecho de encontrar un pelo incrustado, cual fósil en piedra, nos recuerda que el vínculo que nos une a él no es inmaculado: hay más personas que frotan su cuerpo transpirado y sus cabellos grasos en él. Y entonces, entre parientes y amigos, empezamos a buscar sospechosos. Lo medimos, estudiamos su color, ¿rubio oscuro o castaño claro? En eso nos asalta otra pregunta y, tal vez, pista para encontrar al culpable. ¿De qué parte del cuerpo de ese vil rufián será el pelo? Demasiado corto para cabellera de mujer, demasiado largo para pelo de pierna de hombre. La cadena de deducciones se congela en el cerebro y el estómago se nos revuelve. Con las uñas y precisión quirúrgica nos animamos, lo sujetamos y lo retiramos de su soporte pastoso. Lo sostenemos ante nuestros ojos para examinarlo mejor. Reflexión. Una nueva inspección ocular. Parece que sí. Falsa alarma. Se trataba de un pedazo de hilo que se escapó del calzoncillo mientras lo lavábamos rasguñando su textil composición con el jabón la noche anterior. Ahora sí, fuera de peligro podemos bañarnos tranquilos. Pero... ¿qué sucede? El agua caliente se terminó.

\chapter*{Necesitás ir a la manicura, hija}
\addcontentsline{toc}{chapter}{Necesitás ir a la manicura, hija}

Cosas realmente cómicas suceden todos los días a plena luz. Como esa señora, muy sofisticada, vistiendo un trajecito de diseñador y sombrero de ala, que llevaba a su hija por La Capital de negocio en negocio mirando todo y no prestando atención a nada. En una mano llevaba una revista, de las inflexiones de los codos colgaban las bolsas de sus compras y en la otra mano tenía a su niña. Cuando llegaba a una esquina, y un semáforo detenía su galope, soltaba a su primogénita y tomaba el teléfono móvil. Con una habilidad que no habría adivinado, pasaba su pulgar izquierdo por toda la pantalla: miraba el mapa, chequeaba Facebook y contestaba SMSs. Cuando el semáforo la habilitaba, guardaba el teléfono, volvía a tomar a su hija y retomaba la marcha.

Cuando yo la vi, llegaba con la niña caminando de muy mala gana a la avenida 9 de Julio. Cuando frenaron la marcha, había unas doscientas personas esperando para cruzar. La mujer volvió a realizar el ritual y ocupó su mano izquierda respondiendo una invitación a jugar canasta por la tarde con otras señoras paquetas como ella. Lo gracioso ocurrió cuando quiso tomar a su hija y emprender nuevamente la marcha. En lugar de la suave mano de la pequeña, con toda confianza agarró una mano más áspera.

Ceferino Ortega sufre de enanismo, es desocupado y pide monedas por La Capital. A él también lo agarró por sorpresa, cuando de imprevisto sintió un tirón en el brazo y se lo llevaron arrastrando por la avenida más ancha del mundo. La señora no se dio cuenta hasta que llegó al otro extremo. A 140 metros, la pequeña se quedó mirando la escena con sus labios temblando a punto de largarse a llorar.

\chapter*{La convención}
\addcontentsline{toc}{chapter}{La convención}

El 22 de agosto de 2012 se llevó a cabo en la ciudad de Santa Fe una reunión muy particular. Esta tuvo lugar en el salón de convenciones del hotel Los Silos, ubicado en el puerto de la ciudad, junto al casino.

Si se lo miraba desde lo pisos superiores, el lugar era un campo de frutillas. Las caperuzas cubriendo las cabezas de todas las invitadas casi no dejaba ver el gris del mármol que pisaban.

Había caperucitas de alma y caperucitas de profesión. Había caperucitas quinceañeras y caperucitas que ya eran abuelas.
Estaba, por ejemplo, la reconocida actriz Lucía Vertucci, protagonista del éxito Caperucita y el motochorro. E incluso estaba Caperucita Rosa, cuyo nombre de bautismo era Carlos.

En total eran más de doscientos personificaciones del legendario personaje de rizos dorados y sonrisa inocente.

El objetivo de la convención era dividirse las rutas para ir a visitar a las correspondientes abuelitas, ya que en una ciudad tan chica, muchas se encontraban cuando tenían que perderse y eso no era ninguna gracia en los cuentos. No sabían quien era el organizador de semejante reunión, pero todas coincidían en que era una estupenda idea.

Cuando el maestro de ceremonia se disponía a hablar, la música bajó, las luces se atenuaron y todos hicieron silencio. El orador se aclaró la garganta tosiendo y su tos retumbó en toda la sala. Se acomodó las gafas sobre las puntiagudas orejas y tomó entre sus manos cuatro o cinco hojas de papel. No había empezado la primer oración de su discurso cuando caperucita, una, notó que las puertas se habían cerrado y que de la espalda del maestro de ceremonia asomaba una horrible, enorme y peluda cola de lobo.

\chapter*{De literatura, amor y cabello}
\addcontentsline{toc}{chapter}{De literatura, amor y cabello}

Durante esas semanas me encontraba yo completamente entregado a la lectura de una novela de un autor sueco, desconocido en el país. El volumen me lo había traído un primo mío en su último viaje, conociendo mi gusto por la buena literatura y las ediciones en rústica. Quiso el azar y, tal vez, el infortunio que por esos días mi cabellera alcanzara umbrales máximos que me impedían la correcta audición. Con cortinas de cabello como persianas sobre mis orejas, presto me fui caminando a la peluquería de mi amigo Chilín.

La peluquería de Chilín (que dicho sea de paso, ase es exactamente el texto que reza la cartelera fuera del local) goza de muy buena fama entre las señoras del barrio, por lo que, ateniéndome a la espera que me esperaba, me llevé bajo el brazo la copia de tan apasionante novela.

Una vez en el local y efectuados los saludos de rigor (mano y beso a Chilín, beso a su regordeta asistente, solo mano a su barbudo peluquero segundo), viendo que la cantidad de clientas en atención más las que esperaban por su permanente de vísperas de fin de semana superaba la media docena (aunque a estas señoras conviene venderlas por kilo que por unidad), me senté en uno de los sillones a esperar mi turno. Frente a mi había una mesa ratona que me ofrecía 2 tipos de revistas, de chimentos y espectáculos o revistas con fotos de cortes de cabello. Ninguna de las dos me interesaba, por lo que me consagré a la lectura que tenía entre manos (o más precisamente, bajo el brazo).

Pasaba una a una las páginas, humedeciéndome levemente el anular derecho más por mala costumbre que por necesidad hidromecánica. Estaba en el medio de un diálogo sublime entre el personaje principal y su enemigo de turno cuando oí la voz de Chilín que me llamaba. Alcé la mirada y mi molestia se debió notar en la comisura de mis ojos, pero al fin de cuentas era yo quien había concurrido al recinto para recibir un servicio. Sin chistar dejé el libro sobre la mesita ratona y me senté frente al espejo.

La charla con Chilín fue la típica, empezamos por el clima, pasamos al partido de la última fecha, nos quejamos de los políticos de turno, para finalmente llegar a mi tema de conversación preferido, los acontecimientos sociales del barrio. Me puso al día con los nuevos concubinatos, embarazos inesperados, infidelidades y demás delicias. En realidad no me importaban los quienes, siempre viví en un termo y prácticamente no conozco a nadie, pero amaba la pasión con la que Chilín contaba las aventuras y desventuras. El hombre era un profesional y si cambiaba los nombres de los personajes de las historias que me relataba, a doña Chola por algún nombre victoriano, parecía que el mismo Shakespeare me estaba cortando el pelo.

Los pasos finales de su arte siempre eran los mismos, pincel para retirar los sobrantes de su obra, agua de colonia, un jopo bien chapado a la antigua que ya conocía a fuerza de que me pregunte mil veces como me peinaba y a la calle.

Luego de efectuar el pago, enfilé hacia la mesa ratona para tomar mi libro y marcharme cuando noté que no estaba donde lo había dejado. Invadido por unos nervios galopantes y una repentina taquicardia también equina, empecé a revolear los ojos por todo el salón. La angustia duró poco. Me calmé cuando lo vi en las manos de una persona.

La muchacha vestía un rojo escarlata y era dueña de una singular belleza, pero lo que realmente me cautivó fue la voracidad con la que leía. Sus ojos verdes barrían una a una las páginas y en la comisura de su boca se podía percibir pasión y deseo.

Tomé carrera para ir hasta ella y desproveerla del objeto de su fascinación, pero justo cuando me disponía a hacerlo vi tal belleza en el brillo de sus ojos, en la forma en que los cabellos rubios le caían detrás de la oreja, en sus labios que se movían silencioso al ritmo de las palabras, que no pude hacerlo.

Simplemente la dejé leyendo, conocedor de la felicidad que estaba experimentando. Al fin de cuentas, soy un romántico.

\chapter*{El primer romántico}
\addcontentsline{toc}{chapter}{El primer romántico}

En una aldea al lado de un río, hace cientos de miles de años, vivía Urgh. Le habían puesto ese nombre porque ese había sido el ruido que hizo su madre cuando Urgh salió de entre sus piernas. El idioma era sencillo por esos días. Por ejemplo, “dolor” se decía “arrgh” porque ese era el quejido que alguien había hecho cuando se le cayó una piedra en el pie.

Así como el idioma era sencillo, las relaciones entre humanos también lo eran. De chico Urgh había visto cómo trataba su papá a su mamá y en la adolescencia intentó hacer lo mismo. A pesar del hecho de que sus compañeros lo practicaban regularmente y comentaban las satisfacciones obtenidas, a Urgh le hacía un poco de ruido eso de elegir una hembra del montón, pegarle un garrotazo en la cabeza, arrastrarla de los pelos hasta el interior de su cueva y en la oscuridad poseerla.

Lo había intentado un par de veces, pero el resultado nunca había sido como lo esperaba. Una vez golpeó muy despacio y la hembra se despertó mientras era arrastrada, lo mordió y huyó corriendo. Otra vez golpeó muy duro y la hembra no se despertó más. Tenía que haber una técnica mejor.

Urgh estaba cavilando estas ideas, sentado, con un pie en el río, cuando Eigh se le acercó y empezó a beber agua. La miró un buen rato hasta que se animó a hacer la pregunta que tenía en su cabeza desde hacía tiempo.

Eigh, ¿a ustedes les gusta que les peguemos en la cabeza?

Eigh lo miró atónita, como si hubiese dicho una blasfemia, pero luego se inclinó para un costado, como si estuviera pensando y le contestó.

No, la verdad que no. Por lo menos a mí no. Creo que a mi mamá y a mis hermanas tampoco.

¿Y qué te gusta, Eigh?

No se... las flores. Las de color lila.

Urgh se fue corriendo y en menos de un minuto regresó con un racimo de flores lilas mezcladas con un poco de pasto y tierra que había arrancado. Le extendió el brazo a Eigh y se las dio. Tomá, para vos.

Eigh desconfió un poco, pero luego las tomó. Se las acercó a la nariz y las olfateó. Se sonrió y un color rojo le brotó de las pálidas mejillas. Miró a Urgh y volvió a sonreír. En ese momento Eigh sintió un incontenible deseo de saltar encima de Urgh. Pero no lo hizo.

Bastante bien, le contestó. Los hombres tienen que ser así, dulces.

¿Dulces?, ¿como las naranjas?, preguntó Urgh y se chupó el antebrazo para investigar a qué sabía.

No, no dulces así. Suaves.

¿Como los conejos?

¡No, Urgh! Suaves, así. Y acarició la mejilla peluda de su compañero con el dorso de su mano.

Urgh se puso muy nervioso e instintivamente tanteó el suelo en busca de su garrote.

Eigh se dio cuenta de estos movimientos e instintivamente le tomó la mano y se la apoyó en su pecho. Los dos cavernícolas se miraron y se sonrieron. Luego se fundieron y fue la primera vez que Urgh poseyó a una hembra. Eigh por su parte, había disfrutado de esa miel de la que solo había probado unas gotas cuando se despertaba de los garrotazos. Esa noche se fueron a vivir a la misma cueva. Urgh se había convertido en el primer romántico.

Al otro día, Urgh, fascinado con su descubrimiento, no veía la hora de seguir experimentando. Inventó nuevos regalos y conquistó a muchas jovencitas del valle. Collares de diente de sable, bocadillos de mamut y perfumes de distintas flores que no servían para comer. Su índice de conquistas era tan superior al de sus amigos garroteros que pronto todos empezaron a pedirle consejos. Ese día incluso se fueron a una aldea vecina a probar sus técnicas. Urgh pasó de ser el primer romántico a ser el primer casanova. 

Volvió a su cueva muy entrada la noche, cansado y extasiado. Eigh lo esperaba en la puerta con el garrote que él había dejado.

\chapter*{La ucraniana}
\addcontentsline{toc}{chapter}{La ucraniana}


Ayer de pasada escuché a dos mozos hablar sobre el agujero de la cerradura de
una de las habitaciones. No alcancé a escuchar exactamente a qué habitación
se referían, pero decodifique un “setecientos cuarenta y...” y vi que uno
de los dos hacía gestos obscenos mientras el otro se reía a carcajadas. Sin
nada mejor que hacer un día nublado y con lloviznas en toda la península de
Yucatán, dejé el área de desayuno y me fui caminando hasta el bloque siete.

\hyphenation{pa-si-llo}
Una vez en el bloque siete, subí las escaleras hasta el piso cuatro y,
mirando el amplio pasillo, me puse a pensar cuál de las puertas sería el
custodio del entretenimiento de los mozos. Sin nada que me de una señal,
empecé a apoyar mi oreja en cada una de las puertas.

Estaba apoyado bajo un lustroso número 744 cuando la superficie que me
sostenía se esfumó. La puerta se había abierto y yo caí, despatarrado
y haciendo mucho ruido, a los pies de una señorita de zapatos rojos y
taco aguja.

\hyphenation{se-ña-la-ba}
La chica que me miraba desde su altura más los veinte centímetros de
los tacos me empezó a hablar en una lengua que no reconocí. Me levantó
del hombro (era muy fuerte, sumado a la pista del idioma, adivino rusa o
ucraniana, medallista olímpica de lanzamiento de martillo, ¿por qué
no?). Evidentemente se pensó que yo era personal de maestranza o algo
parecido, porque con unas palabras que adiviné como insultos me señalaba
su reloj y a los empujones me llevó hasta el baño de la habitación donde
cataratas de agua emanaban por todos los elementos de grifería.

\hyphenation{chan-gas} \hyphenation{des-en-ros-can-do} Acostumbrado a
arreglármelas y a hacer changas, me arremangué para revisarle la cañería a
la señorita. Por favor absténganse de malos pensamientos que esta no es ese
tipo de historias. Estuve escarbando, destrabando, soplando, desenroscando,
girando, palpando, doblando, estirando, sacudiendo, estrujando y reensamblando
alrededor de media hora hasta que en una de las secciones de la cañería
encontré, atascado, cien kilos de papel higiénico y descartables de aseo
personal varios condensados en una pelota de diez centímetros de diámetro.

Como un héroe que rescata una mascota, se la mostré a la ucraniana (ya la
había bautizado así en mis pensamientos) y me hizo cara de asco. Tiré
la pelota en el cesto de la basura y procedía a lavarme las manos, los
antebrazos, los codos, los brazos y hasta la nuca, era una limpieza sin fin.
Una vez presto y rechinando de limpio, volví mi atención a la señorita. El
agua ya no brotaba a borbotones y su pesadilla cloacal parecía llegar a su
fin. Me tomó de las manos y vi en sus ojos el deseo de agradecerme. Casi
lágrimas le provocaba la emoción y la impotencia de no poder hablarme en
mi idioma natal. Miró con el rabillo del ojo la cama matrimonial y a mi
se me aceleró el corazón. En ese momento escuché la llave, el picaporte
y la puerta.

\hyphenation{cua-tro}
\hyphenation{ce-rra-du-ra}
No miento si digo que podría ser doble de riesgo de Schwarzenegger. El
marido de la ucraniana eran tan ancho de espalda como yo de alto. Tenía
una malla celeste con flores verde fluor y un toallón colgado de la
espalda. Cuando me vió, el rostro se le reconfiguró y empezó a gritarle a
su mujer. Puteadas en ucraniano, supongo, obolonka tvoyei sestry, ty suka,
podyvit'sya na mudak Ty shcho , zhartuyesh. Después me miró a mí. Tenía
la piel roja por el calor del momento y me mostró su dentadura lobuna. Voy
a resumir lo siguiente que pasó diciendo que me hizo todo lo que yo le
hice a las cañerías de su habitación, pero con menos delicadeza. Cuando
pude escaparme y salir de la setecientos cuarenta y cuatro me encontré a
los mozos espiando por la cerradura, muertos de risa.

\end{document}
