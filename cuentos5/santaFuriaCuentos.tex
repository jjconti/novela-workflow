 \documentclass[12pt,twoside,openright,a5paper]{book}

\usepackage[a5paper, hmarginratio={3:2}]{geometry}

\usepackage{times}
\usepackage[Lenny]{fncychap}
%\usepackage[Conny]{fncychap}

\usepackage[spanish]{babel}
\usepackage[utf8]{inputenc}
\usepackage[T1]{fontenc}

\newcommand{\personaje}{Nombre del personaje}

\usepackage{mathptmx}
\usepackage{etoolbox}

\usepackage[titles]{tocloft}

\usepackage{pdfpages}

\renewcommand{\cftchapleader}{\cftdotfill{\cftdotsep}}

% change the space before the titles
\makeatletter
\patchcmd{\@makechapterhead}{\vspace*{50\p@}}{\vspace*{0pt}}{}{}
\patchcmd{\@makeschapterhead}{\vspace*{50\p@}}{\vspace*{0pt}}{}{}
\makeatother

% change the space after the titles
\renewcommand{\DOTI}[1]{%
    \raggedright
    \CTV\FmTi{#1}\par\nobreak
    \vskip 10pt}% original: 40pt
\renewcommand{\DOTIS}[1]{%
    \raggedright
    \CTV\FmTi{#1}\par\nobreak
    \vskip 10pt}% original: 40pt


\title{Santa Furia}
\author{Juanjo Conti}
\date{}


% Evitar viudas y huérfanas
\widowpenalty=10000
\clubpenalty=10000

\begin{document}

\pagestyle{plain}

\maketitle

%\cleardoublepage

\thispagestyle{empty}
\noindent
Edición automágica, 2014.\\

\vspace{0.5cm}

\noindent
\emph{Santa Furia} lleva la licencia 
\emph{Creative Commons Attribution - NonCommercial - ShareAlike 3.0 Unported License}.
Esto significa que podés compartir esta obra y crear obras derivadas
mencionando al autor, pero no ha\-cer un uso comercial de ella.

\vfill

\noindent
%Más información sobre este libro:\\
http://www.juanjoconti.com.ar/cuentos5\\

\noindent
%Más libros del autor:\\
http://www.juanjoconti.com.ar/libros

\cleardoublepage

\noindent
\begin{flushright}
\emph{
\emph{Santa Furia}\\
está dedicado a mi amigo Joel,\\
quien me enseñó que mejor que sembrar cizaña\\
es cortarla con una guadaña.
}
\end{flushright}

\cleardoublepage

\renewcommand*\contentsname{Índice}

\tableofcontents

\chapter*{Prólogo\footnote{Por Lis Gariglio}}

%Crear un libro de cuentos se parece a tallar una escultura. 
%Hay que tomar un pedazo de materia, un bodoque, roca o madera 
%(en este caso, los cuentos que escribí durante 2012 y 2013)
%y luego sacar todo lo que sobra.

Juanjo Conti 
%, a través de un homenaje a Cortázar, 
echa a andar el hilo del relato y logra enlazar todas estas maravillosas historias que convergen en el corazón del libro, en un único cuento, 
%que es 
«Santa Furia». Este, como si fuese un espejo, refleja las distintas temáticas sobre las que se habla en cada trama narrativa. El tema principal es la delgada línea que existe entre lo real y lo no real, entre la vigilia y el sueño, que en 
%«Santa Furia» 
el cuento que da título al libro
desaparece a causa de la violencia. No obstante, a veces es la misma realidad la que genera violencia, 
%que está 
presente en cada relato, en forma explícita o disfrazada de metáfora, pero siempre agazapada en el tejido del texto. Por eso, si se observa con atención, la estructura de la colección no es casual. Porque, así como el personaje del primer cuento entra en el departamento para adueñarse una vez más de él, el lector se sumerge en estas páginas para apropiarse de cada relato y, a través de su lectura, los vuelve a escribir al activar su sentido y su significación. La violencia irá en aumento hasta alcanzar su punto máximo en «Santa Furia» y luego decrece para dar lugar, en los últimos cuentos, al halo misterioso de la muerte que, al igual que cierra el ciclo de la vida, cierra aquí %el proceso narrativo que desarrolló Juajno Conti durante 2012 y 2013.
esta selección de cuentos escritos durante 2012 y 2013.


%Más allá de las temáticas que comparten estas historias, todas nos vuelven a confirmar la genialidad del autor y las marca de su estilo y su escritura: ese final inesperado al que ya nos tiene acostumbrados, un final que no agota el suspenso y que 
%Una selección de cuentos que
%nos demuestra una vez que la lectura es siempre una aventura placentera. 

\input{elDepartamento.title} \input{elDepartamento.txt}
\input{fantasmas.title} \input{fantasmas.txt}
\input{hombreYGato.title} \input{hombreYGato.txt}
\input{barba.title} \input{barba.txt}
\input{joel.title} \input{joel.txt}
\input{naranjas.title} \input{naranjas.txt}
\input{santaFuria.title} \input{santaFuria.txt}
\input{bermellon.title} \input{bermellon.txt}
\input{coccion.title} \input{coccion.txt}
\input{laConvencion.title} \input{laConvencion.txt}
\input{encuentroDominical.title} \input{encuentroDominical.txt}
\input{jazminYVainilla.title} \input{jazminYVainilla.txt}
\input{seDesperto.title} \input{seDesperto.txt}
\input{dosPalabras.title} \input{dosPalabras.txt}

\includepdf{empty.pdf}

\end{document}
