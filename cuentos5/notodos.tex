 \documentclass[12pt,twoside,openright,a5paper]{book}

\usepackage[a5paper]{geometry}

\usepackage{times}
\usepackage[Lenny]{fncychap}
%\usepackage[Conny]{fncychap}

\usepackage[spanish]{babel}
\usepackage[utf8]{inputenc}
\usepackage[T1]{fontenc}

\newcommand{\personaje}{Nombre del personaje}

\usepackage{mathptmx}
\usepackage{etoolbox}

\usepackage[titles]{tocloft}

\renewcommand{\cftchapleader}{\cftdotfill{\cftdotsep}}

% change the space before the titles
\makeatletter
\patchcmd{\@makechapterhead}{\vspace*{50\p@}}{\vspace*{0pt}}{}{}
\patchcmd{\@makeschapterhead}{\vspace*{50\p@}}{\vspace*{0pt}}{}{}
\makeatother

% change the space after the titles
\renewcommand{\DOTI}[1]{%
    \raggedright
    \CTV\FmTi{#1}\par\nobreak
    \vskip 10pt}% original: 40pt
\renewcommand{\DOTIS}[1]{%
    \raggedright
    \CTV\FmTi{#1}\par\nobreak
    \vskip 10pt}% original: 40pt


\title{Cuentos elgidos 2013-2014}
\author{Juanjo Conti}
\date{}

\begin{document}

\pagestyle{plain}

\maketitle

\cleardoublepage

\thispagestyle{empty}
\noindent
Edición automágica, 2014.\\

\vspace{0.5cm}

\noindent
\emph{Cuentos elgidos 2013-2014} lleva la licencia 
\emph{Creative Commons Attribution - NonCommercial - ShareAlike 3.0 Unported License}.
Esto significa que podés compartir esta obra y crear obras derivadas
mencionando al autor, pero no ha\-cer un uso comercial de ella.

\vfill

\noindent
Más información sobre este libro:\\
http://www.juanjoconti.com.ar/cuentos5\\

\noindent
Más libros del autor:\\
http://www.juanjoconti.com.ar/libros

\cleardoublepage

\noindent
\begin{flushright}
\emph{
\emph{Santa Furia}\\
está dedicado a mi amigo Joel,\\
quien me enseñó que mejor que sembrar cizaña\\
es cortarla con una guadaña.
}
\end{flushright}

\cleardoublepage

\renewcommand*\contentsname{Índice}

\tableofcontents

\chapter*{Prologo}


\input{encuentroDominical.title} \input{encuentroDominical.txt}
\input{fantasmas.title} \input{fantasmas.txt}
\input{hombreYGato.title} \input{hombreYGato.txt}
\input{naranjas.title} \input{naranjas.txt}
\input{jazminYVainilla.title} \input{jazminYVainilla.txt}
\input{barba.title} \input{barba.txt}
\input{joel.title} \input{joel.txt}
\input{bermellon.title} \input{bermellon.txt}
 \documentclass[12pt,twoside,openright,a5paper]{book}

\usepackage[a5paper]{geometry}

\usepackage{times}
\usepackage[Lenny]{fncychap}
%\usepackage[Conny]{fncychap}

\usepackage[spanish]{babel}
\usepackage[utf8]{inputenc}
\usepackage[T1]{fontenc}

\newcommand{\personaje}{Nombre del personaje}

\usepackage{mathptmx}
\usepackage{etoolbox}

\usepackage[titles]{tocloft}

\renewcommand{\cftchapleader}{\cftdotfill{\cftdotsep}}

% change the space before the titles
\makeatletter
\patchcmd{\@makechapterhead}{\vspace*{50\p@}}{\vspace*{0pt}}{}{}
\patchcmd{\@makeschapterhead}{\vspace*{50\p@}}{\vspace*{0pt}}{}{}
\makeatother

% change the space after the titles
\renewcommand{\DOTI}[1]{%
    \raggedright
    \CTV\FmTi{#1}\par\nobreak
    \vskip 10pt}% original: 40pt
\renewcommand{\DOTIS}[1]{%
    \raggedright
    \CTV\FmTi{#1}\par\nobreak
    \vskip 10pt}% original: 40pt


\title{Santa Furia}
\author{Juanjo Conti}
\date{}

\begin{document}

\pagestyle{plain}

\maketitle

\cleardoublepage

\thispagestyle{empty}
\noindent
Edición automágica, 2014.\\

\vspace{0.5cm}

\noindent
\emph{Santa Furia} lleva la licencia 
\emph{Creative Commons Attribution - NonCommercial - ShareAlike 3.0 Unported License}.
Esto significa que podés compartir esta obra y crear obras derivadas
mencionando al autor, pero no ha\-cer un uso comercial de ella.

\vfill

\noindent
Más información sobre este libro:\\
http://www.juanjoconti.com.ar/cuentos5\\

\noindent
Más libros del autor:\\
http://www.juanjoconti.com.ar/libros

\cleardoublepage

\noindent
\begin{flushright}
\emph{
\emph{Santa Furia}\\
está dedicado a...
}
\end{flushright}

\cleardoublepage

\renewcommand*\contentsname{Índice}

\tableofcontents

\input{bermellon.title} \input{bermellon.txt}
\input{elDepartamento.title} \input{elDepartamento.txt}
\input{encuentroDominical.title} \input{encuentroDominical.txt}
\input{queComemosHoy.title} \input{queComemosHoy.txt}
\input{asadoReyesCensurado.title} \input{asadoReyesCensurado.txt}
\input{barba.title} \input{barba.txt}
\input{laNenita.title} \input{laNenita.txt}
\input{naranjas.title} \input{naranjas.txt}
\input{librosMarcados.title} \input{librosMarcados.txt}
\input{dosPalabras.title} \input{dosPalabras.txt}
 \documentclass[12pt,twoside,openright,a5paper]{book}

\usepackage[a5paper]{geometry}

\usepackage{times}
\usepackage[Lenny]{fncychap}
%\usepackage[Conny]{fncychap}

\usepackage[spanish]{babel}
\usepackage[utf8]{inputenc}
\usepackage[T1]{fontenc}

\newcommand{\personaje}{Nombre del personaje}

\usepackage{mathptmx}
\usepackage{etoolbox}

\usepackage[titles]{tocloft}

\renewcommand{\cftchapleader}{\cftdotfill{\cftdotsep}}

% change the space before the titles
\makeatletter
\patchcmd{\@makechapterhead}{\vspace*{50\p@}}{\vspace*{0pt}}{}{}
\patchcmd{\@makeschapterhead}{\vspace*{50\p@}}{\vspace*{0pt}}{}{}
\makeatother

% change the space after the titles
\renewcommand{\DOTI}[1]{%
    \raggedright
    \CTV\FmTi{#1}\par\nobreak
    \vskip 10pt}% original: 40pt
\renewcommand{\DOTIS}[1]{%
    \raggedright
    \CTV\FmTi{#1}\par\nobreak
    \vskip 10pt}% original: 40pt


\title{Santa Furia}
\author{Juanjo Conti}
\date{}

\begin{document}

\pagestyle{plain}

\maketitle

\cleardoublepage

\thispagestyle{empty}
\noindent
Edición automágica, 2014.\\

\vspace{0.5cm}

\noindent
\emph{Santa Furia} lleva la licencia 
\emph{Creative Commons Attribution - NonCommercial - ShareAlike 3.0 Unported License}.
Esto significa que podés compartir esta obra y crear obras derivadas
mencionando al autor, pero no ha\-cer un uso comercial de ella.

\vfill

\noindent
Más información sobre este libro:\\
http://www.juanjoconti.com.ar/cuentos5\\

\noindent
Más libros del autor:\\
http://www.juanjoconti.com.ar/libros

\cleardoublepage

\noindent
\begin{flushright}
\emph{
\emph{Santa Furia}\\
está dedicado a...
}
\end{flushright}

\cleardoublepage

\renewcommand*\contentsname{Índice}

\tableofcontents

\input{bermellon.title} \input{bermellon.txt}
\input{elDepartamento.title} \input{elDepartamento.txt}
\input{encuentroDominical.title} \input{encuentroDominical.txt}
\input{queComemosHoy.title} \input{queComemosHoy.txt}
\input{asadoReyesCensurado.title} \input{asadoReyesCensurado.txt}
\input{barba.title} \input{barba.txt}
\input{laNenita.title} \input{laNenita.txt}
\input{naranjas.title} \input{naranjas.txt}
\input{librosMarcados.title} \input{librosMarcados.txt}
\input{dosPalabras.title} \input{dosPalabras.txt}
 \documentclass[12pt,twoside,openright,a5paper]{book}

\usepackage[a5paper]{geometry}

\usepackage{times}
\usepackage[Lenny]{fncychap}
%\usepackage[Conny]{fncychap}

\usepackage[spanish]{babel}
\usepackage[utf8]{inputenc}
\usepackage[T1]{fontenc}

\newcommand{\personaje}{Nombre del personaje}

\usepackage{mathptmx}
\usepackage{etoolbox}

\usepackage[titles]{tocloft}

\renewcommand{\cftchapleader}{\cftdotfill{\cftdotsep}}

% change the space before the titles
\makeatletter
\patchcmd{\@makechapterhead}{\vspace*{50\p@}}{\vspace*{0pt}}{}{}
\patchcmd{\@makeschapterhead}{\vspace*{50\p@}}{\vspace*{0pt}}{}{}
\makeatother

% change the space after the titles
\renewcommand{\DOTI}[1]{%
    \raggedright
    \CTV\FmTi{#1}\par\nobreak
    \vskip 10pt}% original: 40pt
\renewcommand{\DOTIS}[1]{%
    \raggedright
    \CTV\FmTi{#1}\par\nobreak
    \vskip 10pt}% original: 40pt


\title{Santa Furia}
\author{Juanjo Conti}
\date{}

\begin{document}

\pagestyle{plain}

\maketitle

\cleardoublepage

\thispagestyle{empty}
\noindent
Edición automágica, 2014.\\

\vspace{0.5cm}

\noindent
\emph{Santa Furia} lleva la licencia 
\emph{Creative Commons Attribution - NonCommercial - ShareAlike 3.0 Unported License}.
Esto significa que podés compartir esta obra y crear obras derivadas
mencionando al autor, pero no ha\-cer un uso comercial de ella.

\vfill

\noindent
Más información sobre este libro:\\
http://www.juanjoconti.com.ar/cuentos5\\

\noindent
Más libros del autor:\\
http://www.juanjoconti.com.ar/libros

\cleardoublepage

\noindent
\begin{flushright}
\emph{
\emph{Santa Furia}\\
está dedicado a...
}
\end{flushright}

\cleardoublepage

\renewcommand*\contentsname{Índice}

\tableofcontents

\input{bermellon.title} \input{bermellon.txt}
\input{elDepartamento.title} \input{elDepartamento.txt}
\input{encuentroDominical.title} \input{encuentroDominical.txt}
\input{queComemosHoy.title} \input{queComemosHoy.txt}
\input{asadoReyesCensurado.title} \input{asadoReyesCensurado.txt}
\input{barba.title} \input{barba.txt}
\input{laNenita.title} \input{laNenita.txt}
\input{naranjas.title} \input{naranjas.txt}
\input{librosMarcados.title} \input{librosMarcados.txt}
\input{dosPalabras.title} \input{dosPalabras.txt}
\input{santaFuria.title} \input{santaFuria.txt}


\end{document}
  \documentclass[12pt,twoside,openright,a5paper]{book}

\usepackage[a5paper]{geometry}

\usepackage{times}
\usepackage[Lenny]{fncychap}
%\usepackage[Conny]{fncychap}

\usepackage[spanish]{babel}
\usepackage[utf8]{inputenc}
\usepackage[T1]{fontenc}

\newcommand{\personaje}{Nombre del personaje}

\usepackage{mathptmx}
\usepackage{etoolbox}

\usepackage[titles]{tocloft}

\renewcommand{\cftchapleader}{\cftdotfill{\cftdotsep}}

% change the space before the titles
\makeatletter
\patchcmd{\@makechapterhead}{\vspace*{50\p@}}{\vspace*{0pt}}{}{}
\patchcmd{\@makeschapterhead}{\vspace*{50\p@}}{\vspace*{0pt}}{}{}
\makeatother

% change the space after the titles
\renewcommand{\DOTI}[1]{%
    \raggedright
    \CTV\FmTi{#1}\par\nobreak
    \vskip 10pt}% original: 40pt
\renewcommand{\DOTIS}[1]{%
    \raggedright
    \CTV\FmTi{#1}\par\nobreak
    \vskip 10pt}% original: 40pt


\title{Santa Furia}
\author{Juanjo Conti}
\date{}

\begin{document}

\pagestyle{plain}

\maketitle

\cleardoublepage

\thispagestyle{empty}
\noindent
Edición automágica, 2014.\\

\vspace{0.5cm}

\noindent
\emph{Santa Furia} lleva la licencia 
\emph{Creative Commons Attribution - NonCommercial - ShareAlike 3.0 Unported License}.
Esto significa que podés compartir esta obra y crear obras derivadas
mencionando al autor, pero no ha\-cer un uso comercial de ella.

\vfill

\noindent
Más información sobre este libro:\\
http://www.juanjoconti.com.ar/cuentos5\\

\noindent
Más libros del autor:\\
http://www.juanjoconti.com.ar/libros

\cleardoublepage

\noindent
\begin{flushright}
\emph{
\emph{Santa Furia}\\
está dedicado a...
}
\end{flushright}

\cleardoublepage

\renewcommand*\contentsname{Índice}

\tableofcontents

\input{bermellon.title} \input{bermellon.txt}
\input{elDepartamento.title} \input{elDepartamento.txt}
\input{encuentroDominical.title} \input{encuentroDominical.txt}
\input{queComemosHoy.title} \input{queComemosHoy.txt}
\input{asadoReyesCensurado.title} \input{asadoReyesCensurado.txt}
\input{barba.title} \input{barba.txt}
\input{laNenita.title} \input{laNenita.txt}
\input{naranjas.title} \input{naranjas.txt}
\input{librosMarcados.title} \input{librosMarcados.txt}
\input{dosPalabras.title} \input{dosPalabras.txt}
\input{santaFuria.title} \input{santaFuria.txt}


\end{document}



\end{document}
  \documentclass[12pt,twoside,openright,a5paper]{book}

\usepackage[a5paper]{geometry}

\usepackage{times}
\usepackage[Lenny]{fncychap}
%\usepackage[Conny]{fncychap}

\usepackage[spanish]{babel}
\usepackage[utf8]{inputenc}
\usepackage[T1]{fontenc}

\newcommand{\personaje}{Nombre del personaje}

\usepackage{mathptmx}
\usepackage{etoolbox}

\usepackage[titles]{tocloft}

\renewcommand{\cftchapleader}{\cftdotfill{\cftdotsep}}

% change the space before the titles
\makeatletter
\patchcmd{\@makechapterhead}{\vspace*{50\p@}}{\vspace*{0pt}}{}{}
\patchcmd{\@makeschapterhead}{\vspace*{50\p@}}{\vspace*{0pt}}{}{}
\makeatother

% change the space after the titles
\renewcommand{\DOTI}[1]{%
    \raggedright
    \CTV\FmTi{#1}\par\nobreak
    \vskip 10pt}% original: 40pt
\renewcommand{\DOTIS}[1]{%
    \raggedright
    \CTV\FmTi{#1}\par\nobreak
    \vskip 10pt}% original: 40pt


\title{Santa Furia}
\author{Juanjo Conti}
\date{}

\begin{document}

\pagestyle{plain}

\maketitle

\cleardoublepage

\thispagestyle{empty}
\noindent
Edición automágica, 2014.\\

\vspace{0.5cm}

\noindent
\emph{Santa Furia} lleva la licencia 
\emph{Creative Commons Attribution - NonCommercial - ShareAlike 3.0 Unported License}.
Esto significa que podés compartir esta obra y crear obras derivadas
mencionando al autor, pero no ha\-cer un uso comercial de ella.

\vfill

\noindent
Más información sobre este libro:\\
http://www.juanjoconti.com.ar/cuentos5\\

\noindent
Más libros del autor:\\
http://www.juanjoconti.com.ar/libros

\cleardoublepage

\noindent
\begin{flushright}
\emph{
\emph{Santa Furia}\\
está dedicado a...
}
\end{flushright}

\cleardoublepage

\renewcommand*\contentsname{Índice}

\tableofcontents

\input{bermellon.title} \input{bermellon.txt}
\input{elDepartamento.title} \input{elDepartamento.txt}
\input{encuentroDominical.title} \input{encuentroDominical.txt}
\input{queComemosHoy.title} \input{queComemosHoy.txt}
\input{asadoReyesCensurado.title} \input{asadoReyesCensurado.txt}
\input{barba.title} \input{barba.txt}
\input{laNenita.title} \input{laNenita.txt}
\input{naranjas.title} \input{naranjas.txt}
\input{librosMarcados.title} \input{librosMarcados.txt}
\input{dosPalabras.title} \input{dosPalabras.txt}
 \documentclass[12pt,twoside,openright,a5paper]{book}

\usepackage[a5paper]{geometry}

\usepackage{times}
\usepackage[Lenny]{fncychap}
%\usepackage[Conny]{fncychap}

\usepackage[spanish]{babel}
\usepackage[utf8]{inputenc}
\usepackage[T1]{fontenc}

\newcommand{\personaje}{Nombre del personaje}

\usepackage{mathptmx}
\usepackage{etoolbox}

\usepackage[titles]{tocloft}

\renewcommand{\cftchapleader}{\cftdotfill{\cftdotsep}}

% change the space before the titles
\makeatletter
\patchcmd{\@makechapterhead}{\vspace*{50\p@}}{\vspace*{0pt}}{}{}
\patchcmd{\@makeschapterhead}{\vspace*{50\p@}}{\vspace*{0pt}}{}{}
\makeatother

% change the space after the titles
\renewcommand{\DOTI}[1]{%
    \raggedright
    \CTV\FmTi{#1}\par\nobreak
    \vskip 10pt}% original: 40pt
\renewcommand{\DOTIS}[1]{%
    \raggedright
    \CTV\FmTi{#1}\par\nobreak
    \vskip 10pt}% original: 40pt


\title{Santa Furia}
\author{Juanjo Conti}
\date{}

\begin{document}

\pagestyle{plain}

\maketitle

\cleardoublepage

\thispagestyle{empty}
\noindent
Edición automágica, 2014.\\

\vspace{0.5cm}

\noindent
\emph{Santa Furia} lleva la licencia 
\emph{Creative Commons Attribution - NonCommercial - ShareAlike 3.0 Unported License}.
Esto significa que podés compartir esta obra y crear obras derivadas
mencionando al autor, pero no ha\-cer un uso comercial de ella.

\vfill

\noindent
Más información sobre este libro:\\
http://www.juanjoconti.com.ar/cuentos5\\

\noindent
Más libros del autor:\\
http://www.juanjoconti.com.ar/libros

\cleardoublepage

\noindent
\begin{flushright}
\emph{
\emph{Santa Furia}\\
está dedicado a...
}
\end{flushright}

\cleardoublepage

\renewcommand*\contentsname{Índice}

\tableofcontents

\input{bermellon.title} \input{bermellon.txt}
\input{elDepartamento.title} \input{elDepartamento.txt}
\input{encuentroDominical.title} \input{encuentroDominical.txt}
\input{queComemosHoy.title} \input{queComemosHoy.txt}
\input{asadoReyesCensurado.title} \input{asadoReyesCensurado.txt}
\input{barba.title} \input{barba.txt}
\input{laNenita.title} \input{laNenita.txt}
\input{naranjas.title} \input{naranjas.txt}
\input{librosMarcados.title} \input{librosMarcados.txt}
\input{dosPalabras.title} \input{dosPalabras.txt}
\input{santaFuria.title} \input{santaFuria.txt}


\end{document}
  \documentclass[12pt,twoside,openright,a5paper]{book}

\usepackage[a5paper]{geometry}

\usepackage{times}
\usepackage[Lenny]{fncychap}
%\usepackage[Conny]{fncychap}

\usepackage[spanish]{babel}
\usepackage[utf8]{inputenc}
\usepackage[T1]{fontenc}

\newcommand{\personaje}{Nombre del personaje}

\usepackage{mathptmx}
\usepackage{etoolbox}

\usepackage[titles]{tocloft}

\renewcommand{\cftchapleader}{\cftdotfill{\cftdotsep}}

% change the space before the titles
\makeatletter
\patchcmd{\@makechapterhead}{\vspace*{50\p@}}{\vspace*{0pt}}{}{}
\patchcmd{\@makeschapterhead}{\vspace*{50\p@}}{\vspace*{0pt}}{}{}
\makeatother

% change the space after the titles
\renewcommand{\DOTI}[1]{%
    \raggedright
    \CTV\FmTi{#1}\par\nobreak
    \vskip 10pt}% original: 40pt
\renewcommand{\DOTIS}[1]{%
    \raggedright
    \CTV\FmTi{#1}\par\nobreak
    \vskip 10pt}% original: 40pt


\title{Santa Furia}
\author{Juanjo Conti}
\date{}

\begin{document}

\pagestyle{plain}

\maketitle

\cleardoublepage

\thispagestyle{empty}
\noindent
Edición automágica, 2014.\\

\vspace{0.5cm}

\noindent
\emph{Santa Furia} lleva la licencia 
\emph{Creative Commons Attribution - NonCommercial - ShareAlike 3.0 Unported License}.
Esto significa que podés compartir esta obra y crear obras derivadas
mencionando al autor, pero no ha\-cer un uso comercial de ella.

\vfill

\noindent
Más información sobre este libro:\\
http://www.juanjoconti.com.ar/cuentos5\\

\noindent
Más libros del autor:\\
http://www.juanjoconti.com.ar/libros

\cleardoublepage

\noindent
\begin{flushright}
\emph{
\emph{Santa Furia}\\
está dedicado a...
}
\end{flushright}

\cleardoublepage

\renewcommand*\contentsname{Índice}

\tableofcontents

\input{bermellon.title} \input{bermellon.txt}
\input{elDepartamento.title} \input{elDepartamento.txt}
\input{encuentroDominical.title} \input{encuentroDominical.txt}
\input{queComemosHoy.title} \input{queComemosHoy.txt}
\input{asadoReyesCensurado.title} \input{asadoReyesCensurado.txt}
\input{barba.title} \input{barba.txt}
\input{laNenita.title} \input{laNenita.txt}
\input{naranjas.title} \input{naranjas.txt}
\input{librosMarcados.title} \input{librosMarcados.txt}
\input{dosPalabras.title} \input{dosPalabras.txt}
\input{santaFuria.title} \input{santaFuria.txt}


\end{document}



\end{document}



\end{document}
  \documentclass[12pt,twoside,openright,a5paper]{book}

\usepackage[a5paper]{geometry}

\usepackage{times}
\usepackage[Lenny]{fncychap}
%\usepackage[Conny]{fncychap}

\usepackage[spanish]{babel}
\usepackage[utf8]{inputenc}
\usepackage[T1]{fontenc}

\newcommand{\personaje}{Nombre del personaje}

\usepackage{mathptmx}
\usepackage{etoolbox}

\usepackage[titles]{tocloft}

\renewcommand{\cftchapleader}{\cftdotfill{\cftdotsep}}

% change the space before the titles
\makeatletter
\patchcmd{\@makechapterhead}{\vspace*{50\p@}}{\vspace*{0pt}}{}{}
\patchcmd{\@makeschapterhead}{\vspace*{50\p@}}{\vspace*{0pt}}{}{}
\makeatother

% change the space after the titles
\renewcommand{\DOTI}[1]{%
    \raggedright
    \CTV\FmTi{#1}\par\nobreak
    \vskip 10pt}% original: 40pt
\renewcommand{\DOTIS}[1]{%
    \raggedright
    \CTV\FmTi{#1}\par\nobreak
    \vskip 10pt}% original: 40pt


\title{Santa Furia}
\author{Juanjo Conti}
\date{}

\begin{document}

\pagestyle{plain}

\maketitle

\cleardoublepage

\thispagestyle{empty}
\noindent
Edición automágica, 2014.\\

\vspace{0.5cm}

\noindent
\emph{Santa Furia} lleva la licencia 
\emph{Creative Commons Attribution - NonCommercial - ShareAlike 3.0 Unported License}.
Esto significa que podés compartir esta obra y crear obras derivadas
mencionando al autor, pero no ha\-cer un uso comercial de ella.

\vfill

\noindent
Más información sobre este libro:\\
http://www.juanjoconti.com.ar/cuentos5\\

\noindent
Más libros del autor:\\
http://www.juanjoconti.com.ar/libros

\cleardoublepage

\noindent
\begin{flushright}
\emph{
\emph{Santa Furia}\\
está dedicado a...
}
\end{flushright}

\cleardoublepage

\renewcommand*\contentsname{Índice}

\tableofcontents

\input{bermellon.title} \input{bermellon.txt}
\input{elDepartamento.title} \input{elDepartamento.txt}
\input{encuentroDominical.title} \input{encuentroDominical.txt}
\input{queComemosHoy.title} \input{queComemosHoy.txt}
\input{asadoReyesCensurado.title} \input{asadoReyesCensurado.txt}
\input{barba.title} \input{barba.txt}
\input{laNenita.title} \input{laNenita.txt}
\input{naranjas.title} \input{naranjas.txt}
\input{librosMarcados.title} \input{librosMarcados.txt}
\input{dosPalabras.title} \input{dosPalabras.txt}
 \documentclass[12pt,twoside,openright,a5paper]{book}

\usepackage[a5paper]{geometry}

\usepackage{times}
\usepackage[Lenny]{fncychap}
%\usepackage[Conny]{fncychap}

\usepackage[spanish]{babel}
\usepackage[utf8]{inputenc}
\usepackage[T1]{fontenc}

\newcommand{\personaje}{Nombre del personaje}

\usepackage{mathptmx}
\usepackage{etoolbox}

\usepackage[titles]{tocloft}

\renewcommand{\cftchapleader}{\cftdotfill{\cftdotsep}}

% change the space before the titles
\makeatletter
\patchcmd{\@makechapterhead}{\vspace*{50\p@}}{\vspace*{0pt}}{}{}
\patchcmd{\@makeschapterhead}{\vspace*{50\p@}}{\vspace*{0pt}}{}{}
\makeatother

% change the space after the titles
\renewcommand{\DOTI}[1]{%
    \raggedright
    \CTV\FmTi{#1}\par\nobreak
    \vskip 10pt}% original: 40pt
\renewcommand{\DOTIS}[1]{%
    \raggedright
    \CTV\FmTi{#1}\par\nobreak
    \vskip 10pt}% original: 40pt


\title{Santa Furia}
\author{Juanjo Conti}
\date{}

\begin{document}

\pagestyle{plain}

\maketitle

\cleardoublepage

\thispagestyle{empty}
\noindent
Edición automágica, 2014.\\

\vspace{0.5cm}

\noindent
\emph{Santa Furia} lleva la licencia 
\emph{Creative Commons Attribution - NonCommercial - ShareAlike 3.0 Unported License}.
Esto significa que podés compartir esta obra y crear obras derivadas
mencionando al autor, pero no ha\-cer un uso comercial de ella.

\vfill

\noindent
Más información sobre este libro:\\
http://www.juanjoconti.com.ar/cuentos5\\

\noindent
Más libros del autor:\\
http://www.juanjoconti.com.ar/libros

\cleardoublepage

\noindent
\begin{flushright}
\emph{
\emph{Santa Furia}\\
está dedicado a...
}
\end{flushright}

\cleardoublepage

\renewcommand*\contentsname{Índice}

\tableofcontents

\input{bermellon.title} \input{bermellon.txt}
\input{elDepartamento.title} \input{elDepartamento.txt}
\input{encuentroDominical.title} \input{encuentroDominical.txt}
\input{queComemosHoy.title} \input{queComemosHoy.txt}
\input{asadoReyesCensurado.title} \input{asadoReyesCensurado.txt}
\input{barba.title} \input{barba.txt}
\input{laNenita.title} \input{laNenita.txt}
\input{naranjas.title} \input{naranjas.txt}
\input{librosMarcados.title} \input{librosMarcados.txt}
\input{dosPalabras.title} \input{dosPalabras.txt}
 \documentclass[12pt,twoside,openright,a5paper]{book}

\usepackage[a5paper]{geometry}

\usepackage{times}
\usepackage[Lenny]{fncychap}
%\usepackage[Conny]{fncychap}

\usepackage[spanish]{babel}
\usepackage[utf8]{inputenc}
\usepackage[T1]{fontenc}

\newcommand{\personaje}{Nombre del personaje}

\usepackage{mathptmx}
\usepackage{etoolbox}

\usepackage[titles]{tocloft}

\renewcommand{\cftchapleader}{\cftdotfill{\cftdotsep}}

% change the space before the titles
\makeatletter
\patchcmd{\@makechapterhead}{\vspace*{50\p@}}{\vspace*{0pt}}{}{}
\patchcmd{\@makeschapterhead}{\vspace*{50\p@}}{\vspace*{0pt}}{}{}
\makeatother

% change the space after the titles
\renewcommand{\DOTI}[1]{%
    \raggedright
    \CTV\FmTi{#1}\par\nobreak
    \vskip 10pt}% original: 40pt
\renewcommand{\DOTIS}[1]{%
    \raggedright
    \CTV\FmTi{#1}\par\nobreak
    \vskip 10pt}% original: 40pt


\title{Santa Furia}
\author{Juanjo Conti}
\date{}

\begin{document}

\pagestyle{plain}

\maketitle

\cleardoublepage

\thispagestyle{empty}
\noindent
Edición automágica, 2014.\\

\vspace{0.5cm}

\noindent
\emph{Santa Furia} lleva la licencia 
\emph{Creative Commons Attribution - NonCommercial - ShareAlike 3.0 Unported License}.
Esto significa que podés compartir esta obra y crear obras derivadas
mencionando al autor, pero no ha\-cer un uso comercial de ella.

\vfill

\noindent
Más información sobre este libro:\\
http://www.juanjoconti.com.ar/cuentos5\\

\noindent
Más libros del autor:\\
http://www.juanjoconti.com.ar/libros

\cleardoublepage

\noindent
\begin{flushright}
\emph{
\emph{Santa Furia}\\
está dedicado a...
}
\end{flushright}

\cleardoublepage

\renewcommand*\contentsname{Índice}

\tableofcontents

\input{bermellon.title} \input{bermellon.txt}
\input{elDepartamento.title} \input{elDepartamento.txt}
\input{encuentroDominical.title} \input{encuentroDominical.txt}
\input{queComemosHoy.title} \input{queComemosHoy.txt}
\input{asadoReyesCensurado.title} \input{asadoReyesCensurado.txt}
\input{barba.title} \input{barba.txt}
\input{laNenita.title} \input{laNenita.txt}
\input{naranjas.title} \input{naranjas.txt}
\input{librosMarcados.title} \input{librosMarcados.txt}
\input{dosPalabras.title} \input{dosPalabras.txt}
\input{santaFuria.title} \input{santaFuria.txt}


\end{document}
  \documentclass[12pt,twoside,openright,a5paper]{book}

\usepackage[a5paper]{geometry}

\usepackage{times}
\usepackage[Lenny]{fncychap}
%\usepackage[Conny]{fncychap}

\usepackage[spanish]{babel}
\usepackage[utf8]{inputenc}
\usepackage[T1]{fontenc}

\newcommand{\personaje}{Nombre del personaje}

\usepackage{mathptmx}
\usepackage{etoolbox}

\usepackage[titles]{tocloft}

\renewcommand{\cftchapleader}{\cftdotfill{\cftdotsep}}

% change the space before the titles
\makeatletter
\patchcmd{\@makechapterhead}{\vspace*{50\p@}}{\vspace*{0pt}}{}{}
\patchcmd{\@makeschapterhead}{\vspace*{50\p@}}{\vspace*{0pt}}{}{}
\makeatother

% change the space after the titles
\renewcommand{\DOTI}[1]{%
    \raggedright
    \CTV\FmTi{#1}\par\nobreak
    \vskip 10pt}% original: 40pt
\renewcommand{\DOTIS}[1]{%
    \raggedright
    \CTV\FmTi{#1}\par\nobreak
    \vskip 10pt}% original: 40pt


\title{Santa Furia}
\author{Juanjo Conti}
\date{}

\begin{document}

\pagestyle{plain}

\maketitle

\cleardoublepage

\thispagestyle{empty}
\noindent
Edición automágica, 2014.\\

\vspace{0.5cm}

\noindent
\emph{Santa Furia} lleva la licencia 
\emph{Creative Commons Attribution - NonCommercial - ShareAlike 3.0 Unported License}.
Esto significa que podés compartir esta obra y crear obras derivadas
mencionando al autor, pero no ha\-cer un uso comercial de ella.

\vfill

\noindent
Más información sobre este libro:\\
http://www.juanjoconti.com.ar/cuentos5\\

\noindent
Más libros del autor:\\
http://www.juanjoconti.com.ar/libros

\cleardoublepage

\noindent
\begin{flushright}
\emph{
\emph{Santa Furia}\\
está dedicado a...
}
\end{flushright}

\cleardoublepage

\renewcommand*\contentsname{Índice}

\tableofcontents

\input{bermellon.title} \input{bermellon.txt}
\input{elDepartamento.title} \input{elDepartamento.txt}
\input{encuentroDominical.title} \input{encuentroDominical.txt}
\input{queComemosHoy.title} \input{queComemosHoy.txt}
\input{asadoReyesCensurado.title} \input{asadoReyesCensurado.txt}
\input{barba.title} \input{barba.txt}
\input{laNenita.title} \input{laNenita.txt}
\input{naranjas.title} \input{naranjas.txt}
\input{librosMarcados.title} \input{librosMarcados.txt}
\input{dosPalabras.title} \input{dosPalabras.txt}
\input{santaFuria.title} \input{santaFuria.txt}


\end{document}



\end{document}
  \documentclass[12pt,twoside,openright,a5paper]{book}

\usepackage[a5paper]{geometry}

\usepackage{times}
\usepackage[Lenny]{fncychap}
%\usepackage[Conny]{fncychap}

\usepackage[spanish]{babel}
\usepackage[utf8]{inputenc}
\usepackage[T1]{fontenc}

\newcommand{\personaje}{Nombre del personaje}

\usepackage{mathptmx}
\usepackage{etoolbox}

\usepackage[titles]{tocloft}

\renewcommand{\cftchapleader}{\cftdotfill{\cftdotsep}}

% change the space before the titles
\makeatletter
\patchcmd{\@makechapterhead}{\vspace*{50\p@}}{\vspace*{0pt}}{}{}
\patchcmd{\@makeschapterhead}{\vspace*{50\p@}}{\vspace*{0pt}}{}{}
\makeatother

% change the space after the titles
\renewcommand{\DOTI}[1]{%
    \raggedright
    \CTV\FmTi{#1}\par\nobreak
    \vskip 10pt}% original: 40pt
\renewcommand{\DOTIS}[1]{%
    \raggedright
    \CTV\FmTi{#1}\par\nobreak
    \vskip 10pt}% original: 40pt


\title{Santa Furia}
\author{Juanjo Conti}
\date{}

\begin{document}

\pagestyle{plain}

\maketitle

\cleardoublepage

\thispagestyle{empty}
\noindent
Edición automágica, 2014.\\

\vspace{0.5cm}

\noindent
\emph{Santa Furia} lleva la licencia 
\emph{Creative Commons Attribution - NonCommercial - ShareAlike 3.0 Unported License}.
Esto significa que podés compartir esta obra y crear obras derivadas
mencionando al autor, pero no ha\-cer un uso comercial de ella.

\vfill

\noindent
Más información sobre este libro:\\
http://www.juanjoconti.com.ar/cuentos5\\

\noindent
Más libros del autor:\\
http://www.juanjoconti.com.ar/libros

\cleardoublepage

\noindent
\begin{flushright}
\emph{
\emph{Santa Furia}\\
está dedicado a...
}
\end{flushright}

\cleardoublepage

\renewcommand*\contentsname{Índice}

\tableofcontents

\input{bermellon.title} \input{bermellon.txt}
\input{elDepartamento.title} \input{elDepartamento.txt}
\input{encuentroDominical.title} \input{encuentroDominical.txt}
\input{queComemosHoy.title} \input{queComemosHoy.txt}
\input{asadoReyesCensurado.title} \input{asadoReyesCensurado.txt}
\input{barba.title} \input{barba.txt}
\input{laNenita.title} \input{laNenita.txt}
\input{naranjas.title} \input{naranjas.txt}
\input{librosMarcados.title} \input{librosMarcados.txt}
\input{dosPalabras.title} \input{dosPalabras.txt}
 \documentclass[12pt,twoside,openright,a5paper]{book}

\usepackage[a5paper]{geometry}

\usepackage{times}
\usepackage[Lenny]{fncychap}
%\usepackage[Conny]{fncychap}

\usepackage[spanish]{babel}
\usepackage[utf8]{inputenc}
\usepackage[T1]{fontenc}

\newcommand{\personaje}{Nombre del personaje}

\usepackage{mathptmx}
\usepackage{etoolbox}

\usepackage[titles]{tocloft}

\renewcommand{\cftchapleader}{\cftdotfill{\cftdotsep}}

% change the space before the titles
\makeatletter
\patchcmd{\@makechapterhead}{\vspace*{50\p@}}{\vspace*{0pt}}{}{}
\patchcmd{\@makeschapterhead}{\vspace*{50\p@}}{\vspace*{0pt}}{}{}
\makeatother

% change the space after the titles
\renewcommand{\DOTI}[1]{%
    \raggedright
    \CTV\FmTi{#1}\par\nobreak
    \vskip 10pt}% original: 40pt
\renewcommand{\DOTIS}[1]{%
    \raggedright
    \CTV\FmTi{#1}\par\nobreak
    \vskip 10pt}% original: 40pt


\title{Santa Furia}
\author{Juanjo Conti}
\date{}

\begin{document}

\pagestyle{plain}

\maketitle

\cleardoublepage

\thispagestyle{empty}
\noindent
Edición automágica, 2014.\\

\vspace{0.5cm}

\noindent
\emph{Santa Furia} lleva la licencia 
\emph{Creative Commons Attribution - NonCommercial - ShareAlike 3.0 Unported License}.
Esto significa que podés compartir esta obra y crear obras derivadas
mencionando al autor, pero no ha\-cer un uso comercial de ella.

\vfill

\noindent
Más información sobre este libro:\\
http://www.juanjoconti.com.ar/cuentos5\\

\noindent
Más libros del autor:\\
http://www.juanjoconti.com.ar/libros

\cleardoublepage

\noindent
\begin{flushright}
\emph{
\emph{Santa Furia}\\
está dedicado a...
}
\end{flushright}

\cleardoublepage

\renewcommand*\contentsname{Índice}

\tableofcontents

\input{bermellon.title} \input{bermellon.txt}
\input{elDepartamento.title} \input{elDepartamento.txt}
\input{encuentroDominical.title} \input{encuentroDominical.txt}
\input{queComemosHoy.title} \input{queComemosHoy.txt}
\input{asadoReyesCensurado.title} \input{asadoReyesCensurado.txt}
\input{barba.title} \input{barba.txt}
\input{laNenita.title} \input{laNenita.txt}
\input{naranjas.title} \input{naranjas.txt}
\input{librosMarcados.title} \input{librosMarcados.txt}
\input{dosPalabras.title} \input{dosPalabras.txt}
\input{santaFuria.title} \input{santaFuria.txt}


\end{document}
  \documentclass[12pt,twoside,openright,a5paper]{book}

\usepackage[a5paper]{geometry}

\usepackage{times}
\usepackage[Lenny]{fncychap}
%\usepackage[Conny]{fncychap}

\usepackage[spanish]{babel}
\usepackage[utf8]{inputenc}
\usepackage[T1]{fontenc}

\newcommand{\personaje}{Nombre del personaje}

\usepackage{mathptmx}
\usepackage{etoolbox}

\usepackage[titles]{tocloft}

\renewcommand{\cftchapleader}{\cftdotfill{\cftdotsep}}

% change the space before the titles
\makeatletter
\patchcmd{\@makechapterhead}{\vspace*{50\p@}}{\vspace*{0pt}}{}{}
\patchcmd{\@makeschapterhead}{\vspace*{50\p@}}{\vspace*{0pt}}{}{}
\makeatother

% change the space after the titles
\renewcommand{\DOTI}[1]{%
    \raggedright
    \CTV\FmTi{#1}\par\nobreak
    \vskip 10pt}% original: 40pt
\renewcommand{\DOTIS}[1]{%
    \raggedright
    \CTV\FmTi{#1}\par\nobreak
    \vskip 10pt}% original: 40pt


\title{Santa Furia}
\author{Juanjo Conti}
\date{}

\begin{document}

\pagestyle{plain}

\maketitle

\cleardoublepage

\thispagestyle{empty}
\noindent
Edición automágica, 2014.\\

\vspace{0.5cm}

\noindent
\emph{Santa Furia} lleva la licencia 
\emph{Creative Commons Attribution - NonCommercial - ShareAlike 3.0 Unported License}.
Esto significa que podés compartir esta obra y crear obras derivadas
mencionando al autor, pero no ha\-cer un uso comercial de ella.

\vfill

\noindent
Más información sobre este libro:\\
http://www.juanjoconti.com.ar/cuentos5\\

\noindent
Más libros del autor:\\
http://www.juanjoconti.com.ar/libros

\cleardoublepage

\noindent
\begin{flushright}
\emph{
\emph{Santa Furia}\\
está dedicado a...
}
\end{flushright}

\cleardoublepage

\renewcommand*\contentsname{Índice}

\tableofcontents

\input{bermellon.title} \input{bermellon.txt}
\input{elDepartamento.title} \input{elDepartamento.txt}
\input{encuentroDominical.title} \input{encuentroDominical.txt}
\input{queComemosHoy.title} \input{queComemosHoy.txt}
\input{asadoReyesCensurado.title} \input{asadoReyesCensurado.txt}
\input{barba.title} \input{barba.txt}
\input{laNenita.title} \input{laNenita.txt}
\input{naranjas.title} \input{naranjas.txt}
\input{librosMarcados.title} \input{librosMarcados.txt}
\input{dosPalabras.title} \input{dosPalabras.txt}
\input{santaFuria.title} \input{santaFuria.txt}


\end{document}



\end{document}



\end{document}

\input{coccion.title} \input{coccion.txt}
\input{laConvencion.title} \input{laConvencion.txt}
\input{seDesperto.title} \input{seDesperto.txt}
\input{dosPalabras.title} \input{dosPalabras.txt}
\input{elDepartamento.title} \input{elDepartamento.txt}


\end{document}
