\documentclass[12pt,twoside,openright,a5paper]{book}

\usepackage[a5paper]{geometry}

\usepackage{times}
\usepackage[Lenny]{fncychap}
%\usepackage[Conny]{fncychap}

\usepackage[spanish]{babel}
\usepackage[utf8]{inputenc}
\usepackage[T1]{fontenc}

\newcommand{\personaje}{Nombre del personaje}

\usepackage{mathptmx}
\usepackage{etoolbox}

\usepackage[titles]{tocloft}

\renewcommand{\cftchapleader}{\cftdotfill{\cftdotsep}}

% change the space before the titles
\makeatletter
\patchcmd{\@makechapterhead}{\vspace*{50\p@}}{\vspace*{0pt}}{}{}
\patchcmd{\@makeschapterhead}{\vspace*{50\p@}}{\vspace*{0pt}}{}{}
\makeatother

% change the space after the titles
\renewcommand{\DOTI}[1]{%
    \raggedright
    \CTV\FmTi{#1}\par\nobreak
    \vskip 10pt}% original: 40pt
\renewcommand{\DOTIS}[1]{%
    \raggedright
    \CTV\FmTi{#1}\par\nobreak
    \vskip 10pt}% original: 40pt


\title{Contiscopio}
\author{Juanjo Conti}
\date{}

\begin{document}

\pagestyle{plain}

\maketitle

\cleardoublepage

\thispagestyle{empty}
\noindent
Edición automágica, 2014.\\

\vspace{0.5cm}

\noindent
\emph{Contiscopio} lleva la licencia 
\emph{Creative Commons Attribution - NonCommercial - ShareAlike 3.0 Unported License}.
Esto significa que podés compartir esta obra y crear obras derivadas
mencionando al autor, pero no ha\-cer un uso comercial de ella.

\vfill

\noindent
Más información sobre este libro:\\
http://www.juanjoconti.com.ar/cuentos5\\

\noindent
Más libros del autor:\\
http://www.juanjoconti.com.ar/libros

\cleardoublepage

\noindent
\begin{flushright}
\emph{
\emph{Contiscopio}\\
está dedicado a...
}
\end{flushright}

\cleardoublepage

\renewcommand*\contentsname{Índice}

\tableofcontents

\chapter*{Prólogo} \addcontentsline{toc}{chapter}{Prólogo}

El asado de los Reyes fue escrito en colaboración con un demonio llamado Sebastian D. Criado.

Dos palabras está basado en un poema de Alfonsina Storni.

\part*{Anécdotas mejoradas} \addcontentsline{toc}{part}{Anécdotas mejoradas}

\chapter*{Joel} \addcontentsline{toc}{chapter}{Joel}

\input{joel.txt}

\chapter*{Sobre la existencia\\ de los fantasmas} \addcontentsline{toc}{chapter}{Sobre la existencia de los fantasmas}

\input{fantasmas.txt}

\chapter*{La caja}
\addcontentsline{toc}{chapter}{La caja}

\input{laCaja.txt}

\chapter*{Barba}
\addcontentsline{toc}{chapter}{Barba}

\input{barba.txt}

\chapter*{El asado de los Reyes}
\addcontentsline{toc}{chapter}{El asado de los Reyes}

\input{asadoReyes.txt}


\part*{Historias de animales} \addcontentsline{toc}{part}{Historias de animales}

\chapter*{Jazmín y vainilla} \addcontentsline{toc}{chapter}{Jazmín y vainilla}

\input{jazminYVainilla.txt}

\chapter*{Vidas}
\addcontentsline{toc}{chapter}{Vidas}

\input{vidas.txt}

\chapter*{Bermellón}
\addcontentsline{toc}{chapter}{Bermellón}

\input{bermellon.txt}

\chapter*{El hombre que soñó\\ con su gato}
\addcontentsline{toc}{chapter}{El hombre que soñó con su gato}

\input{hombreYGato.txt}

\chapter*{¿Qué comemos hoy?}
\addcontentsline{toc}{chapter}{¿Qué comemos hoy?}

\input{queComemosHoy.txt}

\chapter*{Cocción de un huevo}
\addcontentsline{toc}{chapter}{Cocción de un huevo}

\input{coccion.txt}





\part*{Juegos oníricos} \addcontentsline{toc}{part}{Juegos oníricos}

\chapter*{Banco}
\addcontentsline{toc}{chapter}{Banco}

\input{banco.txt}

\chapter*{25}
\addcontentsline{toc}{chapter}{25}

\input{25.txt}


\chapter*{Los eternos}
\addcontentsline{toc}{chapter}{Los eternos}

\input{losEternos.txt}


\chapter*{La convención}
\addcontentsline{toc}{chapter}{La convención}

\input{laConvencion.txt}




\part*{Historias de dos\\ que no son uno} \addcontentsline{toc}{part}{Historias de dos que no son uno}

\chapter*{El primer romántico} \addcontentsline{toc}{chapter}{El primer romántico}

\input{primerRomantico.txt}

\chapter*{El departamento} \addcontentsline{toc}{chapter}{El departamento}

\input{elDepartamento.txt}

\chapter*{Libros marcados} \addcontentsline{toc}{chapter}{Libros marcados}

\input{librosMarcados.txt}

\chapter*{Dos palabras} \addcontentsline{toc}{chapter}{Dos palabras}

\input{dosPalabras.txt}

\chapter*{Encuentro dominical} \addcontentsline{toc}{chapter}{Encuentro dominical}

\input{encuentroDominical.txt}

\chapter*{Mameluco} \addcontentsline{toc}{chapter}{Mameluco}

\input{mameluco.txt}

\chapter*{Se despertó} \addcontentsline{toc}{chapter}{Se despertó}

\input{seDesperto.txt}

\chapter*{Naranjas para don Bordesio} \addcontentsline{toc}{chapter}{Naranjas para don Bordesio}

\input{naranjas.txt}




\end{document}
